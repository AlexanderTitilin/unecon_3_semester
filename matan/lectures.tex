\documentclass[14pt]{extarticle}
\usepackage{fontspec}
\usepackage[russian, english]{babel}
\setmainfont{Times New Roman}
\usepackage{amssymb}
\usepackage{amsthm}
\usepackage{setspace}
\onehalfspacing
\usepackage{amsmath}
\usepackage{listings}
\usepackage{indentfirst}
\setlength{\parindent}{1.25cm}
\usepackage[right=10mm,left=30mm,top=20mm,bottom=20mm]{geometry}
\newtheorem{theorem}{Теорема}
\newtheorem{corollary}{Следствие}[theorem]
\newtheorem{lemma}[theorem]{Лемма}
\title{Лекции по математическому анализу 3 семестр}
\author{}
\date{}
\begin{document}
\maketitle
\section{Функции нескольких вещественных переменных}
\[
	X \subset \mathbb{R}^{n}
	.\]
\[
	x = (x_1,x_2,\dots,x_{n}) \in X
	.\]
\[
	x \rightarrow f(x)
	.\]
\[
	f(x_1,x_2) = x_1^2 + 3x_2
	.\]
\[
	z = x^{2} + 3y
	.\]
\[
	f : \mathbb{R}^n \to \mathbb{R}
	.\]
\section{Замкнутый промежуток в n-мерном пространстве}
\[
	a = (a_1,\dots,a_{n})
	.\]
\[
	b = (b_1,\dots,b_{n})
	.\]
\[
	x = (x_1,\dots,x_{n})
	.\]
\[
	\forall  i ~ 1 \le  i \le  n ~ a_{i} \le  x_{i} \le  b_{i}
	.\]
% \section{Обобщение круга}
% Замкнутый круг -- множество всех точек, с одинаковым расстоянием до данной
\section{Окрестность}
Окрестность точки -- открытый шар с центром в этой точке
\section{Внутренняя точка}
\[
X \subset \mathbb{R}^{n}
.\] 
\[
a \in \mathbb{R}^{n}
.\] 
a называется внутренней точкой множества $X$, если $\exists  r  B(a,r) \subset X$
\section{Внешняя точка}
a называется внешней точкой по отношению к множеству $X$, если  $\exists  B(a,r)$ $\subset \mathbb{R}^{n} \setminus X$.
\section{Граничная точка}
Если любая граница точки содердит точки и из множества и не оттуда.
\section{Обозначения шаров}
 \[
     B(a,r) = \{x \in \mathbb{R}^{n} \mid \rho(x,a) < R\}
.\] 
\[
     \overline{B}(a,r) = \{x \in \mathbb{R}^{n} \mid \rho(x,a) \le  R\}
.\] 
Множество называется открытым, если всего его точки внутренние
\section{Открытое множество}
Множество A открыто в  X, если
\[
A = X \cap U
.\] 
U - открытое множество
\begin{theorem}
    Пересечение двух открытых множеств является открытым.
\end{theorem}
\begin{proof}
    Надо доказать, что все точки $U \cap V$ внутренние.\\
    \[
    a \in X
    .\] 
    а внутренняя точка множетсва U, значит  $\exists r_1  ~ B(a,r_1) \subset U$\\
    а внутренняя точка множетсва V, значит  $\exists r_2  ~ B(a,r_2) \subset V$
    \[
        r = \min{r_1,r_2}
    .\] 
    \[
    B(a,r) \subset U \cap V
    .\] 
\end{proof}
\begin{theorem}
    \[
        \{U_{\alpha}\}_{\alpha \in I}
    .\] 
    \[
    U_{\alpha} \in R^{n}
    .\] 
    пусть $\forall  \alpha$ $U_{\alpha}$ открытое
    тогда $\bigcup_{\alpha \in I} U_{\alpha}$ открыто
\end{theorem}
\begin{theorem}
    \[
    F \subset \mathbb{R}^{n}
    .\] 
    F - замкнуто $\iff$  $\mathbb{R}^{n} \setminus F$ - открытое
\end{theorem}
\begin{theorem}
    \begin{enumerate}
        \item пересечение любого числа замкнутых множеств являкется замкнутым
        \item объединение конечного числа замкунтых множеств замкнуто
    \end{enumerate}
\end{theorem}
\begin{proof}
     \[
     U_{\alpha} = \mathbb{R}^{n} \setminus F_{\alpha}
     .\] 
     Оно открытое
     \[
     \mathbb{R}^{n} \setminus (\bigcap F_{\alpha}) = 
     \cup (R_{n} \setminus F_{\alpha}) -- открытое множество
     .\] 
\end{proof}
\end{document}
