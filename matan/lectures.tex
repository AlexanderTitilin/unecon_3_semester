\documentclass[14pt]{extarticle}
\usepackage{fontspec}
\usepackage[russian, english]{babel}
\setmainfont{Times New Roman}
\usepackage{amssymb}
\usepackage{amsthm}
\usepackage{setspace}
\onehalfspacing
\usepackage{amsmath}
\usepackage{listings}
\usepackage{indentfirst}
\setlength{\parindent}{1.25cm}
\usepackage[right=10mm,left=30mm,top=20mm,bottom=20mm]{geometry}
\newtheorem{theorem}{Теорема}
\newtheorem{definition}{Определение}
\newtheorem{corollary}{Следствие}[theorem]
\newtheorem{lemma}[theorem]{Лемма}
\DeclareMathOperator{\grad}{grad}
\title{Лекции по математическому анализу 3 семестр}
\author{}
\date{}
\begin{document}
\maketitle
\section{Функции нескольких вещественных переменных}
\[
	X \subset \mathbb{R}^{n}
	.\]
\[
	x = (x_1,x_2,\dots,x_{n}) \in X
	.\]
\[
	x \rightarrow f(x)
	.\]
\[
	f(x_1,x_2) = x_1^2 + 3x_2
	.\]
\[
	z = x^{2} + 3y
	.\]
\[
	f : \mathbb{R}^n \to \mathbb{R}
	.\]
\section{Замкнутый промежуток в n-мерном пространстве}
\[
	a = (a_1,\dots,a_{n})
	.\]
\[
	b = (b_1,\dots,b_{n})
	.\]
\[
	x = (x_1,\dots,x_{n})
	.\]
\[
	\forall  i ~ 1 \le  i \le  n ~ a_{i} \le  x_{i} \le  b_{i}
	.\]
% \section{Обобщение круга}
% Замкнутый круг -- множество всех точек, с одинаковым расстоянием до данной
\section{Окрестность}
Окрестность точки -- открытый шар с центром в этой точке
\section{Внутренняя точка}
\[
	X \subset \mathbb{R}^{n}
	.\]
\[
	a \in \mathbb{R}^{n}
	.\]
a называется внутренней точкой множества $X$, если $\exists  r  B(a,r) \subset X$
\section{Внешняя точка}
a называется внешней точкой по отношению к множеству $X$, если  $\exists  B(a,r)$ $\subset \mathbb{R}^{n} \setminus X$.
\section{Граничная точка}
Если любая граница точки содердит точки и из множества и не оттуда.
\section{Обозначения шаров}
\[
	B(a,r) = \{x \in \mathbb{R}^{n} \mid \rho(x,a) < R\}
	.\]
\[
	\overline{B}(a,r) = \{x \in \mathbb{R}^{n} \mid \rho(x,a) \le  R\}
	.\]
Множество называется открытым, если всего его точки внутренние
\section{Открытое множество}
Множество A открыто в  X, если
\[
	A = X \cap U
	.\]
U - открытое множество
\begin{theorem}
	Пересечение двух открытых множеств является открытым.
\end{theorem}
\begin{proof}
	Надо доказать, что все точки $U \cap V$ внутренние.\\
	\[
		a \in X
		.\]
	а внутренняя точка множетсва U, значит  $\exists r_1  ~ B(a,r_1) \subset U$\\
	а внутренняя точка множетсва V, значит  $\exists r_2  ~ B(a,r_2) \subset V$
	\[
		r = \min{r_1,r_2}
		.\]
	\[
		B(a,r) \subset U \cap V
		.\]
\end{proof}
\begin{theorem}
	\[
		\{U_{\alpha}\}_{\alpha \in I}
		.\]
	\[
		U_{\alpha} \in R^{n}
		.\]
	пусть $\forall  \alpha$ $U_{\alpha}$ открытое
	тогда $\bigcup_{\alpha \in I} U_{\alpha}$ открыто
\end{theorem}
\begin{theorem}
	\[
		F \subset \mathbb{R}^{n}
		.\]
	F - замкнуто $\iff$  $\mathbb{R}^{n} \setminus F$ - открытое
\end{theorem}
\begin{theorem}
	\begin{enumerate}
		\item пересечение любого числа замкнутых множеств являкется замкнутым
		\item объединение конечного числа замкунтых множеств замкнуто
	\end{enumerate}
\end{theorem}
\begin{proof}
	\[
		U_{\alpha} = \mathbb{R}^{n} \setminus F_{\alpha}
		.\]
	Оно открытое
	\[
		\mathbb{R}^{n} \setminus (\bigcap F_{\alpha}) =
		\cup (R_{n} \setminus F_{\alpha}) -- открытое множество
		.\]
\end{proof}
\section{Замыкание}
\begin{definition}
	Пусть $X \subset \mathbb{R}^{n}$. Замыкание $X$,  $\overline{X}$ наименьшее замкнутое множество, в котором лежит X.
\end{definition}
\[
	B(a,r)
	.\]
\[
	\forall  b \in B(a,r)
	.\]
\[
	r_1 = r - d(a,b)
	.\]
Рассмотрим $B(b,r_1)$, докажем, что $B(b,r_1) \subset B(a,r)$
\[
	\forall  x \in B(b,r_1)
	.\]
Нужно доказать, что $x \in B(a,r)$
\[
	d(x,a) < R
	.\]
Итак, $d(x,a) \le  d(x,b) + d(a,b)$
\[
	d(x,a) \le  d(x,b) + r - r_1
	.\]
\section{Компактно}
\begin{definition}
	\[
		X \subset \mathbb{R}^{n}
		.\]
	X компактно, если замкнуто и ограничено
\end{definition}
\begin{definition}
	Множетсво ограничено, если лежит в неком шаре.
\end{definition}
\section{Предел в $\mathbb{R}^{n}$}
\begin{definition}
    \[
        (x_{n} \to a)
    .\] 
    \[
    \forall  \epsilon > 0 ~ \exists  n_{0} ~  \forall  n \ge  n_0 d(x_{n},a) \le  \epsilon
    .\] 
\end{definition}
\begin{theorem}[о покоординатной сходимости]
    $(x^{(n)})$ -- последовательность точек в $\mathbb{R}^{n}$, $a \in \mathbb{R}^{n}$
     \[
     x^{(n)} \to a \iff x_{k}^{(n)} \to a_{k}
     .\] 
\end{theorem}
\begin{proof}
    Для случая $n = 2$
     \[
         (x_{n},y_{n}) , (a,b)
    .\] 
    \[
    |x_{n} - a| \le  \sqrt{(x_{n} - a)^2 + (y_{n} - b)^2}
    .\] 
    \[
    |y_{n} - b| \le  \sqrt{(x_{n} - a)^2 + (y_{n} - b)^2}
    .\] 
\end{proof}
\section{Упражнения}
\begin{enumerate}
    \item 
Пусть множество $X$ замкнуто, тогда оно содержит все свои предельные точки. Верно ли обратное
\end{enumerate}
\section{Предел функции n переменных}
\[
X \subset \mathbb{R}^{n}
.\] 
\[
V = abc
.\] 
\[
    X = \{ (a,b,c) | a>0,b>0,c>0\}
.\] 
\[
x + y^{2} + z^{3} = 1
.\] 
\begin{definition}
    \[
    X \subset \mathbb{R}^{n}, f : X \to \mathbb{R}
    .\] 
    \[
    a \in \mathbb{R}^{n}
    .\] 
    a предельная точка множества $X$
     \[
    A = \lim_{x \to a} f(x)
    .\] 
    если
    \[
    \forall  (x^{(n)}) 
    \begin{cases}
        (x^{(n)}) \in X\\
        x^{(n)} \neq a\\
        x^{(n)} \to a
    \end{cases}
    .\] 
    \[
    f(x^{(n)}) \to A
    .\] 
    \[
    \lim_{x \to a} f(x) = A \iff \forall  \epsilon > 0 \exists  \delta >0 d(x,a) < \delta \implies |f(x) - A| < \epsilon
    .\] 
\end{definition}
\begin{theorem}[Вейрештрасса]
    Пусть $X \subset \mathbb{R}^{n}, f : X \to \mathbb{R}$. $f$ непрерывна на $X$ 
    f принимает на X наименьшее и наибольшее значение
\end{theorem}
\section{Теоремы о непрерывных функциях}
\subsection{Связанное множество}
\begin{definition}
    Путь в $\mathbb{R}^{n}$ -- набор непрерывных функций $x(t) = ( x_1(t),\dots,x_{n}(t) )$ 
    заданных на $[a,b]$
    
    Точка $x(a)$ называется началом пути.  $x(b)$ конечная точка пути. Множество всех точек $x(t), t \in [a,b]$ носитель пути.
\end{definition}
\begin{definition}
    Пусть $X \subset \mathbb{R}^{n}$. X называется связанным, если для любых точек
    $p,q \in X$ существует путь с началом  $p$ , концом  $q$, носитель которого лежит в $X$
\end{definition}
\section{Теорема Вейрштрасса}
\begin{theorem}
    Пусть $X$ -- компактное подмножество $\mathbb{R}^{n}$, $f$ функция заданная на  $X$, и непрерывная во всех точках множества  $X$. Тогда $f$ принимает и наибольшее и наименьшее значение.
\end{theorem}
\begin{proof}
    Cлучай для $n  = 1$
    \[
    \forall  x f(x) \neq M
    .\] 
    \[
    f(x) < M
    .\] 
    \[
    g(x) = \frac{1}{M - f(x)} \le  C
    .\] 
    \[
    M - f(x) \ge  \frac{1}{C}
    .\] 
    \[
    M - \frac{1}{C} \ge  f(x)
    .\] 
    Пусть нет наибольшего значения
    \[
    \exists  x_1  f(x_1) > 1
    .\] 
    Иcходный промежуток  $\delta_1$
    \[
        \delta_1 \supset \delta_2
    .\]
    \[
    x_2 \in \delta_2
    .\] 
    \[
    f(x_2) > 2
    .\] 
    \[
    f(x_{n}) > n
    .\] 
    \[
        \bigcup  \delta = \{a\}
    .\]
    \[
    x_{n} \to a
    .\] 
    \[
    x_{n} \in \delta_{n}, a \in \delta_{n}
    .\] 
    \[
    |\delta_{n}| = \frac{|\delta_1|}{2^{n-1}}
    .\] 
    \[
    |x_{n} - a| < \frac{|\delta_{1}|}{2^{n - 1}} \to 0
    .\] 
    \[
    x_{n} \to a
    .\] 
    \[
    f(x_{n}) \to f(a)
    .\] 
    Неограниченная посл чисел стремится к числу, какакя-то хуйня такого не бывает.
    
    Теперь для функции $2$ переменных
    Докажем что  f ограничена сверху. Предположим это неправда, тогда $\exists x^{(1)} \in \Delta_{1} \cap X,  f(x^{(1)}) >  1$
    \[
   \exists x_2 \in  \Delta x_2 \cap X
    .\] 
    f неограниченна на $\Delta_2 \cap X, f(x^{2}) > 2$
    по лемме \ref{1} $\bigcap \Delta = \{a\}$ состоит из одной точки.
    Тогда последовательность $x^{(n)} \to a$
    \[
    x^{(n)} \in \Delta
    .\] 
    \[
        d(x^{n},a) \le \sqrt{2} \text{размер} \Delta_{n}
    .\] 
    Дальше как и для функции одной переменной.
\end{proof}
\begin{lemma}[О вложенных отрезках] \label{1}
    \[
    n  = 2
    .\] 
    \[
        a_1 \le  x_1 \le  b_1 , x_1 \in [a_1,b_1]
    .\] 
    \[
        a_2 \le  x_2 \le  b_2 , x_2 \in [a_2,b_2]
    .\] 
    \[
        [a_1,b_1] \times [a_2,b_2]
    .\]  
    Дана последовательность вложенных замкнутых $n$ мерных промежутков
    \[
    \varDelta_{1}  \supset \varDelta_2 \supset 
    .\] 
    $d(\varDelta)$ -  длина наибольшей стороны
     \[
    d(\varDelta_{n}) \to 0
    .\] 
    Тогда в пересечение одна точка
\end{lemma}
\begin{proof}
    Спроектируем все промежутки на ось абцисс, мы получим последовательность замкнутых вложенных промежутков, лежаших впромежутке $[a_1,b_1]$ таких что длины этих прожетков стремятся к 0, пересечение таких промежутков состоит из 1 точки $\alpha_1$
    Анлогично спроектировали эти промежутки на ось ординат, получили последовательность замкнутых вложенных промежутков лежаших в $[a_2,b_2]$ по лемму о вложенных промежутках их пересечение  состоит из одной точки $\alpha_2$
    Рассмотрим точку с координатами $\alpha_1,\alpha_2$
\end{proof}
\section{Теорема Больцано}
\begin{theorem}
    Пусть $X$ связное подмножество пространства  $\mathbb{R}^{n}$,
    пусть $f$ непрерывная функция, заданная на  $X$. Пусть  $p,q$ значения функции  $f$, причем  $p < q$. Тогда  $\forall  r \in (p,q) \exists x \in X$ , $f(x) = r$
\end{theorem}
\section{Множество уровня}
\[
f: X \subset \mathbb{R}^{n} \to \mathbb{R}
.\] 
\[
f(x_1,\dots,x_{n})
.\] 
\[
    \{x \mid x \in X, f(x) = c\}
.\] 
\section{Дифферинцирование нескольких переменных}
\begin{definition}[Производная по направлению]
    \[
    \lim_{h \to 0} \frac{f(x_0 + l_1 h, y_0 + l_2 h) - f(x_0,y_0)}{h}
    .\] 
\end{definition}
Пусть $\vec{l} = (0,1)$
\[
\frac{\partial f}{\partial x}=
\lim_{h \to 0} \frac{f(x_0 + h,y_0) - f(x_0,y_0)}{h}
.\] 
частная производная по $x$
\[
\frac{\partial f}{\partial y} (x_0,y_0) = 
\lim_{h \to 0} \frac{f(x_0,y_0  + h) - f(x_0,y_0)}{h}
.\] 
\begin{definition}[Градиент]
    \[
    \grad f (x_0,y_0) = (\frac{\partial f}{\partial x} (x_0,y_0),\frac{\partial f}{\partial x})
    .\] 
\end{definition}
\begin{definition}[точка экстренума (максимума)]
    $(x_0,y_0)$ - точка максимума, если $\exists$ окретсность U, $\forall x  \in U$  $f(x) < f(x_0)$
\end{definition}
\begin{theorem}[Ферма]
    \[
    X \subset \mathbb{R}^{2}
    .\] 
    \[
    f : X \to \mathbb{R}
    .\] 
    $(x_0,y_0)$ внутренняя точка X. Пусть $\exists  \frac{\partial f}{\partial x} (x_0$
\end{theorem}
\subsection{Пример}
\[
f(x,y) = x^2 + 2xy
.\] 
\[
\frac{\partial f}{\partial x} = 2x + 2y = 0
.\] 
\[
\frac{\partial f}{\partial y} = 2x = 0
.\] 
\[
\begin{cases}
    x = 0\\
    y = 0
\end{cases}
.\] 
\[
f(0,0) = 0
.\] 
\section{Условный экстренум}
\[
f(x_0 + h) - f(x_0) f'(x_0) h + o(h)
.\] 
\begin{definition}[Дифференцируемость функции]
    \[
    f(x_0 + h) - f(x_0) = ch + o(h)
    .\] 
\end{definition}
\begin{theorem}[О полном приращении]
    Пусть функция $f$ задана на  $X \subset \mathbb{R}^{2}$, a внутренняя точка $X$, $a = (x_0,y_0)$.  Пусть $f$ имеет частную производную во всех точках в некоторой окрестности точки  $a$. Пусть эти частные производные непрерывны в точке a. Тогда вблизи точки a имеет место
     \[
    f(x,y) - f(x_0,y_0) = \partial_{x} f (x_0,y_0) (x_0 - y_0) + \partial_{y} (x_0,y_0) (y - y_0) + \alpha(x,y) (x - x_0) + \beta(x,y) (y - y_0)
    .\] 
    $\alpha,\beta$ бесконечно малые в  $(x_0,y_0)$
\end{theorem}
\begin{proof}
    
\end{proof}
\begin{definition}[Дифференцируемость функции нескольких переменных]
    
\end{definition}
\end{document}
