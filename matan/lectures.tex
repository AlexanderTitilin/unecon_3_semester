\documentclass[14pt]{extarticle} \usepackage{fontspec}
\usepackage[russian, english]{babel}
\setmainfont{Times New Roman}
\usepackage{amssymb}
\usepackage{amsthm}
\usepackage{setspace}
\onehalfspacing
\usepackage{amsmath}
\usepackage{listings}
\usepackage{indentfirst}
\usepackage{pgfplots}
\setlength{\parindent}{1.25cm}
\usepackage[right=10mm,left=30mm,top=20mm,bottom=20mm]{geometry}
\newtheorem{theorem}{Теорема}
\newtheorem{definition}{Определение}
\newtheorem{corollary}{Следствие}[theorem]
\newtheorem{lemma}[theorem]{Лемма}
\DeclareMathOperator{\grad}{grad}
\title{Лекции по математическому анализу 3 семестр}
\author{}
\date{}
\begin{document}
\maketitle
\section{Функции нескольких вещественных переменных}
\[
	X \subset \mathbb{R}^{n}
	.\]
\[
	x = (x_1,x_2,\dots,x_{n}) \in X
	.\]
\[
	x \rightarrow f(x)
	.\]
\[
	f(x_1,x_2) = x_1^2 + 3x_2
	.\]
\[
	z = x^{2} + 3y
	.\]
\[
	f : \mathbb{R}^n \to \mathbb{R}
	.\]
\section{Замкнутый промежуток в n-мерном пространстве}
\[
	a = (a_1,\dots,a_{n})
	.\]
\[
	b = (b_1,\dots,b_{n})
	.\]
\[
	x = (x_1,\dots,x_{n})
	.\]
\[
	\forall  i ~ 1 \le  i \le  n ~ a_{i} \le  x_{i} \le  b_{i}
	.\]
% \section{Обобщение круга}
% Замкнутый круг -- множество всех точек, с одинаковым расстоянием до данной
\section{Окрестность}
Окрестность точки -- открытый шар с центром в этой точке
\section{Внутренняя точка}
\[
	X \subset \mathbb{R}^{n}
	.\]
\[
	a \in \mathbb{R}^{n}
	.\]
a называется внутренней точкой множества $X$, если $\exists  r  B(a,r) \subset X$
\section{Внешняя точка}
a называется внешней точкой по отношению к множеству $X$, если  $\exists  B(a,r)$ $\subset \mathbb{R}^{n} \setminus X$.
\section{Граничная точка}
Если любая граница точки содердит точки и из множества и не оттуда.
\section{Обозначения шаров}
\[
	B(a,r) = \{x \in \mathbb{R}^{n} \mid \rho(x,a) < R\}
	.\]
\[
	\overline{B}(a,r) = \{x \in \mathbb{R}^{n} \mid \rho(x,a) \le  R\}
	.\]
Множество называется открытым, если всего его точки внутренние
\section{Открытое множество}
Множество A открыто в  X, если
\[
	A = X \cap U
	.\]
U - открытое множество
\begin{theorem}
	Пересечение двух открытых множеств является открытым.
\end{theorem}
\begin{proof}
	Надо доказать, что все точки $U \cap V$ внутренние.\\
	\[
		a \in X
		.\]
	а внутренняя точка множетсва U, значит  $\exists r_1  ~ B(a,r_1) \subset U$\\
	а внутренняя точка множетсва V, значит  $\exists r_2  ~ B(a,r_2) \subset V$
	\[
		r = \min{r_1,r_2}
		.\]
	\[
		B(a,r) \subset U \cap V
		.\]
\end{proof}
\begin{theorem}
	\[
		\{U_{\alpha}\}_{\alpha \in I}
		.\]
	\[
		U_{\alpha} \in R^{n}
		.\]
	пусть $\forall  \alpha$ $U_{\alpha}$ открытое
	тогда $\bigcup_{\alpha \in I} U_{\alpha}$ открыто
\end{theorem}
\begin{theorem}
	\[
		F \subset \mathbb{R}^{n}
		.\]
	F - замкнуто $\iff$  $\mathbb{R}^{n} \setminus F$ - открытое
\end{theorem}
\begin{theorem}
	\begin{enumerate}
		\item пересечение любого числа замкнутых множеств являкется замкнутым
		\item объединение конечного числа замкунтых множеств замкнуто
	\end{enumerate}
\end{theorem}
\begin{proof}
	\[
		U_{\alpha} = \mathbb{R}^{n} \setminus F_{\alpha}
		.\]
	Оно открытое
	\[
		\mathbb{R}^{n} \setminus (\bigcap F_{\alpha}) =
		\cup (R_{n} \setminus F_{\alpha}) -- открытое множество
		.\]
\end{proof}
\section{Замыкание}
\begin{definition}
	Пусть $X \subset \mathbb{R}^{n}$. Замыкание $X$,  $\overline{X}$ наименьшее замкнутое множество, в котором лежит X.
\end{definition}
\[
	B(a,r)
	.\]
\[
	\forall  b \in B(a,r)
	.\]
\[
	r_1 = r - d(a,b)
	.\]
Рассмотрим $B(b,r_1)$, докажем, что $B(b,r_1) \subset B(a,r)$
\[
	\forall  x \in B(b,r_1)
	.\]
Нужно доказать, что $x \in B(a,r)$
\[
	d(x,a) < R
	.\]
Итак, $d(x,a) \le  d(x,b) + d(a,b)$
\[
	d(x,a) \le  d(x,b) + r - r_1
	.\]
\section{Компактно}
\begin{definition}
	\[
		X \subset \mathbb{R}^{n}
		.\]
	X компактно, если замкнуто и ограничено
\end{definition}
\begin{definition}
	Множетсво ограничено, если лежит в неком шаре.
\end{definition}
\section{Предел в $\mathbb{R}^{n}$}
\begin{definition}
    \[
        (x_{n} \to a)
    .\] 
    \[
    \forall  \epsilon > 0 ~ \exists  n_{0} ~  \forall  n \ge  n_0 d(x_{n},a) \le  \epsilon
    .\] 
\end{definition}
\begin{theorem}[о покоординатной сходимости]
    $(x^{(n)})$ -- последовательность точек в $\mathbb{R}^{n}$, $a \in \mathbb{R}^{n}$
     \[
     x^{(n)} \to a \iff x_{k}^{(n)} \to a_{k}
     .\] 
\end{theorem}
\begin{proof}
    Для случая $n = 2$
     \[
         (x_{n},y_{n}) , (a,b)
    .\] 
    \[
    |x_{n} - a| \le  \sqrt{(x_{n} - a)^2 + (y_{n} - b)^2}
    .\] 
    \[
    |y_{n} - b| \le  \sqrt{(x_{n} - a)^2 + (y_{n} - b)^2}
    .\] 
\end{proof}
\section{Упражнения}
\begin{enumerate}
    \item 
Пусть множество $X$ замкнуто, тогда оно содержит все свои предельные точки. Верно ли обратное
\end{enumerate}
\section{Предел функции n переменных}
\[
X \subset \mathbb{R}^{n}
.\] 
\[
V = abc
.\] 
\[
    X = \{ (a,b,c) | a>0,b>0,c>0\}
.\] 
\[
x + y^{2} + z^{3} = 1
.\] 
\begin{definition}
    \[
    X \subset \mathbb{R}^{n}, f : X \to \mathbb{R}
    .\] 
    \[
    a \in \mathbb{R}^{n}
    .\] 
    a предельная точка множества $X$
     \[
    A = \lim_{x \to a} f(x)
    .\] 
    если
    \[
    \forall  (x^{(n)}) 
    \begin{cases}
        (x^{(n)}) \in X\\
        x^{(n)} \neq a\\
        x^{(n)} \to a
    \end{cases}
    .\] 
    \[
    f(x^{(n)}) \to A
    .\] 
    \[
    \lim_{x \to a} f(x) = A \iff \forall  \epsilon > 0 \exists  \delta >0 d(x,a) < \delta \implies |f(x) - A| < \epsilon
    .\] 
\end{definition}
\begin{theorem}[Вейрештрасса]
    Пусть $X \subset \mathbb{R}^{n}, f : X \to \mathbb{R}$. $f$ непрерывна на $X$ 
    f принимает на X наименьшее и наибольшее значение
\end{theorem}
\section{Теоремы о непрерывных функциях}
\subsection{Связанное множество}
\begin{definition}
    Путь в $\mathbb{R}^{n}$ -- набор непрерывных функций $x(t) = ( x_1(t),\dots,x_{n}(t) )$ 
    заданных на $[a,b]$
    
    Точка $x(a)$ называется началом пути.  $x(b)$ конечная точка пути. Множество всех точек $x(t), t \in [a,b]$ носитель пути.
\end{definition}
\begin{definition}
    Пусть $X \subset \mathbb{R}^{n}$. X называется связанным, если для любых точек
    $p,q \in X$ существует путь с началом  $p$ , концом  $q$, носитель которого лежит в $X$
\end{definition}
\section{Теорема Вейрштрасса}
\begin{theorem}
    Пусть $X$ -- компактное подмножество $\mathbb{R}^{n}$, $f$ функция заданная на  $X$, и непрерывная во всех точках множества  $X$. Тогда $f$ принимает и наибольшее и наименьшее значение.
\end{theorem}
\begin{proof}
    Cлучай для $n  = 1$
    \[
    \forall  x f(x) \neq M
    .\] 
    \[
    f(x) < M
    .\] 
    \[
    g(x) = \frac{1}{M - f(x)} \le  C
    .\] 
    \[
    M - f(x) \ge  \frac{1}{C}
    .\] 
    \[
    M - \frac{1}{C} \ge  f(x)
    .\] 
    Пусть нет наибольшего значения
    \[
    \exists  x_1  f(x_1) > 1
    .\] 
    Иcходный промежуток  $\delta_1$
    \[
        \delta_1 \supset \delta_2
    .\]
    \[
    x_2 \in \delta_2
    .\] 
    \[
    f(x_2) > 2
    .\] 
    \[
    f(x_{n}) > n
    .\] 
    \[
        \bigcup  \delta = \{a\}
    .\]
    \[
    x_{n} \to a
    .\] 
    \[
    x_{n} \in \delta_{n}, a \in \delta_{n}
    .\] 
    \[
    |\delta_{n}| = \frac{|\delta_1|}{2^{n-1}}
    .\] 
    \[
    |x_{n} - a| < \frac{|\delta_{1}|}{2^{n - 1}} \to 0
    .\] 
    \[
    x_{n} \to a
    .\] 
    \[
    f(x_{n}) \to f(a)
    .\] 
    Неограниченная посл чисел стремится к числу, какакя-то хуйня такого не бывает.
    
    Теперь для функции $2$ переменных
    Докажем что  f ограничена сверху. Предположим это неправда, тогда $\exists x^{(1)} \in \Delta_{1} \cap X,  f(x^{(1)}) >  1$
    \[
   \exists x_2 \in  \Delta x_2 \cap X
    .\] 
    f неограниченна на $\Delta_2 \cap X, f(x^{2}) > 2$
    по лемме \ref{1} $\bigcap \Delta = \{a\}$ состоит из одной точки.
    Тогда последовательность $x^{(n)} \to a$
    \[
    x^{(n)} \in \Delta
    .\] 
    \[
        d(x^{n},a) \le \sqrt{2} \text{размер} \Delta_{n}
    .\] 
    Дальше как и для функции одной переменной.
\end{proof}
\begin{lemma}[О вложенных отрезках] \label{1}
    \[
    n  = 2
    .\] 
    \[
        a_1 \le  x_1 \le  b_1 , x_1 \in [a_1,b_1]
    .\] 
    \[
        a_2 \le  x_2 \le  b_2 , x_2 \in [a_2,b_2]
    .\] 
    \[
        [a_1,b_1] \times [a_2,b_2]
    .\]  
    Дана последовательность вложенных замкнутых $n$ мерных промежутков
    \[
    \varDelta_{1}  \supset \varDelta_2 \supset 
    .\] 
    $d(\varDelta)$ -  длина наибольшей стороны
     \[
    d(\varDelta_{n}) \to 0
    .\] 
    Тогда в пересечение одна точка
\end{lemma}
\begin{proof}
    Спроектируем все промежутки на ось абцисс, мы получим последовательность замкнутых вложенных промежутков, лежаших впромежутке $[a_1,b_1]$ таких что длины этих прожетков стремятся к 0, пересечение таких промежутков состоит из 1 точки $\alpha_1$
    Анлогично спроектировали эти промежутки на ось ординат, получили последовательность замкнутых вложенных промежутков лежаших в $[a_2,b_2]$ по лемму о вложенных промежутках их пересечение  состоит из одной точки $\alpha_2$
    Рассмотрим точку с координатами $\alpha_1,\alpha_2$
\end{proof}
\section{Теорема Больцано}
\begin{theorem}
    Пусть $X$ связное подмножество пространства  $\mathbb{R}^{n}$,
    пусть $f$ непрерывная функция, заданная на  $X$. Пусть  $p,q$ значения функции  $f$, причем  $p < q$. Тогда  $\forall  r \in (p,q) \exists x \in X$ , $f(x) = r$
\end{theorem}
\section{Множество уровня}
\[
f: X \subset \mathbb{R}^{n} \to \mathbb{R}
.\] 
\[
f(x_1,\dots,x_{n})
.\] 
\[
    \{x \mid x \in X, f(x) = c\}
.\] 
\section{Дифферинцирование нескольких переменных}
\begin{definition}[Производная по направлению]
    \[
    \lim_{h \to 0} \frac{f(x_0 + l_1 h, y_0 + l_2 h) - f(x_0,y_0)}{h}
    .\] 
\end{definition}
Пусть $\vec{l} = (0,1)$
\[
\frac{\partial f}{\partial x}=
\lim_{h \to 0} \frac{f(x_0 + h,y_0) - f(x_0,y_0)}{h}
.\] 
частная производная по $x$
\[
\frac{\partial f}{\partial y} (x_0,y_0) = 
\lim_{h \to 0} \frac{f(x_0,y_0  + h) - f(x_0,y_0)}{h}
.\] 
\begin{definition}[Градиент]
    \[
    \grad f (x_0,y_0) = (\frac{\partial f}{\partial x} (x_0,y_0),\frac{\partial f}{\partial x})
    .\] 
\end{definition}
\begin{definition}[точка экстренума (максимума)]
    $(x_0,y_0)$ - точка максимума, если $\exists$ окретсность U, $\forall x  \in U$  $f(x) < f(x_0)$
\end{definition}
\begin{theorem}[Ферма]
    \[
    X \subset \mathbb{R}^{2}
    .\] 
    \[
    f : X \to \mathbb{R}
    .\] 
    $(x_0,y_0)$ внутренняя точка X. Пусть $\exists  \frac{\partial f}{\partial x} (x_0$
\end{theorem}
\subsection{Пример}
\[
f(x,y) = x^2 + 2xy
.\] 
\[
\frac{\partial f}{\partial x} = 2x + 2y = 0
.\] 
\[
\frac{\partial f}{\partial y} = 2x = 0
.\] 
\[
\begin{cases}
    x = 0\\
    y = 0
\end{cases}
.\] 
\[
f(0,0) = 0
.\] 
\section{Условный экстренум}
\[
f(x_0 + h) - f(x_0) f'(x_0) h + o(h)
.\] 
\begin{definition}[Дифференцируемость функции]
    \[
    f(x_0 + h) - f(x_0) = ch + o(h)
    .\] 
\end{definition}
\begin{theorem}[О полном приращении]
    Пусть функция $f$ задана на  $X \subset \mathbb{R}^{2}$, a внутренняя точка $X$, $a = (x_0,y_0)$.  Пусть $f$ имеет частную производную во всех точках в некоторой окрестности точки  $a$. Пусть эти частные производные непрерывны в точке a. Тогда вблизи точки a имеет место
     \[
    f(x,y) - f(x_0,y_0) = \partial_{x} f (x_0,y_0) (x_0 - y_0) + \partial_{y}f (x_0,y_0) (y - y_0) + \alpha(x,y) (x - x_0) + \beta(x,y) (y - y_0)
    .\] 
    $\alpha,\beta$ бесконечно малые в  $(x_0,y_0)$\\
    Обозначим $h = x - x_0, k = y-y_0$
    \[
        f(x_0 + h) + g(x_0 + k) = \partial_{x} f(x_0,y_0) h + \partial_{y} f(x_0,y_0) k
        + \alpha(h,k)h - \beta(h,k)k
    .\] 
    $\alpha,\beta$ бесконечно малые в  $(0,0)$
\end{theorem}
\begin{proof}
    \[
    f(x_0 + h,y_0 + k) - f(x_0,y_0) = ( f(x_0 + h,y_0 +k)  - f(x_0,y_0 + k) ) +
    (f(x_0,y_0 + k) - f(x_0,y_0)
    .\] 
    \[
        g(x) = f(x,y_0+k)
    .\] 
    \[
    g(x_0 + h) - g(x_0)
    .\] 
    По теореме Лагранжа
    \[
    g(x_0 + h) - g(x_0) = g'(c)h = \partial_{x}f(c,y_0 + h) h
    .\] 
    \[
        g(y) = f(x_0,y)
    .\] 
    \[
    g(y_0 + k) - g(y_0) = g'(d) k = \partial_{y} g(x_0,d) k
    .\] 
    \[
        f(x_0 + h,y_0 +k) -f(x_0,y_0) =      .\] 
    \[
    \partial_{x} f(x_0,y_0) h + \partial_{y} f(x_0,y_0) k + (\partial_{x} f(x,y_0+k) - \partial_{x} f(x_0,y_0))h + (\partial_{y}f (c,y_0 + k) = \partial_{y}f(x_0,y_0))k
    .\] 
\end{proof}
\begin{definition}[Дифференцируемость функции нескольких переменных]
    Пусть $f : X \to \mathbb{R}$, $X \subset \mathbb{R}^{2} , a = (x_0,y_0) \in \mathbb{R}^2$ внутренняя
    $f$ дифференцируема в точке  $a$ , если , $\exists A,B \in \mathbb{R}$ и функции  $\alpha(x,y), \beta(x,y)$ бесконечно малые в  $(x_0,y_0)$ и вблизи $(x_0,y_0)$ имеет место равенство
    \[
    f(x,y) -  f(x_0,y_0) = A (x - x_0) + B(y-y_0) + \alpha(x,y) (x - x_0)  + \beta(x,y) (y - y_0)
    .\] 
\end{definition}
\begin{theorem}
    Если f дифференцируема в $(x_0,y_0)$, то $A = \partial_{x} f(x_0,y_0)$ , $B = \partial_{y} f(x_0,y_0)$
\end{theorem}
\begin{proof}
    Подставим $y = y_0$ 
    \[
    f(x,y_0) - f(x_0,y_0) = A(x - x_0) + \alpha(x,y_0) (x - x_0)
    .\] 
    \[
    \forall  x \neq x_0
    .\] 
    \[
    \frac{f(x - y_0) - f(x_0,y_0)}{x - x_0} = A + \alpha(x,y_0)
    .\] 
    \[
    \frac{\partial f}{\partial x} (x_0,y_0) = \lim_{x \to x_0} \frac{f(x,y_0) - f(x_0,y_0)}{x - x_0} = \lim_{x \to 0} (A + \alpha(x)) =A
    .\] 
\end{proof}
\begin{theorem}
    \[
        f: X \to \mathbb{R}, a  = (x_0,y_0) \in X, \vec{l} = (l_1,l_2) ,||l|| = 1
    .\] 
    Пусть f дифференцируема в точке a, тогда $\exists \frac{\partial f}{\partial \vec{l}} (a)$ 
    \[
    \frac{\partial f}{\partial \vec{l}} (a) = \frac{\partial f}{\partial x} (x_0,y_0) l_1 +  \frac{\partial f}{\partial y} (a) l_2
    .\] 
     \[
    \grad f_{(x_0,y_0)} =  (\frac{\partial f}{\partial x} (x_0,y_0), \frac{\partial f}{\partial y} (x_0,y_0))
    .\] 
\end{theorem}
\begin{proof}
        \[
        f(x_0 + hl_1, y + k l_2) - f(x_0,y_0) = 
        .\] 
    \end{proof}
\section{Направление наибольшего возрастания функции}
\[
- ||\vec{a},\vec{b}|| \le |(\vec{a},\vec{b})| \le ||\vec{a}|| ||\vec{b}||
.\] 
$\vec{l}$ направление наибольшего возрастания, если $\vec{l} = \frac{1}{||\grad(f(x_0,y))||} \grad(f(x_0,y_0)$
\section{}
\[
y' (3y^2) + y' + 1 = 0
.\] 
\[
y' (3y^2 + 1) = -1
.\] 
\[
y' = - \frac{1}{3y^2 + 1}
.\] 
\[
y^3(-2) + y(-2) -2 = 0
.\] 
\[
y(-2) = 1
.\] 
\section{}
\[
    y^2 + y - x = 0
.\] 
\[
y = \frac{-1 \pm \sqrt{1 + 4x} }{2}
.\] 
\begin{theorem}
    \[
    f : X \subset \mathbb{R}^2 \to \mathbb{R}
    .\] 
    X -- открытое множество.
    \[
    x,y: I \to \mathbb{R}
    .\] 
    I -- открытый промежуток
    \[
    t \in T
    .\] 
    \[
    t \to x(t)
    .\] 
    \[
    t \to y(t)
    .\] 
    \[
    F(t) = f(x(t),y(t))
    .\] 
    Пусть $x(t),y(t)$ дифференцируемы в точке $t_0$из I, f дифференцируема  в точке $(x(t_0),y(t_0))$
    \[
    F'(t_0) = \partial_{x}f (x(t_0),y(t_0)) x'(t_0) + \partial_{y} (x(t_0),y(t_0)) y'(t_0)
    .\] 
\end{theorem}
\begin{proof}
    \[
    F'(x_0) = \lim_{t \to t_0}  \frac{F(t) -F(t_0)}{t - t_0}
    .\] 
    \[
    F(t) - F(x_0) = f(x(t),y(t)) - f(x(t_0),y(t_0))
    .\] 
\end{proof}
\begin{theorem}
    \[
    f : X \subset \mathbb{R} \to \mathbb{R}
    .\] 
    X - открытое множество, f дифференцируема на $X$. Пусть  $(x_0,y_0), (x,y) \in X$ 
    \[
    f(x,y) - f(x_0,y_0) = \partial_{x} f(c_1,c_2) (x - x_0) + \partial_{y} f(c_1,c_2)(y - y_0)
    .\] 
\end{theorem}
\begin{proof}
    \[
    f(x_0 + t(x - x_0),y_0 + t(y-y_0)) - f(x_0,y_0) - f(x_0,y_0)
    .\] 
\end{proof}
\begin{corollary}
    Пусть X выпуклое. Пусть $\partial_{x} f, \partial_{y}f$ тождественно равны 0. Тогда f постоянна
\end{corollary}
\begin{corollary}
    Пусть X область (открытое связное множество)
\end{corollary}
\section{К. 3.60}
\subsection{}
\[
3u^2 \partial_{x} u + 3y(u + x \partial_{x} u) = 0
.\] 
\[
y^2 \partial_{x} u + yu + yx \partial u = 0
.\] 
\[
\partial_{x} u (u^2 + xy) = - yu
.\] 
\[
3 u^2 \partial_{y} x
.\] 
\subsection{}
\[
\partial_{x} u  e^{u} - y(x + x \partial_{x} u) = 0
.\] 
\[
e^{u(1,0)} = 2
.\] 
\[
    u = \ln{2}
.\] 
\[
\partial_{x} u (e^{u}  - yx) - yx  = 0
.\] 
\[
2 \partial_{x} u = 0
.\] 
\section{}
\[
    u + \ln{(x + y + u)} = 0
.\] 
\[
    u'_{x}  (1 + u'_{x}) \frac{1}{x + y+ u} = 0
.\] 
\[
u'_{y} + (1 + u'_{y}) \frac{1}{x + y +u} = 0
.\] 
\[
u'_{x} + \frac{1}{x + y + u} + \frac{u'_{x}}{x + y + u} = 0
.\] 
\[
    (1 + \frac{1}{x + y + u}) u'_{x} = -\frac{1}{x + y + u}
.\] 
\[
    u(1,-1) + \ln{u(1,-1)} = 0
.\] 
\[
t = u(1,-1)
.\] 
\[
    t + \ln{t} = 0
.\] 
\section{Теорема о неявной функции}
\begin{theorem}
    Пусть $(x_0,y_0) \in \mathbb{R}^2$, $U$ окретсность точки  $(x_0,y_0)$, пусть функция $F : U \to \mathbb{R}$ Пусть выполняются условия
    \begin{enumerate}
        \item F непрерывно дифференцируема в $U$
        \item  $F(x_0,y_0) = 0$
        \item $F'_{y} (x_0,y_0) \neq 0$
    \end{enumerate}
    Тогда существует открытый промежуток $I_{x} \times I_{y}$, $I_{x}$ окретсность $x_0$,  $I_{y}$ окрестность $y_0$
    \[
        I_{x} =  \{x | |x- x_0| <\alpha\}
    .\] 
    \[
        I_{y} = \{y \mid |y-y_0| < \beta\}
    .\] 
    Которые лежат в $U$ и  $\exists f$ заданная на $I_{x}$ принимающая значения $I_{y}$, такая что  $f \in C^{1}(I_{x})$ (непрерывно дифференцируема на $I_{x}$),
$F(x,y) = 0 \iff y = f(x), $ для  $x \in I_{x}$, $f'(x) = - \frac{F'_{x}(x,f(x))}{F'_{y}(x,f(x))}, \forall  x \in I_{}$
\end{theorem}
\begin{proof}
    Пусть $F$ непрерывно дифференцируема, т.е $F'_{x}, F'_{y}$ непрерывны в  $(x_0,y_0)$, то $F'_{y}(x,y) > 0$ для $(x,y)$ вблизи $(x_0,y_0)$. Рассмотрим $F(x,y_0 -\beta$
    \[
    \forall  x \in (x_0 -\alpha,x_0 + \alpha) F(x,y_0 -\beta) <0
    .\] 
    \[
    F(x_0,y_0 + \beta) > 0
    .\] 
    то сущестует $\alpha_2 \forall  x \in (x_0 - \alpha_2,x_0 + \alpha_2), F(x,y_0 + \beta) > 0$ Пусть $\alpha = \min{\alpha_1,\alpha_2}$\\
    Итак м построили $I_{x} \times I_{y}, I_{x} = (x_0 -\alpha,x_0+\alpha), I_{y}= (x_0 -\beta,x_0 + \beta)$
    \[
    F'_{y}(x,y) > 0 , (x,y) \in Y
    .\] 
    F возрастает значит существует единственное значени $y$, такое что  $F(x,y) =0$
    Нужно доказать, что $f$ дифференцируема  $(x_0 - \alpha;x+\alpha)$
\end{proof}
\subsection {Пример}
\[
x^2 + ^2 - 1 = 0
.\] 
\[
\exists U(x_0,y_0)
.\] 
\[
F(x,y) = 0
.\] 
Данная функция дифференциуема в точке $(x_0,y_0)$
\[
F(x,y) - F(x_0,y_0) = \partial_{x} F(x_0,y_0) (x-x_0) + \partial_{y} F(x_0,y_0)(y - y_0) + \dots
.\] 
\[
    (x_0,y_0) \in C
.\] 
\[
\frac{\partial_{x} F(x_0,y_0)}{\partial_{y} F(x_0,y_0)} + \frac{y - y_0}{x - x_0} = 0
.\] 
\section{К 62}
\[
 u'_{x} - 4u - 4xu'_{x} = 0
.\] 
\[
 u'_{y} - 4x u'_{y} + 2y = 0
.\] 
\[
u^{3}(1,-2) - 4u(1,-2) = 0
.\] 
\[
    u(1,-2) \in \{0,2,-2\}
.\] 
\section{Достаточное условие строгого экстренума}
\begin{theorem}
    \[
        x^{0} = (x_1^{(0)},x_2^{(0)},\dots,x^{(0)}_{n}) \in \mathbb{R}
    .\] 
    \[
        U(x^{(0)}) - \text{Окрестность}
    .\] 
    \[
    f: U(x^{(0)}) \to \mathbb{R}
    .\] 
    Пусть
    \[
    f \in C^{(2)} (U(x^{(0)}))
    .\] 
    По формуле Тейлора
    \[
        f(x^{(0)} + h) = f(x^{(0)}) + \frac{1}{2} \sum_{i,j = 1}^{n} \frac{\partial^2 f}{\partial x_{i} \partial x_{j}} (x^{(0)}) h_{i} h_{j} + o(||h||^2)
    .\] 
    \[
    Q(h) = \sum_{i,j = 1}^{n} \frac{\partial^2 f}{\partial x_{i} \partial x_{j}} (x^{(0)}) h_{i} h_{j}
    .\] 
    Тогда 
    \begin{enumerate}
        \item  если $Q(h)$ положительно определена, то  $x^{(0)}$ точка строгого минимума.
        \item Если $Q(h)$ отрицательно определена, то  $x^{(0)}$ точка строгого максимума
        \item если $Q(h)$ принимает значения разных знаков, то  $x^{(0)}$ не является точкой экстренум
    \end{enumerate}
\end{theorem}
\begin{proof}
    \begin{enumerate}
        \item 
            Пусть $h \neq 0$ , $x_0$ критическая точка
            \[
            f(x^{(0)}) + ||h^2|| (\frac{1}{2} \sum_{i,j = 1}^{n} \frac{\partial^2 f}{\partial x_{i} \partial x_{j}}) (x_0) (\frac{h_{i}}{||h||} \frac{h_{j}}{||h||} + \alpha(h) )
            .\] 
            \[
            \frac{t_{i}}{||h||}  = t_{i}
            .\] 
            \[
                \sum_{i,j = 1} ^{k} \frac{\partial^2 f}{\partial x_{i} \partial x_{j}} * t_{i} *t_{j}
            .\] 
            \[
                ||t|| \text{- } n \text{мерная сфера радиуса~} 1
            .\] 
            Это множество компанктно\\
            По т Вейрштрасса эта квадратичная форма принимает наибольшее и наименьшее значение
            \[
                \sum_{i,j = 1}^{n} \frac{\partial^2 f}{\partial x_{i} \partial x_{j}} t_{i} t_{j}
            .\] 
            обозначили их $m,M$
             \[
            0 < m \le M
            .\] 
            Принимает наименьшее значение равное $\frac{1}{2}m$, тк $\alpha(h)$ бесконечное малое для всех h достаточно близких к 0 $|\alpha(h)| < 0$
             \[
            \frac{1}{2} \sum_{i,j = 1}^{n}
            .\] 
            Следовательно для всех таких h
            \[
            f(x_{0}^{(0)} + h) > f(x^{(0)})
            .\] 
            следовательно $x^{0}$ точка минимума
        \item Аналогично
        \item  Пусть $e_{m}$ та точка единичной сферы, в которой  квадратичная форма принимает наименьшее значение
            \[
            \sum_{i,j = 1}^{n} \frac{\partial ^2 f}{\partial x_{i} \partial x_{j}} (x^{(0)}) t_{i} t_{j}
            .\] 
            $e_{M}$ наибольшее значение. Хочу доказать, что в любой окружность $x^{(0)}$ есть значения и больше и меньше.
            \[
            h = \lambda e_{m}
            .\] 
            \[
            f(x^{0} + \lambda e_{m}) - f(x^{(0)}) = \frac{1}{2} \sum_{i,j = 1}^{n} 
            \frac{\partial ^2 f}{\partial x_{i} \partial x_{j}} ( x^{(0)}) \lambda \alpha_{i} \lambda \alpha_{j} + \alpha(||\lambda e_{m} ||^2) = 
            \frac{1}{2} \lambda^2 (\sum_{i,j}^{n}  \frac{\partial^2 f}{\partial x_{i} \partial x_{j}}) (x^{0})\alpha_i \alpha_{j} + \alpha(\lambda^2 ||e_{m}||) *||e_{m}||^2
            .\] 
    \end{enumerate}
\end{proof}
\section{Ищем критические точки на границах}
Пусть $(x_0,y_0)$ тогда экстренум $f(x,y)$ кривой  $g(x,y) = 0$. 
Тогда  $(f'_{x} (x_0,y_0),f'_{y}(x_0,y_0)$ колинерен $( g'x(x_0,y_0),g'_{y}(x_0,y_0) )$
\[
g(x,y) = 0 \implies y = \phi(x)
.\] 
\[
f(x,\phi(x))' = f'_{x}(x,\phi(x))  + f'_{y}(x,\phi(x)) \phi'(x) = 
f'_{x}(x,\phi(x)) - f'_{y}(x,\phi(x)) \frac{g'_{x}(x,\phi(x))}{g'_{y}(x,\phi(x))} = 0
.\]
\[
\frac{f'_{x}(x,\phi(x))}{f'_{y}(x,\phi(x))} = \frac{g'_{x}(x,\phi(x))}{g'_{y}(x.\phi(x))}
.\] 
\section{Задача лагранжа}
Есть кривая, есть функция
 \[
L(x,y,\lambda) = f(x,y) - \lambda g(y)
.\] 
\[
L'_{x}  = f'_{x} - \lambda g'_{x} =  0
.\] 
\[
    L'_{y} = f'_{y} - \lambda f'_{y} = 0
.\] 
\[
L'_{\lambda} = -g(x,y) = 0
.\] 
\section{Интеграл фнп}
\begin{definition}[Разбиение промежутка]
    Рассмотрим произвол
\end{definition}
\begin{definition}[Интегральные суммы]
    Пусть $I = [a,b] \times [c,d]$. Рассмотрим произвольное разбиение отрезка $[a,b]: a < x_1 < \dots< x_{n - 1} < b$, произвольное разбиение отрезка $[c,d] :
    c < y_1 < y_2 < \dots < y_{m-1} < d$. Рассмотрим 
    \[
        [x_{i -1},x_{i}] \times [y_{j -1},y_{j}]
    .\] 
    \[
        \xi_{i} \in [x_{i - 1},x_{i}]
    .\]
    \[
        \mu_{j}  \in [y_{j -1},y_{j}]
    .\] 
    \[
        (\xi_{i},\mu_{j})  \text{оснащенное разбиение}
    .\] 
    \[
    \Delta x_{i} = x_{i} - x_{i - 1}
    .\] 
    \[
    \Delta y_{j} = y_{j} - y_{j-1}
    .\] 
    \[
    S(f,( \tau,\xi )) = \sum f(\xi_{i},\mu_{j}) \Delta x_{i} \Delta y_{j}
    .\] 
\end{definition}
\subsection{Свойства интегральных сумм}
\begin{enumerate}
    \item 
\[
S(\alpha f, (\tau,\xi)) = \sum \alpha f( \xi_{i},\mu_{j}) \Delta x_{i} \Delta y_{j}
.\] 
\item 
    \[
    f,g : I \to \mathbb{R}^2
    .\] 
    \[
    S(f + g, (\tau,\xi)) = \sum f(\xi_{i},\mu_{j}) \Delta x_{i} \Delta y_{j}  +
    g(\xi_{i},\mu_{j}) \Delta x_{i} \Delta y_{j}
    .\] 
\item
    \[
    f(x,y) \le g(x,y) \forall  x,y
    .\] 
    \[
    S(f,\tau,\xi) \le  S(g, \tau ,\xi)
    .\]
\end{enumerate}
\[
    \lambda(\tau,\xi) - \text{наибольшая из сторон прямоугольника}
.\] 
\begin{definition}
    пусть функция f задана на двумерном промежутке $\Pi$. Число  $I$ называется интегралом по промежутку $\Pi$, если для любой последовательнсоти оснащенных разбиений, последовательность рангов стремится к 0 последоватлеьность частичных сумм стремится к I
\end{definition}
\begin{theorem}
    Если функция интегрируема, то она ограничена
\end{theorem}
\begin{proof}
    Пусть функция не ограничена сверху
\end{proof}
\[
\int_{\Pi} f
.\] 
\[
\int_{\Pi} f(x,y) dx dy
.\] 
\[
\sum_{i = 1,\dots n , j = 1 \dots m}
.\] 
\section{}
\[
    \Delta_{i,j} = [x_{i-1},x_{i}] \times [y_{j-1},y_{j}]
.\] 
Ранг разбиения -- наибольшая из длин промежутка разбиения.
\begin{definition}[Оснащенное разбиение]
    Оснащенное разбиение - это разбиение в каждом промежутке которого выбрано по точке.
    \[
        (\tau,\xi)
    .\] 
\end{definition}
\begin{definition}
    Пусть  функция задана на промежутке $\Pi$ пусть  $( \tau,\xi)$ оснащенное разбиение промежутка
    Интегральной суммой функции $f$ построенной для разбиение  $(\tau,\xi)$ называется число 
     \[
    S(f,(\tau,\xi)) = \sum_{i = 1\dots n, j = 1 \dots m} f(\xi_{ij}) \Delta x_{i} \Delta y_{j}
    .\] 
\end{definition}
\begin{definition}[Множество меры нуль]
    Пусть $X \subset R^2$ x имеет меру нуль, если $\forall \epsilon > 0 \exists $ 
    последовательность промежутков $\Pi_1,\Pi_2,\dots$ 
    \[
    X \subset \bigcap \Pi
    .\] 
    сумма площадей этих $P$ меньше  $\epsilon$
\end{definition}
\subsection{Упражнение}
\[
\cap,\cup,\setminus
.\] 
множетсв меры нуль есть множетсво меры нуль
\[
A \subset B
.\] 
B имеет меру нуль, значит А имеет меру нуль
\subsection{Упражнение}
\[
\partial 
.\] 
граница
\begin{enumerate}
    \item $\partial(A \cup B) \subset \partial(A) \cap \partial(B)$
\end{enumerate}
\subsection{Интеграл по промежутку}
Пусть $(\tau^{(n)},\xi^{(n)})$ последовательность оснащенных разбиений.
Будем говорить, что эта последовательность измельчающаяся, если последовательность рангов 
\[
\lambda(\tau^{(n)},\xi^{(n)}) \to 0
.\] 
\begin{definition}
Пусть функция задана на промежутке $\Pi$. Функция  $f$ называется интегрируемой если
\[
    \exists  I \forall \text{~измельчающийся последовательности} (\tau^{(n)},\xi^{(n)})
.\] 
\[
S(f,(\tau^{(n)},\xi^{(n)})) \to I
.\] 
\end{definition}
Если $f$ интегрируема на множетстве  $\Pi$  $I$ называется интегралом по промежутку  $P$
 \[
I = \int\limits_{\Pi}^{} f 
.\] 
Другие обозначения
\[
\int\limits_{\Pi}^{} f(x,y)  dx dy
.\] 
\[
\iint\limits_{\Pi} f(x,y) dx dy
.\] 
\[
    \int\limits_{\Pi}^{} f(x) dx \text{~x - вектор} 
.\] 
\subsection{Пример}
\[
    \Pi = [a,b] \times [c,d]
.\] 
\[
f(x,y) = C
.\] 
\[
\int\limits_{\Pi}^{}  f(x,y) dxdy = C(b-a)(d-c)
.\] 
\subsection{Пример}
\[
    \Pi = [0,1] \times [0,1]
.\] 
\[
f(x,y) =
\begin{cases}
    0, \text{хотя бы одно число иррационально}\\
    1, \text{оба рациональные}
\end{cases}
.\] 
Докажем, что эта функция неинтегрируема
Сравним два разбиения. Первое оснащение $\xi_{i,j}$ обе координаты рациональны,
второе оснащение $\xi'_{ij}$ одна из координат которых рациональна. Тогда
\[
S(f,(\tau^{(n)},\xi_{ij})) = 1
.\] 
\[
S(f,(\tau^{(n)},\xi')) = 0
.\] 
Множество всех интегрируемых функций на $\Pi$ ,  $R(\Pi)$
\subsection{Простейшие свойства интегралов}
\begin{enumerate}
    \item 
        \[
\int\limits_{\Pi}^{}  \alpha f = \alpha \int\limits_{\Pi}^{} f  
        .\] 
    \item $f,g \in R(\Pi) \implies f + g \in R(\Pi)$
    \item
         $f,g \in R(\pi) f \le  g \implies \int\limits_{\Pi}^{} f \le  \int\limits_{\Pi}^{} g   $
    \item $f \in R(\Pi) \iff$ множество точек разрыва имеет меру нуль
\end{enumerate}
\subsection{Допустимое мноежстево}
\begin{definition}
    Множество $D$ называется допустимым, если его граница  $\partial D$ имеет меру 0
\end{definition}
\subsubsection{Замечание}
Множество D допустимо $\iff$ 
 \[
\forall  \epsilon > 0 \exists M, N
.\]
$M \subset D \subset N$,  площаль  $N$ - площаль  $M$  $< \epsilon_2 $
\subsection{Определение интеграла по допустимому множеству}
\begin{lemma}
    Пусть $D$ допустимое множество, f функция заданная на  $D$, пусть  $\Pi_1,\Pi_2$
    \[
    D \subset \Pi_1, D\subset \Pi_2
    .\] 
    Рассмотрим функции $\overline{f_1}$
    \[
    \overline{f_1} = 
    \begin{cases}
        f(x,y), x \in \Pi_1\\
        0, x \notin \Pi_1
    \end{cases}
    .\] 
    \[
    \overline{f_2} = 
    \begin{cases}
        f(x,y) , x\in \Pi_2\\
        0, x \notin \Pi_2
    \end{cases}
    .\] 
    Тогда
    \[
    \int\limits_{\Pi_1}^{}  \overline{f_1} = \int\limits_{\Pi_2}^{}   \overline{f_2}
    .\] 
\end{lemma}
\begin{proof}
    Рассмотрим  $\Pi = \Pi_1 \cap \Pi_2$
    \[
    \exists  \int\limits_{\Pi}^{} f \iff \exists  \int\limits_{\Pi_1}^{}   f
    .\] 
    \[
    \overline{f(x,y)} = 
    \begin{cases}
        f(x,y), (x,y) \in D\\
        0 , (x,y) \notin D
    \end{cases}
    .\] 
    $\overline{f}$ интегрируема на $\Pi$  $\iff$ множество точек разрыва имеет меру 0,

     $\overline{f_1}$ интегрируемо на $\Pi_1$ $\iff$ множество ее точек разрыва имеет меру 0.
     Докажем
      \[
     \int\limits_{\Pi_1}^{} \overline{f_1}  = \int\limits_{\Pi}^{} \overline{f} 
     .\] 
     Тогда для вычисления интеграла можно рассмотреть какую нибудь одну измельчающуюся последовательность оснащенных разбиений. Тогда интегральная сумма каждого такого разбиения, совпадает с некоторой интегральной суммой функции
\end{proof}
\begin{definition}
    Пусть функциия $f$  задана на допустимом множестве  $D$, пусть  $\Pi$ произвольный промежуток, такой что  $D \subset \Pi$ рассмотрим функцию  $\overline{f}$ заданную на $\Pi$
     \[
    \overline{f}(x,y) = 
    \begin{cases}
        f(x,y) , (x,y) \in D \\
        f(x,y) , (x,y) \notin D
    \end{cases}
    .\] 
    Функция $f$ назывется интегрируемой на $\Pi$, и интеграл функции f по множеству  $D$ называется   $\int\limits_{\Pi}^{} \overline{f} $
\end{definition}
\subsection{Свойства интеграла по произвольному допустимому множества}
\begin{enumerate}
    \item $f \in R(D)$   $\iff$  $\alpha f \in R(D) \land \int\limits_{D}^{} \alpha f = \alpha \int\limits_{D}^{} f  $
     \item $f,g \in R(D) \iff f + g \in R(D) \land \int\limits_{D}^{} (f+g) = \int\limits_{D}^{} f + \int\limits_{D}^{} g   $
     \item $f \le g $ на D $\iff$  $\int\limits_{D}^{} f \le  \int\limits_{D}^{} g  $
    \item Пусть множество $D$ имеет меру нуль, тогда  $\int\limits_{D}^{} f = 0 $
    \item Пусть D допустимое множество, $f,g \in R(D)$, пусть множество точек из  $D$ в которых  $f(x,y) \neq g(x,y)$ имеет меру нуль, тогда $\int\limits_{D}^{} f = \int\limits_{D}^{} g  $
\end{enumerate}
\begin{proof}
    \begin{enumerate}
        \item Самостоятельно
        \item Пусть A множество точек разрыва f, B множество точек разрыв g. Множество точек разрыва $f+g$ лежит в  $A \cup B$
            \[
            \int\limits_{D}^{}   (f + g) = \int\limits_{\Pi}^{}  \overline{f + g} = 
            \int\limits_{\Pi}^{}   \overline{f} + \int\limits_{\Pi}^{}   \overline{g}
            .\] 
    \end{enumerate}
\end{proof}

    \section{Доказательство свойства 4}
    Пусть $f \in R(D)$ Множество $D$ имеет меру 0, тогда  $\int\limits_{D}^{} f = 0 $ \\
    \[
    D \subset \Pi
    .\] 
    \[
    \overline{f} = 
    \begin{cases}
        f(x,y) , (x,y) \in D \\
        0, (x,y) \notin D
    \end{cases}
    .\] 
    \[
    \int\limits_{D}^{} f = \int\limits_{\Pi}^{} \overline{f}   =  \lim S(\overline{f},(\tau^{(n)},\xi^{(n)}))
    .\] 
    для которых  $\lambda(\tau^{(n)}) \to 0$\\
    Известно, что $\overline{f} \in R(\Pi)$, то последовательность оснащенных разбиений, моно выбирать как хотим. D имеет меру нуль, то для кадого $t^{(n)}$ в любом промежутке разбиения. В прямоугольник $\varDelta_{ij}$ выберем точку $\xi_{ij} \notin D$ Тогда
    \[
        S(\overline{f},(\tau^{(n)},\xi^{(n)})) = \sum_{i,j} f(\xi_{ij}) \text{площ} \varDelta_{ij} = 0
    .\] 
    \section{Доказательство свойства 5}
    \[
    f,g \in R(D)
    .\] 
    Множество точек, где $f(x,y) \neq g(x,y)$ имеет меру нуль, то 
    \[
    \int\limits_{D}^{} f = \int\limits_{D}^{} g  
    .\] 
    \[
        \int\limits_{D}^{} f = \int\limits_{D}^{} g \iff \int\limits_{D}^{} (f - g)  =0
    .\] 
    \section{Аддитивность}
    пусть $D_1,D_2$ допустимые множества
    \begin{enumerate}
        \item $D_1 \cup D_2$ , $D_2 \cap D_1$ допустиыме множесва
        \item $f \in R(D_1 \cup D_2) \iff f \in R(D_1) , \in R(D_2) \implies f \in R(D_1 \cap D_2)$
        \item $\int\limits_{D_1 \cup D_2}^{} f = \int\limits_{D_1}^{} f + \int\limits_{D_2}^{} f - \int\limits_{D_1 \cap D_2}^{} f      $ 
        \item Если $D_1 \cap D_2$ имеет меру 0, то 
            \[
            \int\limits_{D_1 \cup D_2}^{}   = \int\limits_{D_1}^{} f + \int\limits_{D_2}^{}  f
            .\] 
    \end{enumerate}
    \subsection{Доказательство}
    \[
        \partial(D_1 \cup D_2) \subset \partial D_1 \cup \partial D_2
    .\] 
    \begin{enumerate}
        \item 
        \item 
            \[
                f \in R(D_1 \cup D_2) \iff D_1 \cup D_2 ~\text{Допустимо}
            .\] 
            и множестсво точек разрыва имеет меру нуль
            \[
            \begin{cases}
                f \in R(D_1)\\
                f \in R(D_2)
            \end{cases}
            \iff 
            \begin{cases}
                D_1,D_2 -- \text{Допустимы}\\
                \text{Множество всех точек разрыва имеет меру нуль}
            \end{cases}
            .\] 
            \item
            Рассмотрим прямоугольник, содержащий как $D_1$, так и $D_2$
            Строим следущие функции, $\overline{f_1},\overline{f_2},\overline{f_3},\overline{f}$
            \[
            \overline{f} = 
            \begin{cases}
                f, \text{на} ~ D_1 \cup D_2\\
                0 
            \end{cases}
            .\] 
            \[
            \overline{f_1} = 
            \begin{cases}
                f, \text{на} D_1\\
                0
            \end{cases}
        \]
        \[
    \overline{f} = \overline{f_1} + \overline{f_2} - \overline{f_3}
        .\] 
        Пусть $(x,y) \in \Pi$
         \begin{enumerate}
            \item $(x,y) \notin D_1 \cap D_2$ все функции 0
            \item $(x,y) \in D_1 \land (x,y) \notin D_2$
                \[
                \overline{f}(x,y) = f(x,y)
                .\] 
                \[
                \overline{f_1(x,y)} = f(x,y)
                .\] 
                \[
                \overline{f_2(x,y)} = 0
                .\] 
                \[
                \overline{f_3}(x,y) = 0
                .\] 
                Равенство выполнилось
        \end{enumerate}
        \[
        \int\limits_{D_1 \cup D_2}^{} f = \int\limits_{\Pi}^{} \overline{f} =
        \int\limits_{\Pi}^{}  (\overline{f_1} + \overline{f_2} - \overline{f_3}) =
        \int\limits_{D_1}^{} f +  \int\limits_{D_2}^{} f - \int\limits_{D_1 \cup D_2}^{} f  
        .\] 
        \item
    \end{enumerate}
    \section{Вычисление двойного интеграла}
    Пусть $f: \Pi \to \mathbb{R}$
    \[
    \iint\limits_{\Pi} f(x,y) dx dy \int\limits_{a}^{b} dx \int\limits_{c}^{d} f(x,y)  dy
    .\] 
    \[
    \iint\limits_{\Pi} f(x,y) dx dy = \int\limits_{c}^{d} dy \int\limits_{a}^{b} f(x,y)   dx
    .\] 
    \section{}
    \[
        \int\limits_{0}^{2} dy \int\limits_{y^2}^{y+2} f(x,y)   
    .\] 
\section{Замечание}
\[
S(f,(\tau^{(n)},\xi)) = \sum_{i=1\dots n,j= 1 \dots n} f(\xi_{i},\mu_{j}) \varDelta x_{i} \varDelta y_{j}
.\] 
\section{Полярные Коорднаты}
$(r,\phi)$
 \[
     x = r \cos{\phi}
.\] 
\[
    y = r \sin {\phi}
.\] 
\subsection{Пример}
\[
\iint_{D} y dx dy
.\] 
\[
    D = \{(x,y) \mid x^2 + y^2 < 2x , x\ge y\}
.\] 
Как делать без полярных все понятно
\begin{proof}
    
\end{proof}
\section{Замена переменной в определнном интегале}
Вспомним первый курс 
\[
\int\limits_{a}^{b} f(g(x)) g'(x) dx  = \int\limits_{g(a)}^{g(b)}  f(t) dt
.\] 
\begin{theorem}
    g непревно дифференцируемая ($g'$ дифференцируема) $g$ обратима
\end{theorem}
\begin{theorem}
    \[
    \iint_{D_{x,y}}
    .\] 
    \[
    \iint_{D_{u,v}} f(u,v) du dv
    .\] 
    \[
        ( g_1,g_2 )_{(x,y)} \to (g_{1}(x,y),g_{2}(x,y))
    .\] 
    \[
    \iint_{D} f(g_1(x,y),g_2(x,y)) |J| dx dy = 
    \iint_{G_{u,v}} f(u,v) du dv
    .\] 
    \[
    ( g_1,g_2 ) :  D_{x,y} \to G_{u,v}
    .\] 
    \[
        (x,y) \to (g_1(x,y),g_2(x,y))
    .\] 
    \[
        J = 
        \begin{vmatrix}
            \partial_{x} g_1 & \partial_{y} g_1\\
            \partial_{x} g_2 & \partial_{y} g_2
        \end{vmatrix} 
    .\] 
    \[
    du = \partial_{x} g_1  dx + \partial_{y} g_1 dy
    .\] 
    \[
    dv = \partial_{x} g_2 dx + \partial_{y} g_2 dy
    .\] 
    \[
    f: (x,y) \to (f_1(x,y),f_2(x,y))
    .\] 
    \[
    \overline{x_0} = (x_0,y_0)
    .\] 
    \[
    \overline{h}=  (h,kj)
    .\] 
    \[
    f(\overline{x_0} + \overline{h}) - f(\overline{x_0}) = L(\overline{h}) + \dots
    .\] 
    \[
    f_1(x_0+h,y_0+k) - f_1(x_0,,y_0) = \partial_{x}f_1(x_0,y_0) h + \partial_{y} f+f_1(x_0,y_0)k + \dots
    .\] 
    \[
  L(h,k) =   \begin{pmatrix} 
    \partial_{x} f_1(x_0,y_0)h + \partial_{y}  f_1(x_0,y_0)k \\
    \partial_{x} f_2(x_0,y_0)h + \partial_{y} f_{2}(x_0,0)
    \end{pmatrix}  =
    .\] 
    \[
    \begin{pmatrix} 
        \partial_{x}  f_1(x_0,y_0)  & \partial_{y} f_1(x_0,y_0) \\
        \partial_{x} f_2(x_0,y_0) & \partial_{y} f_{2}(x_0,y_0) 
    \end{pmatrix} 
    \begin{pmatrix} 
        h\\
        h\\
    \end{pmatrix} 
    .\] 
\end{theorem}
\[
\iint_{D} xy dx
.\] 
\[
    D = \{(x,y)  \mid| x + 2y|  \ge  3, |x - y| \le  3\}
.\] 
\[
u = x + 2y
.\] 
\[
v = x - y
.\] 
\[
J = \begin{vmatrix}
 1 & 2\\
 1 & -1
\end{vmatrix}  = -1 - 2 = -3
.\] 
\[
u - v = 3 y
.\] 
\[
u + 2v = 3 x
.\] 
\[
xy = \frac{(u - v)(u + 2v)}{9}
.\] 
\end{document}

