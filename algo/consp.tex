\documentclass[14pt]{extarticle}
\usepackage{fontspec}
\usepackage[russian, english]{babel}
\setmainfont{Times New Roman}
\usepackage{amssymb}
\usepackage{setspace}
\usepackage{graphicx}
\onehalfspacing
\usepackage{amsmath}
\usepackage{listings}
\usepackage{indentfirst}
\setlength{\parindent}{1.25cm}
\usepackage[right=10mm,left=30mm,top=20mm,bottom=20mm]{geometry}
\newtheorem{theorem}{Теорема}
\newtheorem{definition}{Определение}
\newtheorem{example}{Пример}[definition] \newtheorem{corollary}{Следствие}[theorem] \newtheorem{exmp}{Пример}[theorem]
\newtheorem{lemma}[theorem]{Лемма}
\title{Вопросы к зачоту по структурам}
\author{}
\date{}
\begin{document}
    \maketitle
    \section{Асимптотические оценки времени работы алгоритмов и их смысл}
    Пусть $f,g$ функции принимающие только положитльные значения
    \begin{definition}[Тета]
        $f = \Theta(g) : \exists c_1,c_2 > 0  n_0 : \forall  n > n_0 ~ c_1 g(n) \le f(n) \le  c_2 g(n)$ 
    \end{definition}
    \begin{example}
        \[
        \frac{n^2}{2} - 3n = \Theta(n^2)
        .\] 
        Чтоб доказать нужно найти нужные константы
        \[
        c_1 n^2 \le  \frac{n^2}{2} - 3n \le c_2 n^2
        .\] 
        \[
        c_1 < \frac{1}{2} - \frac{3}{n} \le  c_2 
        .\] 
        \[
        c_1 = 1/4
        .\] 
        \[
        c_2 = 1/2
        .\] 
        при $n \ge  12$
    \end{example}
    \begin{example}
        \[
        c_1 2^{n} \le 2^{n + 1} \le  c_2 2^{n}
        .\] 
        \[
        c_1 \le  2 \le  c_2
        .\] 
    \end{example}
    \begin{definition}[O]
        \[
        f = O(g) : \exists  c ~n_0 \forall  n > n_0: f(n) \le  c g(n)
        .\] 
    \end{definition}
    \begin{definition}[Омега]
        \[
        f =  \Omega(n) : \exists  c, n_0 \forall  n > n_0 f(n) \ge  cg(n)
        .\] 
    \end{definition}
    \begin{definition}[o-малое]
        \[
        f = o(g) : \lim_{n \to \infty} \frac{f(n)}{g(n)} = 0
        .\] 
    \end{definition}
    \section{Теорема о рекурсии}
    Есть рекурсивный алгоритм. Его время работы зависит от параметра n (размера входных данных)
    \begin{enumerate}
        \item Если $n < k, k = const$ решаем задачу нерекурсивно относительно n
        \item иначе разбиваем задачу на  $a$ подзадач размера  $\frac{n}{b}$ каждую решаем рекрсувно
    \end{enumerate}
    Время работы такого алгоритма описываем такой формулой
    \[
    T(n) = 
    \begin{cases}
        g(n), n < k\\
        a * T(\frac{n}{b}) + f(n), n\ge  k
    \end{cases}
    .\] 
    $g(n)$ время для нерекурсивного решения ,  $f(n)$ время для определения подзадач и собирания результатов рекурсивных вызовов
     \begin{theorem}
        Пусть время работы рекурсивного алгоритма выражается формулой
        \[
        T(n) = aT(\frac{n}{b}) + f(n)
        .\] 
        \begin{enumerate}
            \item Если $f(n) = O(n^{c}), c< \log_{b}{a}$ то $T(n) = \Theta(n^{\log_{b}a})$
            \item Если $f(n) = \Theta(n^{c} (\log_{b}a)^{k}$ для $k\ge 0, c= \log_{b}a$, то $T(n) =\Theta(n^{c} (\log{n})^{k+1})$
             \item $f(n) = \Omega(n^{c}) c > \log_{b}a$, то $T(n) = \Theta(f(n))$ 
                 для это еще должно быть
                 \[
                 a f(\frac{n}{b}) < k f(n) ~ k < 1
                 .\] 
        \end{enumerate}
        Частный случай при $a = b$
         \[
        T(n) = a T(\frac{n}{a}) + f(n)
        .\] 
        \begin{enumerate}
            \item Если $f(n) = O(n^{c}) ,c<1$ то $T(n) = \Theta(n)$
            \item Если  $f(n) =  \Theta(n)$ то  $T(n) = \Theta(n \log{n})$
            \item Если $f(n) = \Omega(n^{c}), c> 1$, То $T(n) = \Theta(f(n))$
        \end{enumerate}
    \end{theorem}
    \begin{exmp}
        \[
        T(n) = 9T(\frac{n}{3}) + n
        .\] 
        \[
            \log_{b}{a} = 2
        .\] 
        \[
        n = O(n) \iff T(n) = \Theta(n^{2})
        .\] 
    \end{exmp}
    \begin{exmp}
        \[
        T(n)=T (\frac{2 n}{3}) +1
        .\] 
        \[
        f(n) = 1
        .\] 
        \[
        a = 1
        .\] 
        \[
        b = \frac{3}{2}
        .\] 
        \[
            \log_{3/2} 1 = 0
        .\] 
    \end{exmp}
    \section{Сортировка слиянием}
    Оценка времени работы $T(n) = 2 T (\frac{n}{2}) + n$ 
    \[
    n = \Theta(n)
    .\] 
    По теореме $T = \Theta(\log n)$
    \section{Длинная арифметика}
    Число представляем как список $int$, каждое число меньше некого числа, которое помешается в  $int$. $a = \{a_0,a_1,\dots,a_{n-1}\}$, $\forall i < n -1~ a_{i} < b$
     \[
     a = \sum_{i = 0}^{n-1} a_i b^{i}
     .\] 
     За $b$ удобно брать большую степень 10 , например миллион. Еще удобнее брать большую степень двух как пример  $2^{31}$
     \subsection{Сложение}
     Сложение двух длинных чисел происходит точно так как в столибик. Для сложения нужно сделать $n$ элементарных операций, где  $n$ количесвто цифр самого длинного числа.
    \subsection{Деление с остатотком}
    Опять алгоритм деления в столбик
    \subsection{Умножение}
    Если умножать в столбик, придется сделать  $n^2$ элементарных операций
    \subsubsection{Алгоритм карацубы}
    Пусть есть числа $a,b$ длины  $n$,  $c$ основание системы счисления
     \[
    x = c^{n/2}
    .\] 
    \[
    a = \alpha_1 x + \alpha_2
    .\] 
    \[
    b = \beta_1 x + \beta_2
    .\] 
    \[
    ab = \alpha_1 \beta x^2 + \alpha_1 \beta_2 x + \alpha_2 \beta_1 x + \alpha_2  \beta_2 = \alpha_1 \beta_1 x^2 + (\alpha_1 \beta_2 + \alpha_2 \beta_1) x + \alpha_2 \beta_2
    .\] 
    \[
    T(n) = 4 T(n/2) + O(n)
    .\] 
    \[
    \log_{2}4 = 2
    .\] 
    \[
    T(n) = \Theta(n^2)
    .\] 
    \[
    \alpha_1 \beta_1 x^2 + (\alpha_1 \beta_2 + \alpha_2 \beta_1) x + \alpha_2 \beta_2 = \alpha_1 \beta_1 x^2 + ((\alpha_1 + \beta_1)(\alpha_2 + \beta_2) -\alpha_1 \beta_1 -\alpha_2 \beta_2) + \alpha_2 \beta_2
    .\] 
    \[
    T(n) = 3 T(\frac{n}{2}) + O(n)
    .\] 
    \section{Алгоритм рабина карпа}
    Есть строка, есть подстрока, есть хешфункция, все хеши подстрок нужной длины сранвивем с хешем искомой, если равны, сравниваем посимвольно. Для опитимизации, надо сделать хорошую хеш функцию, чтоб было малок коллизий, чтобы она например зависела от позиций.
    \section{Грамматики,Регулярные выражения}
    Можно определить язык через регулярные выражения. Рассмотрим регулярки в джаве. В джаве есть класс Pattern и класс Mathcher. Pattern содержит компилированные выражения, Mathcer ищут в тексте штуки по регулярке.

    Рассмотрим задачу, есть строка $s$, нужно проверить подходит ли под регулярку 
 \begin{lstlisting}[language=Java] 
 s.mathcher("kjdskjdsk");
\end{lstlisting} 
    такая фигня возвращает boolean
\section{Задача}
Есть две квадратные матрицы, хотим сосчитать произведение. За какое время можем сделать
Наивный алгоритм   $\Theta(n^{3})$
\subsection{Алгоритм Штрассена для умножения матриц}
Сначала рассмотрим умножение комплексных чисел
 
\[
c_1 = (a + bi )
.\] 
\[
c_2 = (c + di)
.\] 
Хотим меньше умножений
\[
A_1 = (a + b)(c - d)
.\] 
\[
A_2 = ad
.\] 
\[
A_3 = bc
.\] 
\[
    (ac - bd) = (ac + bc - ad - bd) + ad - bc = A_1 + A_2 - A_3
.\] 
\[
ad + bc = A_2 + A_3
.\] 
Мы уменшили количество умножений, за счет увеличений количеств сложений.\\
Пусть есть две вещественно значные матрицы, считаем что размер матрицы есть степень 2, если не степень, дополняем нулями.
Каждую матрицу делим на 4 квадрата
\[
A = \begin{pmatrix} 
    a & b\\
    c& d\\
\end{pmatrix} 
.\] 
\[
B = \begin{pmatrix} 
    e & g \\
    f & h\\
\end{pmatrix} 
.\] 
\[
A \times B = 
\begin{pmatrix} 
    a & b\\
    c & d\\
\end{pmatrix} 
\begin{pmatrix} 
    e & g\\
    f & h
\end{pmatrix} 
=
\begin{pmatrix} 
    ae + bf & ag + bh\\ 
    ce + df & cg + dh
\end{pmatrix} 
.\] 
Потребовалось 8 матричных умножений и $n^2$ сложений
\[
T(n) = 8 T(\frac{n}{2}) + \Theta(n^2)
.\] 
по теореме о рекурии $T(n) = \Theta(n^{3})$. Херня полная, нужно уменшить рекурсивные вызовы до 7
\[
r = ae + bg
.\] 
\[
s = ag + bh
.\] 
\[
t = ce + df
.\] 
\[
u = cg +dh
.\] 
\begin{tabular}{|c|c|c|c}
    \hline
    i & $A_{i}$ & $B_{i}$ & $P_{i} = A_{i} \times B_{i}$ \\
    \hline
    1 & a & g - h & ag - ah\\
    2 & $a+b$ & h &  $ah + bh$
\end{tabular}
Бля, лень в випедии посмотрю\\
У булевых матриц используют прикол, так как у булевых операций нет вычитания. Булевы значения меняют на $0,1$ перемножают ка числовые и ненули заменяют на true
\section{Алгоритм без полиноминального решения}
Такие алгоритмы при больших числа не имеют смысла, поэтому используют приближенный алгоритм
\subsection{Задача коммивояжера}
Есть связный граф, вершины соеденны ребрами. Есть вершина, надо пройти по всем хотя бы  один раз и вернуться в исходную. Так е все ребра имееют вес, нужен минималльны
\includegraphics{g.pdf}
Задача, решается , если в графе выполняется неравенство треугольника
\begin{enumerate}
    \item Можно пребрать все гамильтоновы циклы, очень долго и гавно вообще.
    \item Жадный алгоритм, дает рещультат не более чем в два раза хуже оптимального.
    \item Построение пути по минимальному скелету, не более чем в два раза хуже оптимального
    \item Линейное программирование, но может скатиться в полнй перебор
\end{enumerate}
\subsection{Жадный алгоритм}
Жадный алгоритм на каждом шаге пытается доавить в имеющийся цикл одну вершину, ближайшую к циклу. Этот алгоримт хорошо работает в случае полного графа
\section{Слова и алфавиты}
\[
    \text{Алфавит~} A = \{a_1,a_2,\dots,a_{n}\}
.\] 
\[
b \in A
.\] 
множество симвлов (букв)
\[
    \text{Слово~} \alpha = a_1 a_2 \dots a_{k}
.\] 
Последоватлеьность символов
\[
\mid \alpha \mid := k
.\] 
длина слова
\[
\mid \epsilon \mid := 0
.\] 
пустое слово, не содержит символов
\[
A^{k} 
.\] 
множество слов над алфавитом $A$ длины  $k$
Введем операцию котенации (конкатенации)
 \[
\alpha = a_1 a_2 \dots a_{n} \in A^{n}
.\] 
\[
\beta = b_1 b_2 \dots b_{m} \in A^{m}
.\] 
\[
\alpha \beta = a_1 a_2 \dots a_{n} b_1 b_2 \dots b_{m} \in A^{n+m}
.\] 
\[
\alpha \epsilon = \epsilon \alpha = \alpha
.\] 
\[
    A^{+} := \bigcup \limits_{i = 1}^{\infty} A^{i}
.\] 
Множество всех непустых слов над алфавитом $A$
\[
    A^{*} := \bigcup \limits_{i = 0}^{\infty} A^{i}
.\] 
Множество всех слов над алавитом $A$ 
\[
\alpha \in A^{*}, \beta \in B^{*} \alpha \beta \in (A \cup B)^{*}
.\] 
\[
    L \subset A^{*} \text{~Язык над алфавитом}
.\] 
\[
L, M \subset A^{*}
.\] 
\[
    LM = \{\alpha \beta \mid \alpha \in L , \beta \in M\}
.\] 
котенация языков
\subsection{Регуляные язык}
Рассмотрим алфавит $A = \{a,b,\dots,\}$. Регулярные выражения включают символы языка и операции объединения и котенации
 \begin{enumerate}
     \item Выражение $a \in A$ задает язык  $L = \{a\}$
    \item  $E_1$ выражение задает язык $L_1$, $E_2$ задает язык $L_2$ 
        \[
            E_1 | E_2 ,\text{задает~} L_1 \cup L_2
        .\] 
    \item $E_1 E_2$ задает $L_1 L_2$
    \item $*,+$ переносятся с выражений на языки
\end{enumerate}
\subsubsection{Примеры}
\begin{enumerate}
    \item $A = {a,b,c}, E = a(a|b|c|)*c$ задает язык, определяющй множество слов, которые начинаются на a,кончаются на с 
         \[
        A = a|b|c
        .\] 
        \[
        E = a A^{*}c
        .\]
    \item 
        \[
            \lambda = \{\epsilon\}
        .\] 
        \[
            [A] = A|\lambda
        .\] 
        \[
            A^{[k]} = \bigcup \limits_{i = 0}^{k} A^{i}
        .\] 
        \[
            A = \{0,1,2,\dots,9,+,-,E,.\}
        .\] 
        Задаем регулярку для правильных вещественных чисел
        \[
        D = 0|1|\dots|9
        .\] 
        \[
            [+|-]D^{+}[.D^{+}][E[+-]D^{+}]
        .\] 
\end{enumerate}
\subsection{Графическое задание регулярного выражения}
Регулярное выражение можно задать с помощью графа, где есть сток и исток
Дуга может быть помечена символом.
\includegraphics{g1.pdf}
\includegraphics{g2.pdf}\\
\includegraphics{g3.pdf}
\subsection{Анализ регулярного выражения}
\begin{enumerate}
    \item Строим граф регулярного выражения
    \item В слове рассматриваем последовательно каждый символ
    \item Просматриваем все пути в графе, отбрасываем тупиковые
\end{enumerate}
Если хоть один путь привел в сток, слово принадлежит языку
\section{Формальные грамматики}
Формальная Грамматка $G$ это совокупность следущиих объектов
 \begin{enumerate}
    \item Терминальный (основной) алфавит $T$
    \item Нетерминальный (вспомогательный алфавит) $N$  $T \cap N = \emptyset, T \cup N = V$
    \item Начальный символ  $I \in N$
    \item Конечное множество  $\Pi$ правил вида  $\alpha \to \beta , \alpha \in V^{+},\beta \in V^{*}$ $\alpha$ содержит хотя бы один нетерминальный символ
\end{enumerate}
\subsection{Вывод слов в формальных грамматиках}
Вывод слова в грамматике -- последовательность слов $\omega_1,\dots,\omega_{k}$,
$\omega_1 = I$ начальное слово $\omega_{k}$ состоит только из терминальных символов
\subsection{Пример}
\begin{enumerate}
    \item Терминальны алфавит $\{0,1\}$
    \item  Нетерминальный алфавит  $\{A\}$
    \item  $I = A$
    \item  $\{A \to AA, A \to 01, 1A \to A1\}$
\end{enumerate}
\[
A \to AA \to A01 \to AA01 \to 01A01 \to 0A101 \to 001101
.\] 
\subsection{Классы грамматик}
\begin{enumerate}
    \item $K_0$ грамматики общего вида $\alpha \to \beta , \alpha \in V^{+},b \in V^{*}$
     \item $K_1$ контекстно-зависимые грамматики $\alpha A \beta \to \alpha \gamma \beta, \alpha,\beta \in V^{*},A \in N,\gamma \in V^{+}$
     \item $K_2$ контекстно-свободные грамматики $A \to \gamma, A \in N,\gamma \in V^{+}$
     \item $K_3$ автоматные(регуляные) грамматики $A \to \gamma B, \lor A \to \gamma,
         A, B \in N,\gamma\in T^{+}$
\end{enumerate}
\subsection{Дерево вывода}
Рассматриваем только контекстно-свобоодные грамматики
\[
    T = \{i,d,(,),+,-,*\}
.\] 
\[
    N = \{E\}
.\] 
\begin{enumerate}
    \item $E \to E + E$
    \item  $E \to E - E$
    \item  $E \to E*E$
    \item  $E \to (E)$
    \item  $E \to i$
    \item  $E \to d$
\end{enumerate}
Вывод слова $i + d * i + d$
 \begin{enumerate}
    \item $E$
    \item  $E + E$
    \item  $E + E +E$
    \item  $E + E * E + E$
    \item  $i +E * E +E$
    \item  $i + d * E + E$
    \item  $i + d * i + E$
    \item  $i + d * i + d$
\end{enumerate}
\includegraphics[scale=0.5]{tree.pdf}\\
дерево вывода не единственное.\\
Определим однозначную грамматику для этого языка
\[
    N = \{E,T,P\}
.\] 
\begin{enumerate}
    \item $E \to E + T$
    \item  $E \to E - T$
    \item  $E \to T$
    \item  $T \to T * P$
    \item  $T \to P$
    \item  $P \to (E)$
    \item  $P \to i$
    \item  $P \to d$
\end{enumerate}
слово $i + d* i + d$
 \begin{enumerate}
    \item $E$
    \item  $ E + T$
    \item  $E + T + T$
    \item $E + T * P + T$
    \item  $T + T * P + T$
    \item  $P +T * P + T$
    \item  $P + P * P + T$
     \item $P + P * P + P$
     \item $i + P * P + P$
     \item $i + d * P + P$
    \item $i + d * i +P$
    \item  $i + d * i + d$
\end{enumerate}
\subsection{Автоматные грамматики}
\subsubsection{Пример}
\[
    T = \{a,d\}
.\] 
\[
    N = \{D,H\}
.\] 
\begin{enumerate}
    \item $D \to aH$
    \item $D \to a$
    \item  $H \to aH$
    \item  $H \to dH$
    \item  $H \to a$
    \item  $H \to d$
\end{enumerate}
слово $aadad$
\begin{enumerate}
    \item $D$
    \item  $aH$
    \item  $aaH$
    \item  $aadH$
    \item  $aadaH$
    \item  $aadad$
\end{enumerate}
Дерево автоматной грамматики можно представить в виде графа.
\section{Автоматы}
\begin{definition}[Детерминированный конечный автомат]
    \[
    A = <T,N,I,K,F>
    .\] 
    \begin{enumerate}
        \item $T$ термнальный альфавит, алфавит анализируемого слова
        \item  $N$ нетерминальный алфавит или алфавит состояний
        \item  $I \in N$ начальное состояние автомата
        \item  $K \subset N$ множество заключительных состояний
        \item  $F$ детерминированная функция переходов  $F: T \times N \to N$
    \end{enumerate}
\end{definition}
Слово $\alpha = \alpha_0 \alpha_1 \dots \alpha_{k}$ считается допущенным автоматом $A$,
если в последовательности состояний  $Q_0 Q_1 \dots Q_{k} Q_{k + 1}, Q_0 = I, Q_{i + 1} = F(a_1,Q_{i}) Q_{k + 1} \in K$ иначе недопущенно
\begin{definition}[Недетерминированный конечнй автомат]
    тоже самое, но $F : T \times N \to 2^{N}$
\end{definition}
Слово $\alpha = \alpha_0 \alpha_1 \dots \alpha_{k}$ считается допущенным автоматом $D$,
если в последовательности состояний  $Q_0 Q_1 \dots Q_{k} Q_{k + 1}, Q_0 = I, Q_{i + 1} \in F(a_1,Q_{i}) Q_{k + 1} \in K$ иначе недопущенно
\subsection{Пример}
\begin{enumerate}
    \item $T = \{d,s\}$
    \item  $N = \{I,H,E\}$
    \item  $I$
    \item  $K = \{H\}$ 
    \item $F$ задаем с помощью таблицы\\
        \begin{tabular}{|c|c|c|}
            \hline
            & s & d\\
            \hline
            I & H & E\\
            \hline
            E & E & E\\
            \hline
        \end{tabular}
\end{enumerate}
\section{Быстрое преобразование Фурье}
Пусть задан многолчлен с комплексными коээфициентами степени меньше $n$
 \[
     A(x) = a_0 + a_1 x + a_2 x^2 + \dots a_{n - 1} x^{n-1}
.\] 
$O(n^2)$ время работы умножения многочленов. Надо быстро получить значения в точках. 
\[
    A(x) = \sum_{i=0}^{n-1} a_{i} x^{i}
.\] 
\[
    B(x) = \sum_{i=0}^{n-1} b_{i} x^{i}
.\] 
\section{Лямбда исчисление}
\[
\lambda x. e
.\] 
\[
e_1 e_2
.\] 
$+2$ -- функция, которую можно применить к другому числу.
 \[
\lambda x. (\lambda y . + x y)
.\] 
\[
    +~2 ~6 \to 7
.\] 
пример $\delta$ редукции.  
\[
    (\lambda x.e) a \to e_{x}^{a}
.\] 
\subsection{Примеры}
\begin{enumerate}
    \item $(\lambda x. x) 2 \to  2$
    \item  $(\lambda x. x) (\lambda x.x) \to \lambda x.x$
    \item  $(\lambda x. x x)(\lambda x. x x) \to (\lambda x .x x) (\lambda x.x x)$
         Зациклилось все нахуй
\end{enumerate}
У нас два вопроса. Может ли после при применениее редукции в разном порядке, получить новую норальную форму. Нет.
\subsection{Порядки редукции}
\begin{enumerate}
    \item Апликативный, когда стараемся вычислить самые внутренние части, ели есть внешний редекс и внутренний редекс (внутри которого нет ничего), стараемся применить к второму.
    \item Нормальный порядок редукции,  при котором мы применяем редукции к самому левому из самых внешних.
\end{enumerate}
Ленивые вычисления
\[
    (\lambda x. + x x) (* 3~4)
.\] 
\[
+ (* 3~ 4) (* 3 ~4)
.\] 
\[
+ 12~12
.\] 
\[
24
.\] 
\subsection{Как рекурсивную функцию предоставить в  лямбда выражение}
\subsubsection{Факториал}
\[
if ~ e_1 ~ e_2 ~ e_3
.\] 
\[
eq_0
.\] 
\[
f =\lambda n. if ~ (eq_0 n)~1~(*~n~(f~(- n~1)))
.\] 
\[
FB = \lambda f. \lambda n . if ~ (eq_0 n) 1 (* ~ n (~ f (- ~ n ~ 1)))
.\] 
Найти функцию, которая ищет неподвижную точку, такую функцию назовем комбинатор неподвижной точки, $Y$-комбинатор
 \[
Y F = F(Y ~ F)
.\] 
\[
Y ~ FB =  fact
.\] 
Прилумано огромное количество таких комбинаторов\
\[
Y = \lambda h. (\lambda x . h (x ~ x)) (\lambda x.h (x x))
.\] 
\[
Y F = (\lambda x .F(x,x))(\lambda x . F(x,x)) = F ((\lambda x. F(x,x)) (\lambda x . F(x x))) = F(Y ~ F)
.\] 
\[
Y( FB) 2 = FB(Y FB) 2
.\] 
\[
\lambda n . if (eq_0 n ) 1 (* n ((Y ~FB) (n - 1)))(2)
.\] 
\[
* 2 (Y ~ FB) 1
.\] 
\[
* 2 1 (Y FB) 0
.\] 
\[
* 2 ~ 1 ~ 1
.\] 
\subsection{Чистое лямбда исчисление}
\subsubsection{Логика}
\begin{enumerate}
    \item TRUE $\lambda x . \lambda y . x$
    \item FALSE  $\lambda x . \lambda y . y$
    \item IF  $\lambda p . \lambda x . \lambda y . p~x~y$
    \item AND  $\lambda a . \lambda b. \text{IF} a~b~\text{FALSE}$
    \item OR  $\lambda a . \lambda b.  \text{IF} ~ a ~ b ~ TRUE$
    \item NOT  $\lambda p . p \text{FALSE} ~ \text{TRUE}$
\end{enumerate}
\subsection{Списки}
\begin{enumerate}
    \item nil пустой список
    \item cons получает хрень и список выдает список, где голова хрень, хвост список
    \item car -- голова
    \item cdr - хвост
    \item null -- поясняет пустой или нет
\end{enumerate}
\begin{enumerate}
    \item CONS  $\lambda a . \lambda b . \lambda p . p ~ a ~ a$
    \item CAR  $\lambda a . \lambda b $
\end{enumerate}
\section{SKI-исчисление}
\[
I ~ e = e
.\] 
\[
K ~ a ~ b = a
.\] 
\[
S ~ f ~ g ~ x = f ~ x ~ (g ~ x)
.\] 
\[
I = S ~ K ~ K
.\] 
\[
S ~ K ~ K ~  x = K ~ x ~ (K ~ x) = x
.\] 
\section{Функция абстрагирования}
\[
abs(x,E)
.\] 
\[
    [x] E ~ \lambda x.E
.\] 
\[
    [x] e_1 ~ e_2 = S [x] e_1 [x] e_2
.\] 
\end{document}
