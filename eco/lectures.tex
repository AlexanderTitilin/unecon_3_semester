\documentclass[14pt]{extarticle}
\usepackage{fontspec}
\usepackage[russian, english]{babel}
\setmainfont{Times New Roman}
\usepackage{amssymb}
\usepackage{setspace}
\usepackage{graphicx}
\usepackage{pgfplots}
\onehalfspacing
\usepackage{amsmath}
\usepackage{listings}
\usepackage{indentfirst}
\setlength{\parindent}{1.25cm}
\usepackage[right=10mm,left=30mm,top=20mm,bottom=20mm]{geometry}
\newtheorem{theorem}{Теорема}
\newtheorem{corollary}{Следствие}[theorem]
\newtheorem{lemma}[theorem]{Лемма}
\newtheorem{definition}{Определение}
\title{Экономика}
\author{}
\date{}
\begin{document}
\maketitle
\section{05.09}
\begin{definition}
	Микроэкономика изучает поведение отдельных экономических субъектов.
	В центре внимания -- цены и объемы.
\end{definition}
\begin{definition}
	Макроэкономика изучает функционирование экономической системы в целом и крупных ее секторов.
\end{definition}
\subsection{Ограниченность ресурсов}
Задача состоит в том, что экономические агенты являются рациональными и направлены на максимализирующую деятельность. Наша задача посмотреть как это моделируется.
\subsection{Задача}
Есть чел, его задача описывается функцией полезность
\[
	U = (Q  - 5)^{0.6} (20 - L)^{0.3}
	.\]
Технология производства
\[
	Q = 10 L^{0.75}
	.\]
Надо максимализировать
\[
	\Phi = (Q - 5)^{0.6} (20 - L)^{0,3} - \lambda (Q - 10 L^{0,75})
	.\]
Дрочь про фнп, на матане не прошли еще
\subsection{Вопросы экономистов}
\begin{enumerate}
	\item что производить
	\item для кого производить
	\item как производить
	\item когда производить
\end{enumerate}
Решая задачи, экономисты делают разные модели.
\subsection{Рынок}
\begin{definition}
	Рынок -- общественный механизм распределния благ, посредством добровольного обмена.
\end{definition}
Использование денег упрощает обмен.
\subsection{Создание экономической теории}
\begin{enumerate}
	\item Наблюдение экономической деятельности.
	\item Введение понятий, выдвижение гипотез.
	\item Создание научной концепции
	\item Логическая и практическая проверка концепции
	\item Прошла -- удовретворительная теория, иначе уточнить наблюдении, гипотез или концепции.
\end{enumerate}
\subsection{Классификация экономических моделей}
\begin{enumerate}
	\item Частичная или общая
	\item Статическая или динамическая
	\item Оптимизационная или равновесная
	\item Детерминированная или стохастическая (влияет не влияет вероятность)
\end{enumerate}
% \subsection{Проникновение математики в экономику}
\subsection{Теория поведения потребителя}
Полезность, которую получает индивид, можно измерить.
Таблица Менгера.
\section{12.09}
\subsection{Благо антиблаго}
Есть функция полезности, зависит от количеста. Если растет благо, иначе антиблаго.
\subsection{Общая полезность}
полезность от потребления всего набора плаг
\subsection{Предельная полезность}
Прирост полезности при потребление
\subsection{Гипотеза убывания предельной нормы замещения. Первый закон Госсена}
В каждом акте потребления предельная полезность убывает.
\subsection{Второй закон Госсена}
В общем виде максимализация полезности потрибителя имеет следущий вид
\[
	U = U(Q_1,Q_2,\dots,Q_{n}) \to \max
	.\]
\[
	\text{пред} M = \sum_{i=1}^{n} P_1 Q_{n}
	.\]
Математически это сводится к задаче лагрнанжа
\[
	L = U(Q_1,Q_2,\dots,Q_{n}) - \lambda(\sum P_{i} Q_{i} - M)
	.\]
Взяли все частные производные
\[
	\frac{M U_1}{P_1} = \frac{M U_2}{P_{2}} = \dots = \frac{M U_{n}}{P_{n}}  = \lambda
	.\]
$\lambda$ предельная полезность денег.
\subsection{Простейшая функция спроса}
Используя таблицу Менгела или задачу Лагранжа можно вывести функцию индвидуального спроса на товар

Пусть, у нас 2 благаю Потребности описываются функцией вида
\[
	U = (x_1,x_2) = x_1^{a} x_2^{b}
	.\]
\[
	M U_1 = \frac{\partial U}{\partial x_1} = ax_1^{a -1} x_2^{b}
	.\]
\[
	M U_2 = \frac{\partial U}{\partial x_2} bx_1^{a} x_2^{b_{01}}
	.\]
\[
	\frac{ax_1^{a-1} x_2^{b}}{bx_1^{a} x_2^{b -1}} = \frac{ax_2}{bx_1}
	.\]
Из второго закона Госсена
\[
	x_2 = \frac{b p_1}{a p_1} x_1
	.\]
\[
	x_1 = \frac{a m}{(a + b) p_1}
	.\]
\[
	p_1 x_1 + p_2 x_2 = m
	.\]
\subsection{Еще функция полезности. Функция стоуна}
\[
	U = (Q_{a} - k)^{\alpha} (Q_{b} - l)^{\beta} (Q_{c} - m)^{\gamma}
	.\]
\[
	M = P_{a} Q_{a} + P_{b} Q_{b} + P_{c} Q_{c}
	.\]
\[
	\Phi = U - ()
	.\]
\subsection{Квазилинейная функция}
\[
	U = Q_{F} + \sqrt{Q_{G}}
	.\]
\[
	\Phi = Q_{F} + \sqrt{Q_{G}} - \lambda (P_{F} Q_{F} + P_{G} Q_{G}  - M)
	.\]
\subsection{Кривые безразличия}
\begin{enumerate}
	\item Кривая безразличия содержит все одинаково предпочтительные наборы благ
	\item один и тот же уровень полезноcти
\end{enumerate}
\subsection{Потребительский выбор}
\[
	P_{c} = 2, P_{F} = 1, M = 80
	.\]
\subsection{Еще задача}
Дано
\[
	M = 60 , P_{c} = 2, P_{f}
	.\]
\[
	U = C^{0.5}F^{0.25}
	.\]
\[
	\Phi = C^{0,5} F^{0,25} - \lambda(2C +F - 60)
	.\]
\[
	\begin{cases}
		\frac{0.5 F ^{0,25}}{C^{0,5}} = 2\lambda \\
		\frac{0,25 C^{0,5}}{F^{0,75}}= \lambda
	\end{cases}
	.\]
\[
	0.125 F^{-0.5} = 2 \lambda^2
	.\]
\subsection{Производная или эластичность}
Мы исследуем спрос на картошку. Приувелечение цены на 10 копеек на кг, объем спроса снижается на 10 кг в год
\[
	\frac{dQ}{dP} = - \frac{10}{0,1} = -100 \frac{\text{кг}^2}{\text{руб}*\text{год}}
	.\]
Производная зависит от единиц измерения благ. Заменим картошку на водку и не сможем сравнить.
Для избеганий такой проблемы юзают понание эластичсности
\[
	e  = \frac{d Q}{d P} * \frac{P}{Q}
	.\]
измеряется проблема
\subsection{Сорта эластичности}
\begin{enumerate}
	\item Точечная или дуговая (если функция непрерывная или дискретная)
	\item Пряма -- по цене данного блага, перекрестная -- по цене другого блага.
	\item По доходу
\end{enumerate}
\subsection{Перекрестная эластичность}
\[
	e_{i,j}^{D} = \frac{\varDelta Q_{i}}{\varDelta P_{j}} * \frac{P_{j}}{Q_{i}}
	.\]
\begin{enumerate}
	\item $e_{ij}^{D} > 0$  взаимозаменяемые блага
	\item $e_{ij} < 0$ взаимодополняемые блага
\end{enumerate}
\section{Задачи}
\subsection{}
\[
	Q_{x} = A - BP
	.\]
P - цена\\
\[
	Q_{x} = 40,  P = 10
	.\]
\[
	\epsilon = -2
	.\]
Надо расчитать излишек потребителя
\[
	\epsilon = \frac{d Q}{d P} * \frac{P}{Q} = -2
	.\]
\[
	-B  * \frac{10}{40} =  B * \frac{1}{4} = 2
	.\]
\[
	B = 8
	.\]
\[
	40 = A - 80
	.\]
\[
	A = 120
	.\]
\[
	Q_{x} =120 -8P
	.\]
\[
	P = \frac{Q_{x} - 120}{-8} = - \frac{Q_{x}}{8} + 15
	.\]
\[
	Q_1 = 1, P_1 = 14,875
	.\]
\[
	Q_2 = 2 , P_2 = 14,75\dots
	.\]
\[
	Q_{40} = 40, P_{40} = 10
	.\]
\[
	RD = 100 - \text{Выгода}
	.\]
\subsection{Задача про ценность денег}
\[
	U(c_1,c_2)
	.\]
\[
	A = (c_1,c_2)
	.\]
\[
	B = (500,500)
	.\]
\[
	\varDelta = 500 - c_1 \text{-- сбережения}
	.\]
\[
	c_2 = I_2 + (I_1 - c_1) (1 + i)
	.\]
i -- ставка в процентах
\[
	PV = 1 + i
	.\]
\[
	FV =  I_1 (1 + i)
	.\]
\begin{definition}[Коэфициент Дисконтирования]
	\[
		\frac{1}{1 + i}
		.\]
\end{definition}
\begin{definition}[коэфициент наращения]
	\[
		1 + i
		.\]
\end{definition}
\[
	FV = 500 * 2 = 1000
	.\]
\[
	PV = \frac{500}{2} = 250
	.\]
\subsection{}
Прокатная цена $I = 450$, ставка  $i = 7\%$,  $P_{k} = ?$
\[
	P_{k} = PV = \frac{450}{(1.07)} + \frac{450}{1.07^2} + \dots + \frac{450}{(1.07)^{n}}, n\to \infty
	.\]
\[
	P_{k} =\lim_{n \to \infty} \frac{I  (1 + i)^{n} - 1}{i(1 + i)^n} = \frac{I}{i} = 450 * \frac{100}{7} = 6428,57
	.\]
Формула вечной ренты
\section{Теория производства}
\begin{definition}
	Производство -- процесс преобразования одних благ и услуг в другие. (факторов производства в продукцию)
\end{definition}
\subsection{Факторы производства}
\begin{enumerate}
	\item Труд (L)
	\item Капитал (K)
\end{enumerate}
\begin{definition}
	Производственная функция -- зависимость между количеством используемых факторов производства и максимально возможным выпуском продукции.
	\[
		Q = f(L,K)
		.\]
	\[
		Q = Q(L,K)
		.\]
\end{definition}
С увеличением фактора выпуск растет в замедленном  темпе.
\subsection{Производственная функция короткого периода}
\[
	Q = Q(L)
	.\]
\begin{tabular}{c|c|c|c}
	L & Q  & AP & MP \\
	\hline
	1 & 30 & 10 &    \\
	\hline
	2 & 30 & 15 & 20 \\
	\dots            \\
\end{tabular}\\
$AP = \frac{Q}{L}$ average product \\
$MP = \frac{dQ}{dL}$ marginal product\\
$Q = Al + bL^2 - c L^3$
\begin{definition}
	Эластичность выпуска по фактору -- на сколько процентов возрастает выпуск при увлеечении фактора на 1 \%
\end{definition}
\[
	\epsilon_{Q,L} =  \frac{dQ}{Q} / \frac{dL}{L} = \frac{d Q}{d L} * \frac{L}{Q} = \frac{MP_{L}}{AP_{l}}
	.\]
\subsection{Производственная функция длительного периода}
\[
	Q = Q(L,K)
	.\]
\begin{tabular}{|c|c|c|c|c|}
	K,L & 10 & 10 \\
	\hline
	50  & 33 & 40
\end{tabular}
\[
	Q = L^{0.75} K^{0,25}
	.\]
Формула Кобба-Дугласа.\\
Изовкванты -- линии равного выпуска
\[
	Q = AL^{\alpha} K^{\beta}
	.\]
\[
	\epsilon_{Q,L} - \frac{MP_{L}}{AP_{L}} = \frac{\alpha AK^{\beta} L^{\alpha -1}}{AK^{\beta} K^{\alpha - 1}} = \alpha
	.\]
\[
	MP_{l} = \alpha K^{\beta} L^{a-1}
	.\]
\[
	MP_{k} = \beta K^{\beta -1} L^{\alpha}
	.\]
\[
	MRTS _{kl} = \frac{MP_{K}}{MO_{l}} = \frac{\beta L}{\alpha K}
	.\]
\[
	MRTS_{LK} =\frac{ \alpha P_{l}}{M P_{k}}
	.\]
\subsection{Задача}
У фирмы есть деньги $C = 80$.  $r=2$ аренда,  $w = 1$ зарплата
\\
$rK + wL = C$ Уравнение изокосты, бюджетное уравнение
\[
	\Phi = wL + rL + \lambda (L^{\alpha} K^{\beta} - Q) \to \min
	.\]
Технология задана, нужно минимизировать
\[
	A = (40,20)
	.\]
В точке A $MRTS_{L,K} = \frac{20}{40} = 0.5$
\begin{definition}[Отдача от масштаба]
	\[
		Q = L^{\alpha} K^{\beta}
		.\]
	\[
		(nL)^{\alpha} (n K)^{\beta} = nQ , \alpha+\beta = 1 ~ \text{постоянная отдача}
		.\]
	\[
		(nL)^{\alpha} (n K)^{\beta} = mQ , m > n,\alpha + \beta >1 ~ \text{растущая отдача}
		.\]
	\[
		(nL)^{\alpha} (n K)^{\beta} = mQ , m < n, \alpha + \beta < 1 ~ \text{снижающася отдача}
		.\]
\end{definition}
\begin{definition}
	Затраты -- ценность израсходованых при производстве блага услуш факторов производства.
\end{definition}
\begin{definition}
	Бухгалтерские затраты -- фактические денежные расходы, плюс амортизационные начисления на капитальное оборудование
\end{definition}
\begin{definition}
	Экономичесие затраты -- фактические денежные расходы + вмененные затраты
\end{definition}
\begin{definition}
	Вмененные затраты -- разность между созданной факторами произовдства ценностью при их наиболее эффективном использовании и фактической их оплатой.
\end{definition}
\subsection{Разновидности затрат}
\begin{enumerate}
	\item Постоянные затраты (FC) -- затраты независящие от объема
	\item Переменные затраты (VC) -- затраты, меняющиеся при изменении объема выпуска
	\item Общие затраты (TC) -- затраты на весь обхем выпуска
	\item Средние совокупные затраты ATC = $\frac{TC}{Q}$
\end{enumerate}
\subsection{Задача}
\[
Q = L^{0,75}K^{0,25}
.\] 
цены факторов $w = 5,r=2$, 625 ед капитала, труда можно сколько хаотят
 \begin{enumerate}
    \item Насколько средние совокупные затраты превышаеют средние переменные при выпуске 40 ед
        \[
        2 * 625 / 40 = 31.25
        .\] 
\end{enumerate}
\subsection{Еще задача}
Прибыль $\pi(Q) = TR - TF = PQ  - TC(Q)$. Надо максимализировать, $P = const$
\begin{enumerate}
    \item 
        \[
        \frac{d \pi}{d Q} = P - \frac{d TC}{d Q} 0 \iff P = M C(Q)
        .\] 
    \item 
        \[
        \frac{d^2 \pi}{d Q^2} = - \frac{d^2 TC}{d Q^2} < 0
        .\] 
\end{enumerate}
\subsection{Задача}
Фирма максимализирует прибыль
\[
Q = L^{0,25}K^{0,25}
.\] 
\[
w = 2
.\] 
\[
r = 8
.\] 
Вывести функцию предложения по цене
\[
\frac{0,25 K}{0,25 L} = \frac{2}{8}
.\] 
\[
K =0,25 L
.\] 
Условие равновесия
\[
    K = \frac{Q^{4}}{L}
.\] 
\[
0,25 L =\frac{Q^{4}}{L}
.\] 
\[
L = 2 Q ^2
.\] 
\[
K =0.5 Q^2
.\] 
\[
    LTC = 2 * 2Q^2 + 8 * 0,5 Q^2 = 8 Q ^2
.\] 
\[
LMC = 16 Q
.\] 
Условие максимализации $P = 16 Q$
\subsection{Разновидности производственных функций}
\subsubsection{Производственная функция Леонтьева}
 \[
     Q = \min{(\frac{L}{a},\frac{K}{b}}
.\] 
$a,b$ необходимый расход на единицу продукции.
\section{74}
\[
Q = a - bP
.\] 
\section*{6}
\[
TU(x) = 24x - x^2
.\] 
\[
TU(y) = 28y - 2y^{3}
.\] 
Индивид потребляет 5 ед  $x$, 2 ед  $y$
Предельная полезность денег $\lambda = 1/3$
Найти цены
Юзаем второй закон госсена
\[
\frac{MU_{x}}{P_{x}} = \frac{MU_{y}}{P_{y}} = \lambda
.\] 
\[
MU_{x} = TU'(x) = 22 -2x
.\] 
\[
MU_{y} = 28 - 6y^{3}
.\] 
\[
MU_{x}(5) = 22 - 10 = 12
.\] 
\[
MU_{2} = 28 - 24 = 4
.\] 
\[
\frac{12}{P_{x}} = \frac{1}{3}
.\] 
\[
P_{x}=  13
.\] 
\[
\frac{4}{P_{y}} = \frac{1}{3}
.\] 
\[
P_{y} = 12
.\] 
\section{39}
\[
U = (x + 5)^{0.5} (y + 9)^{0,25}
.\] 
\[
MRS = \frac{U'_{x}}{U'_{y}} = \frac{0.5 (x + 5)^{-0.5} (y + 9)^{0,25}}{0,25 (x + 5)^{0,5} (y + 9)^{-0,75}} = 2 \frac{y + 9}{x + 5} = \frac{P_{x}}{P_{y}}
.\] 
\[
120 = P_{x} y + P_{y} y
.\] 
\[
2 y Py + 18 Py =  x P_{x} + 5P_{x}
.\] 
\[
x^{d} = \frac{240 - P_{x} + 18P_{y}}{3Px} = \frac{240 - 15 + 18}{9} = 27
.\] 
\section{}
\[
    Q = \sqrt{K * L} 
.\] 
Пусть $Q = 50$,  $w = 10$,  $r = 5$. Нужно понять сколько фирма тратит труда и капитала. постоянная отдача от масшатаба
\[
MRTS = \frac{w}{r} = 2
.\] 
\[
50 = \sqrt{KL} 
.\] 
\[
\frac{2500}{L} = K
.\] 
\[
k
.\] 
\section{Основы теории спроса и предложения}
\subsection{Изменение отраслевого спроса}
Договримся, что функции спроса и предложения -- линейные. Используем метод сравнительной статики, есть $Q_0,Q_1$, $P_0,P_1$ Cравниваем два состояние
\subsection{Изменение отраслевого предложения}
Там смотрим так же
\subsection{Приложение к экономике}
Пример $D_{1900},S_{1900}$ спрос и предложение меди в 1900, $D_{1950},S_{1950}$ спрос и предложение на медь в 1950. Соеденим точки получим линию долговременного изменения цены и потребления. Это реализуемые точки
\subsection{Определение параметро линейных функций спроса и предложения по эластичность}
\[
Q^{D} = a- bP \implies e^{d} = -b \frac{P^{*}}{Q^{*}} \implies b =-e^{D} \frac{Q^{*}}{P^{*}}
.\] 
\[
a = (1 - e^{D})Q^{*}
.\] 
Спрос
% \[
% Q^{s} = m + nP \iff e^{S} = n \frac{P^{*}}{Q^{*}} \iff n = e^{s}\frac{Q^{*}}{P^{*}}
% .\] 
% \[
% m = (1 - e^{S}) Q^{*}
% .\] 
Предложение
\subsection{Задача}
\[
P = 20
.\] 
\[
Q = 6000
.\] 
\[
\varDelta = 20 \%
.\] 
\[
b = 3 * \frac{6000}{20} = 900
.\] 
\[
a = 600 * ( 1 + 3) = 24000
.\] 
\[
n = 2 * (6000 / 20) = 600
.\] 
\[
m = 6000 (1 -2) = -60
.\] 
\[
Q^{D}
.\] 
\[
Q^{S}
.\] 
\subsection{Излишек потребителя, излишек производителя}
Излишек потребителя -- выигрыш от покупки продукты. Излишек производителя -- выигрыш от продажи продукта
\subsection{}
Рассмотрим воздействие на рынок налогов дотаций и фиксированных цен.
\begin{enumerate}
    \item Налоги на продажи, бывает двух типов -- акциз (с каждой штуки) и налог с оборота (в процентах от выручки)
    \item Дотации
\end{enumerate}
\subsection{Краткосрочные введения акциза}
Акциз на сиграеты повысили -- продавец перекладывает на покупателя. Цена увеличилась -- продажи уменьшились.
Потребитель проиграл, производитель проиграл, государство выиграло.
общество потеряло
Если нарисовать все понятнее\\
Чем менее эластичен спрос, тем больше налоговое бремя падает на покупателей и наоборот.
\subsection{Дотации}
Пусть государство хочет, чтоб люди больше пили молока. Поэтому субсидируют продавцов, 
за кажду штуку даем $h$ денег, чтоб продавец продавал дешевле. Продукция стала дешевле, излишек покупателя вырос, фирма выиграла, государство потеряло, общество потерялою
\\
Потери общества возникают из-за того, что становится выгодным производитьб данную продукцию и не выгодно похожую.
\section{Изменение в излишке потребителя и производителя при директивных ценах}
Правительство ввело потолок цен, производитель проиграл, ищлишек покупателя вырастает в предельном случае, но однозначно нельзя определить
\section{Динамика}
\subsection{Модель Яна Тинбергена}
Функция спроса $Q_{t}^{D} = 20 - P_{t}$ \\

Функция предложения $Q_{t}S = -10 + 0.5P_{t-1}$ 
\[
2- P_{t} = -10 + 0.5P_{t-1}
.\] 
Условие равновесия выполняется при $P_{t} = P_{t-1} = 140$ 
Увеличили спрос до $Q^{D}_{t} = 230 - P_{t}$
\subsection{Паутинообразная модель ценообразования}
\[
Q_{t}^{D} = a - b P_{t}
.\] 
\[
Q^{S}_{t} = m + n P_{t - 1}
.\] 
\[
P_{t}= \frac{a - m}{b} - \frac{n}{b} P_{t - 1}
.\] 
\[
P_1 = \alpha + \beta P_0
.\] 
\[
P_{t} = \alpha(\sum_{n=0}^{t_1} \beta^{n}) + \beta^{t} P_0
.\] 
\[
P_{t} = \frac{\alpha}{1 -\beta } + (P_0  - \frac{\alpha}{1 - \beta}) \beta^{t}
.\] 
\[
|\beta| > 1, P_{t} \to \infty
.\] 
\[
|\beta| < 1, P_{t} \to \frac{\alpha}{1-  \beta} \iff |b| > n
.\] 
\subsection{Ожидания}
\begin{enumerate}
    \item Cтатическое ожидание $P_{t}^{e} = P_{t -1}$
     \item Адаптивные ожидания $P_{t}^{e} = P^{e}_{t-1} + 0,25(P_{t-1} - P^{e}_{t-1}$
\end{enumerate}
\subsection{Рост спроса}
\[
Q_{0}^{D} = 9 - P_0
.\] 
\[
P_0^{e} = 2,5
.\] 
\section{Импортные пошлины}
В открытой экономике внутренняя цена равна мировой цене $P_{w}$
\section{Задача на затраты}
Дана функция затрат
\[
TC = 16 + 4Q + Q^2
.\] 
\[
P = 20
.\] 
\begin{enumerate}
    \item 
\[
ATC \to \min
.\] 
\item
\[
 \pi \to \max
.\] 
прибыль\\
Найти излишек производителя
\[
RS = ?
.\] 
\end{enumerate}
\[
FC = 16
.\] 
\[
VC = 4Q + Q^2
.\] 
\[
ATC = \frac{TC}{Q} = \frac{16}{Q} +  4 +  Q
.\] 
\[
ATC' =  -\frac{16}{Q^2} + 1 = 0
.\] 
\[
Q = 4
.\] 
\[
ATC(4) = 12
.\] 
\[
\pi =  PQ - TC(Q) = = -Q^2 +16Q - 16 
.\] 
\[
\pi' = -2Q + 16
.\] 
\[
Q = 8
.\] 
\[
\pi(8) = 48
.\] 
\[
ATC(8) = 2  + 4 + 8 = 14
.\] 
\[
MC = TC' = 4 + 2Q = P
.\] 
\[
Q^{S} = \frac{P}{s} - 2
.\] 
\section{}
Пусть функция спроса имеет вид $Q^{D} =  12  - P$
 \[
 Q^{S} = -3 + 4P
 .\] 
 \[
     t^{s} (\text{акцизка}) = 2
 .\] 
 Определить $P,Q$ до налога , а так же все параметры рынка
 \[
 Q^{D} = Q^{S}
 .\] 
 \[
 P_0^{*} = 3
 .\] 
 \[
 Q_0^{*} = 9
 .\] 
 \[
     P^{+} (\text{Брутто}) = P^{-} +t = P^{+} + t
 .\] 
 \[
 Q_1^{s} = -3 + (P^{+} - 2)
 .\] 
 \section{Факторы производства в экономике}
 идея в том, что на рынке сталкиваются продаватели и продавцы. Покупатели свою полезность максимализируют, продавцы свою.
 \[
 \max U(F_{j}) \to F_{j}^{S}
 .\] 
 \[
 \max  \pi(F_{j}) \to F_{j}^{D}
 .\] 
 Есть две цены прокатная и капитальная.
 \begin{definition}[Прокатная цена]
     $r_{j}$ -- плата за использование в единиицу времени.
 \end{definition}
 \begin{definition}[Капитальная цена]
     $PV_{j}$ -- плата за приобретение фактора\\
     PV - Present value
 \end{definition}
 При использовании фактора производства в течение единицы времени нужно сравнить
 \[
 w \iff P * MP_{L}
 .\] 
 сколько в деньгих приносит предельного продукта работник\\
 При покупке нужно сравнивать
 \[
 k \iff \sum \pi
 .\] 
 \[
 k \iff \sum_{k=1}^{T} \frac{\pi_{k}}{(1 + i)^{k}} \equiv PV
 .\] 
 Мы привели затраты сделаныые в разное временя к текущему. Это процедура дисконтирования
 \begin{definition}[Net present value]
     \[
     NPV = PV - I_0
     .\] 
 \end{definition}
 \[
 PV = \frac{I}{1 + i}
 .\] 
 \[
     FV(\text{Future Value}) = I(1 + i)
 .\] 
 \subsection{Пример}
 Cегодняшнаяя ценность 1000 рублей полученных через год при ставку 10 процентов равна
 \[
 PV = \frac{1000}{1 + i} = 909.09
 .\] 
 \[
 FV = 909.09(1  + i) = 1000
 .\] 
 Деньги сегодня и деньги завтра разные вещи.\\
 Если срок получения дохода $n$ лет, то  $PV - \frac{I}{(1 + i)^{n}}$ \\
 $(1 + i)$ коэффициент наращения,  $\frac{1}{1 + i}$ коэфициент дисконтирования.
\subsection{}
Если все $\pi$ одинаковые , то  $PV$ геометрическая прогрессия
 \[
a_1 = \frac{a}{1 + i}
.\] 
\[
a_{n} = \frac{a}{(1 + i)^{T}}, q = \frac{1}{1 + i}
.\] 
\begin{definition}
    Аннуитет -- поток постоянных доходов, поступающих через одинаковое промежутки времени\\ 
    PV аннуитета
    \[
    PV(a) = (\frac{a}{i + 1} - \frac{a}{(1 + i)^{T + 1}}) / (1 - \frac{1}{1 + i}) =
    a \frac{(1 + i)^{T} - 1}{i(1 + i)^{T}}
    .\] 
\end{definition}
Чтобы поток будущих доходов предствить в виде аннуитета нужно
\begin{enumerate}
    \item Определить $PV$
    \item Умножить  $PV$ на аннуитетный коэффициент
\end{enumerate}
\subsection{Пример}
4 года тратим по 120 на строительство шахтыю 10 лет добываем 100 тон угля, при затратах 80.После закртия 50 тратим. $i = 10$
\[
FV_{4}
.\] 
\section{Монополия}
\subsection{Цена на моноплизированном рынке}
\begin{definition}[Монополия]
    \begin{enumerate}
        \item Один продавец на закрытом рынке
        \item Рынок гомогенного блага
        \item Полное и симметричное распределения информации
    \end{enumerate}
\end{definition}
\begin{definition}[Факторы закрытия рынка]
    \begin{enumerate}
        \item Исключительное право
        \item Сговор
        \item Большие невозвратные капитальные затраты
    \end{enumerate}
\end{definition}
\subsection{Модель}
\[
    \pi(Q) = P(Q)Q - TC(Q)
.\]
\[
\pi'  = p + P' Q - TC' = 0
.\] 
\[
    P + P' Q \text{Предельная выручка}
.\] 
\[
\pi'' = TR'' - TC'' < 0
.\] 
В точке пересечения кривых $MR$,  $MC$ предельная выручка должна снижаться быстрее предельных затрат.
\subsection{Предельная выручка}
\[
Q^{D} = a - bP
.\] 
\[
P = \frac{a}{b} - \frac{Q}{b}
.\] 
\[
P = g - h Q
.\] 
\[
Tr = PQ = (g - HQ)Q = gQ -HQ^2
.\] 
\[
MR = g - 2hQ
.\] 
\section{Максимализация прибыли}
\[
Q = 40 - P
.\] 
\[
TR = 40Q - Q^2
.\] 
\[
TC = 50 + Q^2
.\] 
\[
MC = 2Q
.\] 
\[
MR = MC
.\] 
\[
40 - 2q = 2q
.\] 
\[
q = 10, p =30
.\] 
\section{Чистые потери от монопольной власти}
\[
P_{m}, Q_{m}
.\] 
Ущерь от монополизации рынка двоякий, изъятие части излишка покупателей, сокращения объема продаж
\section{Ценовая дискриминация}
\begin{definition}
    Ценовая дискриминация -- продажа однородной прудкции в одно и тоже время по разным ценам, различия в ценах не связаны с затратами на производство и доставку товара на рынок
\end{definition}
Основание -  различия в эластичности спроса отдельного потребителя при различных объемах спроса.\\
Основное условие осуществления -- невозможность перепродажи блага.
\subsection{Ценовая дисркиминация 3-й степени}
\[
\pi = \sum Pq  - TC(Q)
.\] 
\[
Q = \sum q
.\] 
\[
\pi'_{q_1} = P  + P'_{q_1} q_1 - TC'_{q_1} = 0
.\] 
\[
\pi'_{q_{n}} = P + P'_{q_{n}} q_{n} - TC(Q)'_{q_{n}} = 0
.\] 
\[
TC'_{q_1} = Tc'_{q_2} = \dots =  MC(Q)
.\] 
\[
MR_1(q_1) = MR_2(q_2) = \dots = MC(q)
.\] 
\section{Рынки несовершенной конкуренции}
\subsection{Монополичстическая конкуренция}
\begin{enumerate}
    \item Много продавцов много покупателей
    \item Свободный вход и выход
    \item Симметричное распределние информации
    \item Рынок гетерогенного блага
\end{enumerate}
Функция спроса на продукцию моноплистического конкурента
\[
q_{i}^{D} = \alpha - \beta P_{i} + \gamma ( \frac{\sum_{j=1,j\neq i}^{n} P_{j}}{n - 1} - P_{i})
.\] 
Если
\[
    P_1 = p_2 = \dots = P_{n}
.\] 
то 
\[
q_{i}^{D} = \alpha - \beta P
.\] 
\end{document}

