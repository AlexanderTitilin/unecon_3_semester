\documentclass[14pt]{extarticle}
\usepackage{fontspec}
\usepackage[russian, english]{babel}
\setmainfont{Times New Roman}
\usepackage{amssymb}
\usepackage{setspace}
\onehalfspacing
\usepackage{amsmath}
\usepackage{listings}
\usepackage{indentfirst}
\setlength{\parindent}{1.25cm}
\usepackage[right=10mm,left=30mm,top=20mm,bottom=20mm]{geometry}
\newtheorem{theorem}{Теорема}
\newtheorem{corollary}{Следствие}[theorem]
\newtheorem{lemma}[theorem]{Лемма}
\newtheorem{definition}{Определение}
\title{Экономика}
\author{}
\date{}
\begin{document}
    \maketitle
    \section{05.09}
    \begin{definition}
        Микроэкономика изучает поведение отдельных экономических субъектов.
        В центре внимания -- цены и объемы.
    \end{definition}
    \begin{definition}
        Макроэкономика изучает функционирование экономической системы в целом и крупных ее секторов.
    \end{definition}
    \subsection{Ограниченность ресурсов}
    Задача состоит в том, что экономические агенты являются рациональными и направлены на максимализирующую деятельность. Наша задача посмотреть как это моделируется.
    \subsection{Задача}
    Есть чел, его задача описывается функцией полезность
    \[
    U = (Q  - 5)^{0.6} (20 - L)^{0.3}
    .\] 
    Технология производства
    \[
    Q = 10 L^{0.75}
    .\] 
    Надо максимализировать
    \[
    \Phi = (Q - 5)^{0.6} (20 - L)^{0,3} - \lambda (Q - 10 L^{0,75})
    .\] 
    Дрочь про фнп, на матане не прошли еще
    \subsection{Вопросы экономистов}
    \begin{enumerate}
        \item что производить
        \item для кого производить
        \item как производить
        \item когда производить
    \end{enumerate}
    Решая задачи, экономисты делают разные модели.
    \subsection{Рынок}
    \begin{definition}
        Рынок -- общественный механизм распределния благ, посредством добровольного обмена.
    \end{definition}
    Использование денег упрощает обмен.
    \subsection{Создание экономической теории}
    \begin{enumerate}
        \item Наблюдение экономической деятельности.
        \item Введение понятий, выдвижение гипотез.
        \item Создание научной концепции
        \item Логическая и практическая проверка концепции
        \item Прошла -- удовретворительная теория, иначе уточнить наблюдении, гипотез или концепции.
    \end{enumerate}
    \subsection{Классификация экономических моделей}
    \begin{enumerate}
        \item Частичная или общая
        \item Статическая или динамическая
        \item Оптимизационная или равновесная
        \item Детерминированная или стохастическая (влияет не влияет вероятность)
    \end{enumerate}
    % \subsection{Проникновение математики в экономику}
    \subsection{Теория поведения потребителя}
    Полезность, которую получает индивид, можно измерить.
    Таблица Менгера.
\end{document}

