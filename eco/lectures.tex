\documentclass[14pt]{extarticle}
\usepackage{fontspec}
\usepackage[russian, english]{babel}
\setmainfont{Times New Roman}
\usepackage{amssymb}
\usepackage{setspace}
\onehalfspacing
\usepackage{amsmath}
\usepackage{listings}
\usepackage{indentfirst}
\setlength{\parindent}{1.25cm}
\usepackage[right=10mm,left=30mm,top=20mm,bottom=20mm]{geometry}
\newtheorem{theorem}{Теорема}
\newtheorem{corollary}{Следствие}[theorem]
\newtheorem{lemma}[theorem]{Лемма}
\newtheorem{definition}{Определение}
\title{Экономика}
\author{}
\date{}
\begin{document}
    \maketitle
    \section{05.09}
    \begin{definition}
        Микроэкономика изучает поведение отдельных экономических субъектов.
        В центре внимания -- цены и объемы.
    \end{definition}
    \begin{definition}
        Макроэкономика изучает функционирование экономической системы в целом и крупных ее секторов.
    \end{definition}
    \subsection{Ограниченность ресурсов}
    Задача состоит в том, что экономические агенты являются рациональными и направлены на максимализирующую деятельность. Наша задача посмотреть как это моделируется.
    \subsection{Задача}
    Есть чел, его задача описывается функцией полезность
    \[
    U = (Q  - 5)^{0.6} (20 - L)^{0.3}
    .\] 
    Технология производства
    \[
    Q = 10 L^{0.75}
    .\] 
    Надо максимализировать
    \[
    \Phi = (Q - 5)^{0.6} (20 - L)^{0,3} - \lambda (Q - 10 L^{0,75})
    .\] 
    Дрочь про фнп, на матане не прошли еще
    \subsection{Вопросы экономистов}
    \begin{enumerate}
        \item что производить
        \item для кого производить
        \item как производить
        \item когда производить
    \end{enumerate}
    Решая задачи, экономисты делают разные модели.
    \subsection{Рынок}
    \begin{definition}
        Рынок -- общественный механизм распределния благ, посредством добровольного обмена.
    \end{definition}
    Использование денег упрощает обмен.
    \subsection{Создание экономической теории}
    \begin{enumerate}
        \item Наблюдение экономической деятельности.
        \item Введение понятий, выдвижение гипотез.
        \item Создание научной концепции
        \item Логическая и практическая проверка концепции
        \item Прошла -- удовретворительная теория, иначе уточнить наблюдении, гипотез или концепции.
    \end{enumerate}
    \subsection{Классификация экономических моделей}
    \begin{enumerate}
        \item Частичная или общая
        \item Статическая или динамическая
        \item Оптимизационная или равновесная
        \item Детерминированная или стохастическая (влияет не влияет вероятность)
    \end{enumerate}
    % \subsection{Проникновение математики в экономику}
    \subsection{Теория поведения потребителя}
    Полезность, которую получает индивид, можно измерить.
    Таблица Менгера.
\section{12.09}
\subsection{Благо антиблаго}
Есть функция полезности, зависит от количеста. Если растет благо, иначе антиблаго.
\subsection{Общая полезность}
полезность от потребления всего набора плаг
\subsection{Предельная полезность}
Прирост полезности при потребление
\subsection{Гипотеза убывания предельной нормы замещения. Первый закон Госсена}
В каждом акте потребления предельная полезность убывает.
\subsection{Второй закон Госсена}
В общем виде максимализация полезности потрибителя имеет следущий вид
\[
U = U(Q_1,Q_2,\dots,Q_{n}) \to \max
.\] 
\[
    \text{пред} M = \sum_{i=1}^{n} P_1 Q_{n}
.\] 
Математически это сводится к задаче лагрнанжа
\[
L = U(Q_1,Q_2,\dots,Q_{n}) - \lambda(\sum P_{i} Q_{i} - M)
.\] 
Взяли все частные производные
\[
\frac{M U_1}{P_1} = \frac{M U_2}{P_{2}} = \dots = \frac{M U_{n}}{P_{n}}  = \lambda
.\] 
$\lambda$ предельная полезность денег. 
\subsection{Простейшая функция спроса}
Используя таблицу Менгела или задачу Лагранжа можно вывести функцию индвидуального спроса на товар

Пусть, у нас 2 благаю Потребности описываются функцией вида
 \[
U = (x_1,x_2) = x_1^{a} x_2^{b}
.\] 
\[
M U_1 = \frac{\partial U}{\partial x_1} = ax_1^{a -1} x_2^{b}
.\] 
\[
M U_2 = \frac{\partial U}{\partial x_2} bx_1^{a} x_2^{b_{01}}
.\] 
\[
    \frac{ax_1^{a-1} x_2^{b}}{bx_1^{a} x_2^{b -1}} = \frac{ax_2}{bx_1}
.\] 
Из второго закона Госсена
\[
x_2 = \frac{b p_1}{a p_1} x_1
.\] 
\[
x_1 = \frac{a m}{(a + b) p_1}
.\] 
\[
p_1 x_1 + p_2 x_2 = m 
.\] 
\subsection{Еще функция полезности. Функция стоуна}
\[
U = (Q_{a} - k)^{\alpha} (Q_{b} - l)^{\beta} (Q_{c} - m)^{\gamma}
.\] 
\[
M = P_{a} Q_{a} + P_{b} Q_{b} + P_{c} Q_{c}
.\] 
\[
 \Phi = U - ()
.\] 
\subsection{Квазилинейная функция}
\[
U = Q_{F} + \sqrt{Q_{G}} 
.\] 
\[
\Phi = Q_{F} + \sqrt{Q_{G}} - \lambda (P_{F} Q_{F} + P_{G} Q_{G}  - M)
.\] 
\subsection{Кривые безразличия}
\begin{enumerate}
    \item Кривая безразличия содержит все одинаково предпочтительные наборы благ
    \item один и тот же уровень полезноcти
\end{enumerate}
\subsection{Потребительский выбор}
\[
P_{c} = 2, P_{F} = 1, M = 80
.\] 
\subsection{Еще задача}
Дано
\[
M = 60 , P_{c} = 2, P_{f}
.\] 
\[
U = C^{0.5}F^{0.25}
.\] 
\[
\Phi = C^{0,5} F^{0,25} - \lambda(2C +F - 60)
.\] 
\[
    \begin{cases}
        \frac{0.5 F ^{0,25}}{C^{0,5}} = 2\lambda\\
        \frac{0,25 C^{0,5}}{F^{0,75}}= \lambda
    \end{cases}
.\] 
\[
    0.125 F^{-0.5} = 2 \lambda^2
.\] 
\subsection{Производная или эластичность}
Мы исследуем спрос на картошку. Приувелечение цены на 10 копеек на кг, объем спроса снижается на 10 кг в год
\[
    \frac{dQ}{dP} = - \frac{10}{0,1} = -100 \frac{\text{кг}^2}{\text{руб}*\text{год}}
.\] 
Производная зависит от единиц измерения благ. Заменим картошку на водку и не сможем сравнить.
Для избеганий такой проблемы юзают понание эластичсности
\[
    e  = \frac{d Q}{d P} * \frac{P}{Q}
.\] 
измеряется проблема
\subsection{Сорта эластичности}
\begin{enumerate}
    \item Точечная или дуговая (если функция непрерывная или дискретная)
    \item Пряма -- по цене данного блага, перекрестная -- по цене другого блага.
    \item По доходу
\end{enumerate}
\subsection{Перекрестная эластичность}
\[
e_{i,j}^{D} = \frac{\varDelta Q_{i}}{\varDelta P_{j}} * \frac{P_{j}}{Q_{i}}
.\] 
\begin{enumerate}
    \item $e_{ij}^{D} > 0$  взаимозаменяемые блага
    \item $e_{ij} < 0$ взаимодополняемые блага
\end{enumerate}
\end{document}
