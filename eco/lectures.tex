\documentclass[14pt]{extarticle}
\usepackage{fontspec}
\usepackage[russian, english]{babel}
\setmainfont{Times New Roman}
\usepackage{amssymb}
\usepackage{setspace}
\usepackage{xcolor}
\usepackage{graphicx}
\usepackage{pgfplots}
\onehalfspacing
\usepackage{amsmath}
\usepackage{listings}
\usepackage{indentfirst}
\setlength{\parindent}{1.25cm}
\usepackage[right=10mm,left=30mm,top=20mm,bottom=20mm]{geometry}
\newtheorem{theorem}{Теорема}
\newtheorem{corollary}{Следствие}[theorem]
\newtheorem{lemma}[theorem]{Лемма}
\newtheorem{definition}{Определение}
\title{Экономика}
\author{}
\date{}
\begin{document}
\maketitle
\section{05.09}
\begin{definition}
	Микроэкономика изучает поведение отдельных экономических субъектов.
	В центре внимания -- цены и объемы.
\end{definition}
\begin{definition}
	Макроэкономика изучает функционирование экономической системы в целом и крупных ее секторов.
\end{definition}
\subsection{Ограниченность ресурсов}
Задача состоит в том, что экономические агенты являются рациональными и направлены на максимализирующую деятельность. Наша задача посмотреть как это моделируется.
\subsection{Задача}
Есть чел, его задача описывается функцией полезность
\[
	U = (Q  - 5)^{0.6} (20 - L)^{0.3}
	.\]
Технология производства
\[
	Q = 10 L^{0.75}
	.\]
Надо максимализировать
\[
	\Phi = (Q - 5)^{0.6} (20 - L)^{0,3} - \lambda (Q - 10 L^{0,75})
	.\]
Дрочь про фнп, на матане не прошли еще
\subsection{Вопросы экономистов}
\begin{enumerate}
	\item что производить
	\item для кого производить
	\item как производить
	\item когда производить
\end{enumerate}
Решая задачи, экономисты делают разные модели.
\subsection{Рынок}
\begin{definition}
	Рынок -- общественный механизм распределния благ, посредством добровольного обмена.
\end{definition}
Использование денег упрощает обмен.
\subsection{Создание экономической теории}
\begin{enumerate}
	\item Наблюдение экономической деятельности.
	\item Введение понятий, выдвижение гипотез.
	\item Создание научной концепции
	\item Логическая и практическая проверка концепции
	\item Прошла -- удовретворительная теория, иначе уточнить наблюдении, гипотез или концепции.
\end{enumerate}
\subsection{Классификация экономических моделей}
\begin{enumerate}
	\item Частичная или общая
	\item Статическая или динамическая
	\item Оптимизационная или равновесная
	\item Детерминированная или стохастическая (влияет не влияет вероятность)
\end{enumerate}
% \subsection{Проникновение математики в экономику}
\subsection{Теория поведения потребителя}
Полезность, которую получает индивид, можно измерить.
Таблица Менгера.
\section{12.09}
\subsection{Благо антиблаго}
Есть функция полезности, зависит от количеста. Если растет благо, иначе антиблаго.
\subsection{Общая полезность}
полезность от потребления всего набора плаг
\subsection{Предельная полезность}
Прирост полезности при потребление
\subsection{Гипотеза убывания предельной нормы замещения. Первый закон Госсена}
В каждом акте потребления предельная полезность убывает.
\subsection{Второй закон Госсена}
В общем виде максимализация полезности потрибителя имеет следущий вид
\[
	U = U(Q_1,Q_2,\dots,Q_{n}) \to \max
	.\]
\[
	\text{пред} M = \sum_{i=1}^{n} P_1 Q_{n}
	.\]
Математически это сводится к задаче лагрнанжа
\[
	L = U(Q_1,Q_2,\dots,Q_{n}) - \lambda(\sum P_{i} Q_{i} - M)
	.\]
Взяли все частные производные
\[
	\frac{M U_1}{P_1} = \frac{M U_2}{P_{2}} = \dots = \frac{M U_{n}}{P_{n}}  = \lambda
	.\]
$\lambda$ предельная полезность денег.
\subsection{Простейшая функция спроса}
Используя таблицу Менгела или задачу Лагранжа можно вывести функцию индвидуального спроса на товар

Пусть, у нас 2 благаю Потребности описываются функцией вида
\[
	U = (x_1,x_2) = x_1^{a} x_2^{b}
	.\]
\[
	M U_1 = \frac{\partial U}{\partial x_1} = ax_1^{a -1} x_2^{b}
	.\]
\[
	M U_2 = \frac{\partial U}{\partial x_2} bx_1^{a} x_2^{b_{01}}
	.\]
\[
	\frac{ax_1^{a-1} x_2^{b}}{bx_1^{a} x_2^{b -1}} = \frac{ax_2}{bx_1}
	.\]
Из второго закона Госсена
\[
	x_2 = \frac{b p_1}{a p_1} x_1
	.\]
\[
	x_1 = \frac{a m}{(a + b) p_1}
	.\]
\[
	p_1 x_1 + p_2 x_2 = m
	.\]
\subsection{Еще функция полезности. Функция стоуна}
\[
	U = (Q_{a} - k)^{\alpha} (Q_{b} - l)^{\beta} (Q_{c} - m)^{\gamma}
	.\]
\[
	M = P_{a} Q_{a} + P_{b} Q_{b} + P_{c} Q_{c}
	.\]
\[
	\Phi = U - ()
	.\]
\subsection{Квазилинейная функция}
\[
	U = Q_{F} + \sqrt{Q_{G}}
	.\]
\[
	\Phi = Q_{F} + \sqrt{Q_{G}} - \lambda (P_{F} Q_{F} + P_{G} Q_{G}  - M)
	.\]
\subsection{Кривые безразличия}
\begin{enumerate}
	\item Кривая безразличия содержит все одинаково предпочтительные наборы благ
	\item один и тот же уровень полезноcти
\end{enumerate}
\subsection{Потребительский выбор}
\[
	P_{c} = 2, P_{F} = 1, M = 80
	.\]
\subsection{Еще задача}
Дано
\[
	M = 60 , P_{c} = 2, P_{f}
	.\]
\[
	U = C^{0.5}F^{0.25}
	.\]
\[
	\Phi = C^{0,5} F^{0,25} - \lambda(2C +F - 60)
	.\]
\[
	\begin{cases}
		\frac{0.5 F ^{0,25}}{C^{0,5}} = 2\lambda \\
		\frac{0,25 C^{0,5}}{F^{0,75}}= \lambda
	\end{cases}
	.\]
\[
	0.125 F^{-0.5} = 2 \lambda^2
	.\]
\subsection{Производная или эластичность}
Мы исследуем спрос на картошку. Приувелечение цены на 10 копеек на кг, объем спроса снижается на 10 кг в год
\[
	\frac{dQ}{dP} = - \frac{10}{0,1} = -100 \frac{\text{кг}^2}{\text{руб}*\text{год}}
	.\]
Производная зависит от единиц измерения благ. Заменим картошку на водку и не сможем сравнить.
Для избеганий такой проблемы юзают понание эластичсности
\[
	e  = \frac{d Q}{d P} * \frac{P}{Q}
	.\]
измеряется проблема
\subsection{Сорта эластичности}
\begin{enumerate}
	\item Точечная или дуговая (если функция непрерывная или дискретная)
	\item Пряма -- по цене данного блага, перекрестная -- по цене другого блага.
	\item По доходу
\end{enumerate}
\subsection{Перекрестная эластичность}
\[
	e_{i,j}^{D} = \frac{\varDelta Q_{i}}{\varDelta P_{j}} * \frac{P_{j}}{Q_{i}}
	.\]
\begin{enumerate}
	\item $e_{ij}^{D} > 0$  взаимозаменяемые блага
	\item $e_{ij} < 0$ взаимодополняемые блага
\end{enumerate}
\section{Задачи}
\subsection{}
\[
	Q_{x} = A - BP
	.\]
P - цена\\
\[
	Q_{x} = 40,  P = 10
	.\]
\[
	\epsilon = -2
	.\]
Надо расчитать излишек потребителя
\[
	\epsilon = \frac{d Q}{d P} * \frac{P}{Q} = -2
	.\]
\[
	-B  * \frac{10}{40} =  B * \frac{1}{4} = 2
	.\]
\[
	B = 8
	.\]
\[
	40 = A - 80
	.\]
\[
	A = 120
	.\]
\[
	Q_{x} =120 -8P
	.\]
\[
	P = \frac{Q_{x} - 120}{-8} = - \frac{Q_{x}}{8} + 15
	.\]
\[
	Q_1 = 1, P_1 = 14,875
	.\]
\[
	Q_2 = 2 , P_2 = 14,75\dots
	.\]
\[
	Q_{40} = 40, P_{40} = 10
	.\]
\[
	RD = 100 - \text{Выгода}
	.\]
\subsection{Задача про ценность денег}
\[
	U(c_1,c_2)
	.\]
\[
	A = (c_1,c_2)
	.\]
\[
	B = (500,500)
	.\]
\[
	\varDelta = 500 - c_1 \text{-- сбережения}
	.\]
\[
	c_2 = I_2 + (I_1 - c_1) (1 + i)
	.\]
i -- ставка в процентах
\[
	PV = 1 + i
	.\]
\[
	FV =  I_1 (1 + i)
	.\]
\begin{definition}[Коэфициент Дисконтирования]
	\[
		\frac{1}{1 + i}
		.\]
\end{definition}
\begin{definition}[коэфициент наращения]
	\[
		1 + i
		.\]
\end{definition}
\[
	FV = 500 * 2 = 1000
	.\]
\[
	PV = \frac{500}{2} = 250
	.\]
\subsection{}
Прокатная цена $I = 450$, ставка  $i = 7\%$,  $P_{k} = ?$
\[
	P_{k} = PV = \frac{450}{(1.07)} + \frac{450}{1.07^2} + \dots + \frac{450}{(1.07)^{n}}, n\to \infty
	.\]
\[
	P_{k} =\lim_{n \to \infty} \frac{I  (1 + i)^{n} - 1}{i(1 + i)^n} = \frac{I}{i} = 450 * \frac{100}{7} = 6428,57
	.\]
Формула вечной ренты
\section{Теория производства}
\begin{definition}
	Производство -- процесс преобразования одних благ и услуг в другие. (факторов производства в продукцию)
\end{definition}
\subsection{Факторы производства}
\begin{enumerate}
	\item Труд (L)
	\item Капитал (K)
\end{enumerate}
\begin{definition}
	Производственная функция -- зависимость между количеством используемых факторов производства и максимально возможным выпуском продукции.
	\[
		Q = f(L,K)
		.\]
	\[
		Q = Q(L,K)
		.\]
\end{definition}
С увеличением фактора выпуск растет в замедленном  темпе.
\subsection{Производственная функция короткого периода}
\[
	Q = Q(L)
	.\]
\begin{tabular}{c|c|c|c}
	L & Q  & AP & MP \\
	\hline
	1 & 30 & 10 &    \\
	\hline
	2 & 30 & 15 & 20 \\
	\dots            \\
\end{tabular}\\
$AP = \frac{Q}{L}$ average product \\
$MP = \frac{dQ}{dL}$ marginal product\\
$Q = Al + bL^2 - c L^3$
\begin{definition}
	Эластичность выпуска по фактору -- на сколько процентов возрастает выпуск при увлеечении фактора на 1 \%
\end{definition}
\[
	\epsilon_{Q,L} =  \frac{dQ}{Q} / \frac{dL}{L} = \frac{d Q}{d L} * \frac{L}{Q} = \frac{MP_{L}}{AP_{l}}
	.\]
\subsection{Производственная функция длительного периода}
\[
	Q = Q(L,K)
	.\]
\begin{tabular}{|c|c|c|c|c|}
	K,L & 10 & 10 \\
	\hline
	50  & 33 & 40
\end{tabular}
\[
	Q = L^{0.75} K^{0,25}
	.\]
Формула Кобба-Дугласа.\\
Изовкванты -- линии равного выпуска
\[
	Q = AL^{\alpha} K^{\beta}
	.\]
\[
	\epsilon_{Q,L} - \frac{MP_{L}}{AP_{L}} = \frac{\alpha AK^{\beta} L^{\alpha -1}}{AK^{\beta} K^{\alpha - 1}} = \alpha
	.\]
\[
	MP_{l} = \alpha K^{\beta} L^{a-1}
	.\]
\[
	MP_{k} = \beta K^{\beta -1} L^{\alpha}
	.\]
\[
	MRTS _{kl} = \frac{MP_{K}}{MO_{l}} = \frac{\beta L}{\alpha K}
	.\]
\[
	MRTS_{LK} =\frac{ \alpha P_{l}}{M P_{k}}
	.\]
\subsection{Задача}
У фирмы есть деньги $C = 80$.  $r=2$ аренда,  $w = 1$ зарплата
\\
$rK + wL = C$ Уравнение изокосты, бюджетное уравнение
\[
	\Phi = wL + rL + \lambda (L^{\alpha} K^{\beta} - Q) \to \min
	.\]
Технология задана, нужно минимизировать
\[
	A = (40,20)
	.\]
В точке A $MRTS_{L,K} = \frac{20}{40} = 0.5$
\begin{definition}[Отдача от масштаба]
	\[
		Q = L^{\alpha} K^{\beta}
		.\]
	\[
		(nL)^{\alpha} (n K)^{\beta} = nQ , \alpha+\beta = 1 ~ \text{постоянная отдача}
		.\]
	\[
		(nL)^{\alpha} (n K)^{\beta} = mQ , m > n,\alpha + \beta >1 ~ \text{растущая отдача}
		.\]
	\[
		(nL)^{\alpha} (n K)^{\beta} = mQ , m < n, \alpha + \beta < 1 ~ \text{снижающася отдача}
		.\]
\end{definition}
\begin{definition}
	Затраты -- ценность израсходованых при производстве блага услуш факторов производства.
\end{definition}
\begin{definition}
	Бухгалтерские затраты -- фактические денежные расходы, плюс амортизационные начисления на капитальное оборудование
\end{definition}
\begin{definition}
	Экономичесие затраты -- фактические денежные расходы + вмененные затраты
\end{definition}
\begin{definition}
	Вмененные затраты -- разность между созданной факторами произовдства ценностью при их наиболее эффективном использовании и фактической их оплатой.
\end{definition}
\subsection{Разновидности затрат}
\begin{enumerate}
	\item Постоянные затраты (FC) -- затраты независящие от объема
	\item Переменные затраты (VC) -- затраты, меняющиеся при изменении объема выпуска
	\item Общие затраты (TC) -- затраты на весь обхем выпуска
	\item Средние совокупные затраты ATC = $\frac{TC}{Q}$
\end{enumerate}
\subsection{Задача}
\[
Q = L^{0,75}K^{0,25}
.\] 
цены факторов $w = 5,r=2$, 625 ед капитала, труда можно сколько хаотят
 \begin{enumerate}
    \item Насколько средние совокупные затраты превышаеют средние переменные при выпуске 40 ед
        \[
        2 * 625 / 40 = 31.25
        .\] 
\end{enumerate}
\subsection{Еще задача}
Прибыль $\pi(Q) = TR - TF = PQ  - TC(Q)$. Надо максимализировать, $P = const$
\begin{enumerate}
    \item 
        \[
        \frac{d \pi}{d Q} = P - \frac{d TC}{d Q} 0 \iff P = M C(Q)
        .\] 
    \item 
        \[
        \frac{d^2 \pi}{d Q^2} = - \frac{d^2 TC}{d Q^2} < 0
        .\] 
\end{enumerate}
\subsection{Задача}
Фирма максимализирует прибыль
\[
Q = L^{0,25}K^{0,25}
.\] 
\[
w = 2
.\] 
\[
r = 8
.\] 
Вывести функцию предложения по цене
\[
\frac{0,25 K}{0,25 L} = \frac{2}{8}
.\] 
\[
K =0,25 L
.\] 
Условие равновесия
\[
    K = \frac{Q^{4}}{L}
.\] 
\[
0,25 L =\frac{Q^{4}}{L}
.\] 
\[
L = 2 Q ^2
.\] 
\[
K =0.5 Q^2
.\] 
\[
    LTC = 2 * 2Q^2 + 8 * 0,5 Q^2 = 8 Q ^2
.\] 
\[
LMC = 16 Q
.\] 
Условие максимализации $P = 16 Q$
\subsection{Разновидности производственных функций}
\subsubsection{Производственная функция Леонтьева}
 \[
     Q = \min{(\frac{L}{a},\frac{K}{b}}
.\] 
$a,b$ необходимый расход на единицу продукции.
\section{74}
\[
Q = a - bP
.\] 
\section*{6}
\[
TU(x) = 24x - x^2
.\] 
\[
TU(y) = 28y - 2y^{3}
.\] 
Индивид потребляет 5 ед  $x$, 2 ед  $y$
Предельная полезность денег $\lambda = 1/3$
Найти цены
Юзаем второй закон госсена
\[
\frac{MU_{x}}{P_{x}} = \frac{MU_{y}}{P_{y}} = \lambda
.\] 
\[
MU_{x} = TU'(x) = 22 -2x
.\] 
\[
MU_{y} = 28 - 6y^{3}
.\] 
\[
MU_{x}(5) = 22 - 10 = 12
.\] 
\[
MU_{2} = 28 - 24 = 4
.\] 
\[
\frac{12}{P_{x}} = \frac{1}{3}
.\] 
\[
P_{x}=  13
.\] 
\[
\frac{4}{P_{y}} = \frac{1}{3}
.\] 
\[
P_{y} = 12
.\] 
\section{39}
\[
U = (x + 5)^{0.5} (y + 9)^{0,25}
.\] 
\[
MRS = \frac{U'_{x}}{U'_{y}} = \frac{0.5 (x + 5)^{-0.5} (y + 9)^{0,25}}{0,25 (x + 5)^{0,5} (y + 9)^{-0,75}} = 2 \frac{y + 9}{x + 5} = \frac{P_{x}}{P_{y}}
.\] 
\[
120 = P_{x} y + P_{y} y
.\] 
\[
2 y Py + 18 Py =  x P_{x} + 5P_{x}
.\] 
\[
x^{d} = \frac{240 - P_{x} + 18P_{y}}{3Px} = \frac{240 - 15 + 18}{9} = 27
.\] 
\section{}
\[
    Q = \sqrt{K * L} 
.\] 
Пусть $Q = 50$,  $w = 10$,  $r = 5$. Нужно понять сколько фирма тратит труда и капитала. постоянная отдача от масшатаба
\[
MRTS = \frac{w}{r} = 2
.\] 
\[
50 = \sqrt{KL} 
.\] 
\[
\frac{2500}{L} = K
.\] 
\[
k
.\] 
\section{Основы теории спроса и предложения}
\subsection{Изменение отраслевого спроса}
Договримся, что функции спроса и предложения -- линейные. Используем метод сравнительной статики, есть $Q_0,Q_1$, $P_0,P_1$ Cравниваем два состояние
\subsection{Изменение отраслевого предложения}
Там смотрим так же
\subsection{Приложение к экономике}
Пример $D_{1900},S_{1900}$ спрос и предложение меди в 1900, $D_{1950},S_{1950}$ спрос и предложение на медь в 1950. Соеденим точки получим линию долговременного изменения цены и потребления. Это реализуемые точки
\subsection{Определение параметро линейных функций спроса и предложения по эластичность}
\[
Q^{D} = a- bP \implies e^{d} = -b \frac{P^{*}}{Q^{*}} \implies b =-e^{D} \frac{Q^{*}}{P^{*}}
.\] 
\[
a = (1 - e^{D})Q^{*}
.\] 
Спрос
% \[
% Q^{s} = m + nP \iff e^{S} = n \frac{P^{*}}{Q^{*}} \iff n = e^{s}\frac{Q^{*}}{P^{*}}
% .\] 
% \[
% m = (1 - e^{S}) Q^{*}
% .\] 
Предложение
\subsection{Задача}
\[
P = 20
.\] 
\[
Q = 6000
.\] 
\[
\varDelta = 20 \%
.\] 
\[
b = 3 * \frac{6000}{20} = 900
.\] 
\[
a = 600 * ( 1 + 3) = 24000
.\] 
\[
n = 2 * (6000 / 20) = 600
.\] 
\[
m = 6000 (1 -2) = -60
.\] 
\[
Q^{D}
.\] 
\[
Q^{S}
.\] 
\subsection{Излишек потребителя, излишек производителя}
Излишек потребителя -- выигрыш от покупки продукты. Излишек производителя -- выигрыш от продажи продукта
\subsection{}
Рассмотрим воздействие на рынок налогов дотаций и фиксированных цен.
\begin{enumerate}
    \item Налоги на продажи, бывает двух типов -- акциз (с каждой штуки) и налог с оборота (в процентах от выручки)
    \item Дотации
\end{enumerate}
\subsection{Краткосрочные введения акциза}
Акциз на сиграеты повысили -- продавец перекладывает на покупателя. Цена увеличилась -- продажи уменьшились.
Потребитель проиграл, производитель проиграл, государство выиграло.
общество потеряло
Если нарисовать все понятнее\\
Чем менее эластичен спрос, тем больше налоговое бремя падает на покупателей и наоборот.
\subsection{Дотации}
Пусть государство хочет, чтоб люди больше пили молока. Поэтому субсидируют продавцов, 
за кажду штуку даем $h$ денег, чтоб продавец продавал дешевле. Продукция стала дешевле, излишек покупателя вырос, фирма выиграла, государство потеряло, общество потерялою
\\
Потери общества возникают из-за того, что становится выгодным производитьб данную продукцию и не выгодно похожую.
\section{Изменение в излишке потребителя и производителя при директивных ценах}
Правительство ввело потолок цен, производитель проиграл, ищлишек покупателя вырастает в предельном случае, но однозначно нельзя определить
\section{Динамика}
\subsection{Модель Яна Тинбергена}
Функция спроса $Q_{t}^{D} = 20 - P_{t}$ \\

Функция предложения $Q_{t}S = -10 + 0.5P_{t-1}$ 
\[
2- P_{t} = -10 + 0.5P_{t-1}
.\] 
Условие равновесия выполняется при $P_{t} = P_{t-1} = 140$ 
Увеличили спрос до $Q^{D}_{t} = 230 - P_{t}$
\subsection{Паутинообразная модель ценообразования}
\[
Q_{t}^{D} = a - b P_{t}
.\] 
\[
Q^{S}_{t} = m + n P_{t - 1}
.\] 
\[
P_{t}= \frac{a - m}{b} - \frac{n}{b} P_{t - 1}
.\] 
\[
P_1 = \alpha + \beta P_0
.\] 
\[
P_{t} = \alpha(\sum_{n=0}^{t_1} \beta^{n}) + \beta^{t} P_0
.\] 
\[
P_{t} = \frac{\alpha}{1 -\beta } + (P_0  - \frac{\alpha}{1 - \beta}) \beta^{t}
.\] 
\[
|\beta| > 1, P_{t} \to \infty
.\] 
\[
|\beta| < 1, P_{t} \to \frac{\alpha}{1-  \beta} \iff |b| > n
.\] 
\subsection{Ожидания}
\begin{enumerate}
    \item Cтатическое ожидание $P_{t}^{e} = P_{t -1}$
     \item Адаптивные ожидания $P_{t}^{e} = P^{e}_{t-1} + 0,25(P_{t-1} - P^{e}_{t-1}$
\end{enumerate}
\subsection{Рост спроса}
\[
Q_{0}^{D} = 9 - P_0
.\] 
\[
P_0^{e} = 2,5
.\] 
\section{Импортные пошлины}
В открытой экономике внутренняя цена равна мировой цене $P_{w}$
\section{Задача на затраты}
Дана функция затрат
\[
TC = 16 + 4Q + Q^2
.\] 
\[
P = 20
.\] 
\begin{enumerate}
    \item 
\[
ATC \to \min
.\] 
\item
\[
 \pi \to \max
.\] 
прибыль\\
Найти излишек производителя
\[
RS = ?
.\] 
\end{enumerate}
\[
FC = 16
.\] 
\[
VC = 4Q + Q^2
.\] 
\[
ATC = \frac{TC}{Q} = \frac{16}{Q} +  4 +  Q
.\] 
\[
ATC' =  -\frac{16}{Q^2} + 1 = 0
.\] 
\[
Q = 4
.\] 
\[
ATC(4) = 12
.\] 
\[
\pi =  PQ - TC(Q) = = -Q^2 +16Q - 16 
.\] 
\[
\pi' = -2Q + 16
.\] 
\[
Q = 8
.\] 
\[
\pi(8) = 48
.\] 
\[
ATC(8) = 2  + 4 + 8 = 14
.\] 
\[
MC = TC' = 4 + 2Q = P
.\] 
\[
Q^{S} = \frac{P}{s} - 2
.\] 
\section{}
Пусть функция спроса имеет вид $Q^{D} =  12  - P$
 \[
 Q^{S} = -3 + 4P
 .\] 
 \[
     t^{s} (\text{акцизка}) = 2
 .\] 
 Определить $P,Q$ до налога , а так же все параметры рынка
 \[
 Q^{D} = Q^{S}
 .\] 
 \[
 P_0^{*} = 3
 .\] 
 \[
 Q_0^{*} = 9
 .\] 
 \[
     P^{+} (\text{Брутто}) = P^{-} +t = P^{+} + t
 .\] 
 \[
 Q_1^{s} = -3 + (P^{+} - 2)
 .\] 
 \section{Факторы производства в экономике}
 идея в том, что на рынке сталкиваются продаватели и продавцы. Покупатели свою полезность максимализируют, продавцы свою.
 \[
 \max U(F_{j}) \to F_{j}^{S}
 .\] 
 \[
 \max  \pi(F_{j}) \to F_{j}^{D}
 .\] 
 Есть две цены прокатная и капитальная.
 \begin{definition}[Прокатная цена]
     $r_{j}$ -- плата за использование в единиицу времени.
 \end{definition}
 \begin{definition}[Капитальная цена]
     $PV_{j}$ -- плата за приобретение фактора\\
     PV - Present value
 \end{definition}
 При использовании фактора производства в течение единицы времени нужно сравнить
 \[
 w \iff P * MP_{L}
 .\] 
 сколько в деньгих приносит предельного продукта работник\\
 При покупке нужно сравнивать
 \[
 k \iff \sum \pi
 .\] 
 \[
 k \iff \sum_{k=1}^{T} \frac{\pi_{k}}{(1 + i)^{k}} \equiv PV
 .\] 
 Мы привели затраты сделаныые в разное временя к текущему. Это процедура дисконтирования
 \begin{definition}[Net present value]
     \[
     NPV = PV - I_0
     .\] 
 \end{definition}
 \[
 PV = \frac{I}{1 + i}
 .\] 
 \[
     FV(\text{Future Value}) = I(1 + i)
 .\] 
 \subsection{Пример}
 Cегодняшнаяя ценность 1000 рублей полученных через год при ставку 10 процентов равна
 \[
 PV = \frac{1000}{1 + i} = 909.09
 .\] 
 \[
 FV = 909.09(1  + i) = 1000
 .\] 
 Деньги сегодня и деньги завтра разные вещи.\\
 Если срок получения дохода $n$ лет, то  $PV - \frac{I}{(1 + i)^{n}}$ \\
 $(1 + i)$ коэффициент наращения,  $\frac{1}{1 + i}$ коэфициент дисконтирования.
\subsection{}
Если все $\pi$ одинаковые , то  $PV$ геометрическая прогрессия
 \[
a_1 = \frac{a}{1 + i}
.\] 
\[
a_{n} = \frac{a}{(1 + i)^{T}}, q = \frac{1}{1 + i}
.\] 
\begin{definition}
    Аннуитет -- поток постоянных доходов, поступающих через одинаковое промежутки времени\\ 
    PV аннуитета
    \[
    PV(a) = (\frac{a}{i + 1} - \frac{a}{(1 + i)^{T + 1}}) / (1 - \frac{1}{1 + i}) =
    a \frac{(1 + i)^{T} - 1}{i(1 + i)^{T}}
    .\] 
\end{definition}
Чтобы поток будущих доходов предствить в виде аннуитета нужно
\begin{enumerate}
    \item Определить $PV$
    \item Умножить  $PV$ на аннуитетный коэффициент
\end{enumerate}
\subsection{Пример}
4 года тратим по 120 на строительство шахтыю 10 лет добываем 100 тон угля, при затратах 80.После закртия 50 тратим. $i = 10$
\[
FV_{4}
.\] 
\section{Монополия}
\subsection{Цена на моноплизированном рынке}
\begin{definition}[Монополия]
    \begin{enumerate}
        \item Один продавец на закрытом рынке
        \item Рынок гомогенного блага
        \item Полное и симметричное распределения информации
    \end{enumerate}
\end{definition}
\begin{definition}[Факторы закрытия рынка]
    \begin{enumerate}
        \item Исключительное право
        \item Сговор
        \item Большие невозвратные капитальные затраты
    \end{enumerate}
\end{definition}
\subsection{Модель}
\[
    \pi(Q) = P(Q)Q - TC(Q)
.\]
\[
\pi'  = p + P' Q - TC' = 0
.\] 
\[
    P + P' Q \text{Предельная выручка}
.\] 
\[
\pi'' = TR'' - TC'' < 0
.\] 
В точке пересечения кривых $MR$,  $MC$ предельная выручка должна снижаться быстрее предельных затрат.
\subsection{Предельная выручка}
\[
Q^{D} = a - bP
.\] 
\[
P = \frac{a}{b} - \frac{Q}{b}
.\] 
\[
P = g - h Q
.\] 
\[
Tr = PQ = (g - HQ)Q = gQ -HQ^2
.\] 
\[
MR = g - 2hQ
.\] 
\section{Максимализация прибыли}
\[
Q = 40 - P
.\] 
\[
TR = 40Q - Q^2
.\] 
\[
TC = 50 + Q^2
.\] 
\[
MC = 2Q
.\] 
\[
MR = MC
.\] 
\[
40 - 2q = 2q
.\] 
\[
q = 10, p =30
.\] 
\section{Чистые потери от монопольной власти}
\[
P_{m}, Q_{m}
.\] 
Ущерь от монополизации рынка двоякий, изъятие части излишка покупателей, сокращения объема продаж
\section{Ценовая дискриминация}
\begin{definition}
    Ценовая дискриминация -- продажа однородной прудкции в одно и тоже время по разным ценам, различия в ценах не связаны с затратами на производство и доставку товара на рынок
\end{definition}
Основание -  различия в эластичности спроса отдельного потребителя при различных объемах спроса.\\
Основное условие осуществления -- невозможность перепродажи блага.
\subsection{Ценовая дисркиминация 3-й степени}
\[
\pi = \sum Pq  - TC(Q)
.\] 
\[
Q = \sum q
.\] 
\[
\pi'_{q_1} = P  + P'_{q_1} q_1 - TC'_{q_1} = 0
.\] 
\[
\pi'_{q_{n}} = P + P'_{q_{n}} q_{n} - TC(Q)'_{q_{n}} = 0
.\] 
\[
TC'_{q_1} = Tc'_{q_2} = \dots =  MC(Q)
.\] 
\[
MR_1(q_1) = MR_2(q_2) = \dots = MC(q)
.\] 
\section{Рынки несовершенной конкуренции}
\subsection{Монополичстическая конкуренция}
\begin{enumerate}
    \item Много продавцов много покупателей
    \item Свободный вход и выход
    \item Симметричное распределние информации
    \item Рынок гетерогенного блага
\end{enumerate}
Функция спроса на продукцию моноплистического конкурента
\[
q_{i}^{D} = \alpha - \beta P_{i} + \gamma ( \frac{\sum_{j=1,j\neq i}^{n} P_{j}}{n - 1} - P_{i})
.\] 
Если
\[
    P_1 = p_2 = \dots = P_{n}
.\] 
то 
\[
q_{i}^{D} = \alpha - \beta P
.\] 
\section{Измерение рыночной власти}
\subsection{Типы рынков}
\begin{enumerate}
    \item чистая монополия
    \item доминирование одной фирмы
    \item тесная олигополия
    \item широкая олигополия
    \item монопличстическая конкурения
    \item совершенная конкуренция
\end{enumerate}
\subsection{Показатели концентрациии}
\[
C_{k} = \sum_{i = 1}^{k} Y_{i}
.\] 
Индекс концетрациии, расчитывается для $2,3,8,12$ крупнейших фирми в отрасли
\subsubsection{Недостатки индекса концентрации}
 \begin{enumerate}
    \item Не учитывает всего рынка
    \item Не измеряет дифференциацию в ядре рынка
\end{enumerate}
\[
HHI = \sum Y^2
.\] 
Индекс герфинляоя хиршмана
\begin{enumerate}
    \item $HHI < 1000$ рыное неконцентрированный
    \item  $1000 < HHI < 1800$ рынок умеренно концетрированный, более 1499 доп проверка
    \item  $HHI > 1800$ рынок высококонцетрированный
\end{enumerate}
Модифицированный индекс
\[
HHI = n \tau^2 + 1 / n
.\] 
\[
    \tau = \sum \frac{(Y_{i} - Y_{\text{ср}})}{n}
.\] 
\subsection{Индекс Линда}
Считают по крупнейшим фирмам.
\[
IL_{2} = (Y_1/ Y_2) \times 100\%
.\] 
Для двух фирм
\[
IL_3 = 1/2 (\frac{Y_1}{(Y_2 + Y_3) / 2} + \frac{(Y_1 + Y_2)/2}{Y_3}) 100\%
.\] 
\[
IL_4 = 1/3 (\frac{Y_1}{(Y_2 + Y_3 + Y_4)/3} + \frac{(Y_1 + Y_2)/2}{(Y_3 + Y_4)/2}
+ \frac{(Y_1 + Y_2 + Y_3)/3}{Y_4}) 100\%
.\] 
\subsection{Индекс энтропии}
\[
    E = \sum Y_{i} \ln{\frac{1}{Y_{i}}}
.\] 
\section{Кривая Лоренцы и индекс Джини}
Индекс дини соотношение площали треугольника и площали под кривой лоренца
\section{Общее равновесие}
\[
Q_{A}^{D} = 32 - 3P_{a} + 2P_{B}
.\] 
\[
Q_{B}^{D} = 44 - 2P_{B} + P_{A}
.\] 
\[
Q_{A}^{S} = -10 + 2P_{A} - P_{B}
.\] 
\[
Q_{B}^{S} = -5 + P_{B} - 0.5P_{A}
.\] 
\[
    32 - 3P_{a} + 2P_{B} = -10 + 2P_{B} - P_{B} \iff \textcolor{red}{P_{A} = 8,4 + 0,6 P_{B}}
.\] 
\[
    \textcolor{blue}{ P_{B} = 16,3 + 0,5P_{A}
}.\] 
\[
P_{A} = 26,P_{B} = 29,3
.\] 
\subsection{Модель общего экономического равновесия}
Потребители $i = 1,2,\dots,l$
 \[
U_{i} = U(Q_{i1} , \dots,Q_{in},F_{i}) \to \max
.\] 
F свободное время
\[
M_{i} = wL_{i}^{S} + \sum_{z = 1}^{Z} r_{z} K^{s}
.\] 
Оптимальное поведение потребителя на рынках
\[
Q_{ij} = f(p_1,p_2,\dots,p_{n},w,r_1,r_2,\dots,r_{z})
.\] 
\[
L_{i}^{s} = \phi (p_1,p_2,\dots,p_{n},w,r_1,r_2,\dots,r_{z})
.\] 
Отраслевой спрос на блага
\[
Q_{j}^{D} = \sum_{i =1 }^{l} Q_{ij}
.\] 
\[
L_{j}^{S} = \sum_{i =1}^{L} L_{ij}
.\] 
\textcolor{blue}{Фирмы} продающие j благо $v - 1,2,\dots,V$
\[
\pi_{v,j} = f(Q_{v,j}) \to \max
.\] 
\[
Q_{v,j} = \phi(L_{v,j},K_{v,j})
.\] 
Отраслевое предложение
\[
    Q_{j}^{S} = \sum\limits{v=1}{V} Q^{S}_{vj}
.\] 
\[
    L_{j}^{D} = \sum_{v =1}^{V} L^{D}_{vj}
.\] 
\[
K_{z,j}^{D}= \sum K^{D}
.\] 
\[
Q_{j}^{D} = q_{j}^{S}
.\] 
условие равновесия на рынках благ\\
Условие долгосрочного равновесия
\[
P_{j}Q_{j}^{S} = wL_{j}^{D} + \sum_{z=1}^{Z}r_{z}K_{z,j}^{D}
.\] 
\section{Макроэкономика}
\subsection{Предмет и метод макроэкономического анализа}
\subsubsection{Вопросы макроэкономики}
\begin{enumerate}
    \item Что такое деньги и какова и их роль?
    \item Что такое уровень цен и чем определяется его динамика?
    \item Чем опрееделеяется уровень занятости?
    \item Какие фактора определяют колебания экономической конъюктуры?
    \item Каковы факторы и условия стабильного экономического роста?
    \item Какое воздействие на экономическую конъюктуру страны оказывает остальной мир?
    \item Как государство может способствовать стабильному экономическому росту благосостояния населения?
\end{enumerate}
\subsection{Агрегирование}
Агрегирование: макроэкономические субъекты
\begin{enumerate}
    \item Сектор домашних хозяйств (предложения труда и капитала, спрос на потребительские товары, сбережения)
    \item Предпринимательский сектор (спрос на факторы производства, предложения благ, инвестирование)
    \item Государственных сектор (налоги, расходы на проиводства общественных благ, регулирование денежного побраения, стабилиационная политика)
    \item Сектор <<остальной мир>> (межстранновая торговоя, обмен валют, перелив капитала)
\end{enumerate}
Агрегирование: макроэкономческие блага
\begin{enumerate}
    \item Рынок блага (для отребления  реальных инвестиций, индекс цен)
    \item Рынок труда (один вид, редуциованный)
    \item Рынок финансов (спрос на инвестиции и предложение сбережений)
    \item Рынок денег (количество денег находящихся в обращение и желание хозяйств и фирм иметь в составе воего имущества деньги;ставка процента -- цена денег)
\end{enumerate}
Реальный сектор экономики -- рынки блага и труда\\
Монетарный сектор экономики -- рынки денег и финансов
\subsection{Моделирование}
Модель -- упрощенное представление действительности за счет абстрагирование от несущественных свойств наблюдаемого объекта.\\
Экзогенные данные -- из наблюдений и гипотез\\
Эндогенные переменные -- результат решения модели
\subsection{Задача}
В начале года есть 10 тонн зерна и 25 комбанинов\\
За год произвели 100 тонн зерна и 10 комбайнов\\
Потребили 80 тонн зерна, из которых 10 тонн на семена, и списали 3 комбайна\\
Национальный доход -- 90 тонн зерна и 7 комбайнов.\\
Пусть  1 тонне зерна 3 рубля добавнной ценность, а в 1 комбайне 360 рублей 
\[
90 * 3 + 360 * 7 = 2790
.\] 
Усложнающие факторы при
\begin{enumerate}
    \item Определении натурального содержания
    \item ценностной оценке
    \item взаимоотношении с остальным миром.
\end{enumerate}
\begin{definition}[ВВП]
    Сумма добавленных ценностей в рыночных ценах минус чистые косвенные налоги
\end{definition}
\begin{definition}[ВНП]
    ВВП минус добавленная ценность, созданная иностранными фирмамами, плюс добавленная ценность, созданная отечественными фирмами за границей
\end{definition}
\begin{definition}[НД]
    ВНП минус амортизация (износ) основного капитала
\end{definition}
\subsection{Уровень (масштаб) и индекс цен}
Отностельные цены $\overline{p_1},\overline{p_2},\dots,\overline{p_{n-1}},1$ \\
$P$ денежная оценка  n-го блага \\
Денежные цены : $\overline{p_1}P,\overline{p_2}P,dots \overline{p_{n-1}}P, P$\\
Номинальны НД $\sum_{i =  1}^{n} P \overline{p_{i}} q_{i} = P \sum_{i =1 }^{n} \overline{p_{i}}q_{i} = Py = Y$
\[
    \frac{P_1}{P_0} =  \frac{ P_1 \sum_{i = 1}^{n} \overline{p_{i0}}Q_{i0} Q_{i0}}{P_0 \sum_{i = 1}^{n} \overline{p}_{i0} Q_{i0}} = \frac{\sum_{i = 1}^{n} P_{i 1} Q_{i 0}}{\sum_{i = 1}^{n} P_{i 0} Q_{i 0 }}
.\] 
Индекс Ласпейресса, индексс Пааше, если в последней формуле 0 заменить на 1.
\subsection{Комбайны}
\includegraphics[scale=0.5]{hui.jpg}
\subsection{Система национального счетоводства}
\begin{definition}[Бюджет]
    Доходы и расходы экономического субъекта за определенный период
\end{definition}
\begin{definition}[Народнохозяйственный трудооборот]
    взаимопереплетение бюджетов макроэкономических субъектов
\end{definition}
\subsubsection{Народнохозяйственный кругооборот}
Счет бюджета семьи\\
\begin{tabular}{|c|c|c|c|}
    Доходы & Деньги & Расходы& Деньги\\
    \hline
    Зарплата & 45 & Товары и услуги & 18\\
    \hline
    Дивиденты & 10 & Налоги & 18\\
    \hline
    Пособие & 5 & Акции & 14\\
    \hline 
    Подарок & 5 & & \\
    \hline
\end{tabular}\\
Сальда $3$\\
Бюджетное уравнние $45 + 10 + 5 + 15 = 40 = 18 + 14 + 3$\\
В виде таблицы можно представлени от и постуления к
\subsection{Условие равновесия народнохозяйственного кругооборота}
\begin{enumerate}
    \item $G$ госзакупки
    \item  $I$ инвестиции
    \item  $E$ экспорт
    \item  $T$ налоги
    \item $S$ сбережение
    \item $Z$ оплата импорта
\end{enumerate}
\[
S + T +Z = I + G + E
.\] 
Условие равновесия
\begin{definition}[СНС]
    Набор счетов и таблиц, используемых для определниия показателей.
\end{definition}
Каждый счет составляются по институциональным секторам и экономике в целом \\
Институциональные сектора 
\begin{enumerate}
    \item нефинансовые педприятия
    \item финансовые учреждения
    \item государственные учреждения
    \item некоммерческие органиации
    \item домашне хозяйства
\end{enumerate}
\section{Задача}
Есть рынок $A$ и рынок  $B$
\begin{tabular}{|c|c|c|c|}
    \hline
    Рынок A & & Рынок B & \\
    \hline
    $\phi$ &  $y_{i}$ & $\phi$ &  $y_{i}$ \\
    \hline
    1 & 60 & 1 & 30\\
    \hline
    2 & 30 & 2 & 25\\
    \hline
    3 & 5 & 3 & 25\\
    \hline
    4 & 5 &4 & 20 \\
    \hline
\end{tabular}
\begin{enumerate}
    \item  Cчитаем индекс концетрациии
        \begin{enumerate}
            \item
            \[
            I_{c} (2) =90
            .\] 
            \[
            I_{c} (3) = 95
            .\] 
            \[
            I_{c} (4) = 100
            .\] 
            \item
                \[
                I_{c}(2) = 55
                .\] 
                \[
                I_{c}(3) = 80
                .\] 
                \[
                I_{c}(4) = 100
                .\] 
        \end{enumerate}
    \item Считаем ндекс Гершиндаля-Хиршмана
        \begin{enumerate}
            \item 
                \[
                I_{hh}  = 4550
                .\] 
            \item $I_{hh} = 2500$
        \end{enumerate}
        Оба рынка высококонцетриррованные
    \item считаем Индекс Линда для двух. У первого $6.21$, у второго  $1.15$
    \item $\sigma^2 = \sum_{i = 1}^{n} \frac{ ( Y_{i} -  Y_{\text{avg}} )^2 }{n} $
    \begin{enumerate}
        \item 
    \end{enumerate}
\item $E = \sum_{i = 1}^{n} Y_{i} \ln{\frac{1}{Y_{i}}}$
\end{enumerate}
Построим и расчитаем кривые лоренца и индексы джини для двух рынков\\
\begin{tabular}{|c|c|c|c|}
    A & & B &\\
    \hline
    Доля в ч.& Кумулятивная доля & $y_{i}$ & кд \\
    25 & 25 & 60 & 60 \\
    \hline
    25 & 50 & 30 & 90 \\
    \hline
    25 & 75 & 5 & 95 \\
    \hline
    25 & 100 & 5 & 100\\
    \hline
\end{tabular}\\
\begin{tabular}{|c|c|c|c|}
    A & & B &\\
    \hline
    Доля в ч.& Кумулятивная доля & $y_{i}$ & кд \\
    25 & 25 & 30 & 30 \\
    \hline
    25 & 50 & 25 & 55 \\
    \hline 
    25 & 75 & 25 & 80 \\
    \hline
    25 & 100 & 20 100\\
    \hline
\end{tabular}
\section{Еще задача}
\begin{tabular}{|c|c|c|c|c|c|}
    \hline
    N & $P_{A}$ & $P_{B}$ & $Q_{A}$ & $Q_{B}$ & Расход (R)\\
    \hline
    1 & 4 & 5 & 8 & 10.4 & 84 \\
    \hline
    2 & 6 & 3 & 8 & 11 & 81\\
    \hline
    3 & 8 & 2 & 8 & 6.5 &77 \\
    \hline
\end{tabular}\\
Индекс расходов $I_{r} = \frac{\sum_{i = 1}^{n} P_{i1} Q_{1i}}{\sum_{i=1}^{n} P_{0i} Q_{0i}} = \frac{81}{84} = 0.96$
\[
I_{L} = \frac{\sum_{i = 1}^{n} P_{1i} Q_{0i}}{\sum_{i = 1}^{n}P_{0i}Q_{0i}} = 0.94
.\] 
Индекс ласпейреса
\[
    I_{p}=\frac{\sum P_{1i} Q_{1i}}{\sum P_{01} Q_{1i}} = 0.93
.\] 
Индекс Пааше
\[
I_{f} = \sqrt{I_{p}  I_{l}}  = \sqrt{0.94 * 0.93}  = 0.93
.\] 
Индекс фишера
\section{}
ВНП можно посчитать
\begin{enumerate}
    \item по потоку доходов. Сколько было выплаченно по зарплате и жалованию + Доходы от 
        собсвенности + Рентные платежи + Прибыль корпораций + проценты за кредит + субсидии гос предприятиям
        - косвенные налоги + плата за потребление капитала
    \item по потоку расходов. Потребительские расходы + валовые частные инвестиции +
        госудасрвтенные закупкии + чистый экспорт товаров и услуг
    \item по добавленой стоимости
\end{enumerate}
дефлятор внп - цена рыночной корзины выданной в этом году делены на цены аналогичной рыночной корзины в базовом году\\
\section{Задачи}
\subsection{}
Ягоды продали за 120, к ним доавили сахар за 20, из этого сварено варенье, проданное за 200. 
Тесто стоило 50. Пирог из этого продано за 300ю Внп = $120 + 60 + 50 = 230$
\subsection{}
В прошлой задаче добили ндс  $10\%$  $120*1.1 + 60 * 1.1 + 60 * 1.1 = 253$ Ндс =  $12+6 +5 = 23$, НД $=230$ 
Цена пирога $253 + 20 + 50 = 323$
\subsection{}
Теперь бабушка Нина эстонка. Внп =  $60 + 50$, c ндс  $110 * 1.1=121$.\\
Пусть для эстонцев введена пошлина 18 рублей. Тогда внп  $120 + 18 + 121 = 259$
\subsection{}
Теперь все бабушки гражданки РФ. Бабушка Катя не согласна за 132, а только за 120.
Бабушка Лииза не согласна за $120+2-+60*1.1 = 206$, а только за 200. Покупатель пирога согласен 
заплатть только 300. Насколько пирог  измени внп? \\\
ВНП =  $110 + 54 + 45 + 21 =  230$  Нд =  $110+54+23$ Препод сказал, что тут ошипка
\section{}
ВВП номинальный 1700, дефлятор  $50\%$, ввп реальный, так как уровень цен снизился $3400$

\section{}
\begin{enumerate}
    \item ВНП 100
    \item Амортизация основоного капитала - 10
    \item Расходы домохозяйств на приобретение товаров и услуг - 50
    \item гос закупки - 15
    \item чистый экспорт - 4
    \item косвеные налоги - 5
    \item трасфертные платежи - 2.5
\end{enumerate}
Национальный доход = 100 - 10
\section{Рынок денег}
\begin{enumerate}
    \item Сущность и функции денег
    \item Предложение денег
\end{enumerate}
\begin{tabular}{|c|c|c|}
    \hline
    Участники обмена & До & После\\
    \hline
    Антон &  $60a$ & $42a + 18b$\\
    \hline
    Борис &   $60 b$ &  $22b + 18a + 10 v$\\
    \hline
    Василий &  $40v$ &  $16v + 10b + 14g$\\
    \hline
    Глеб &  $30g$ & $16g + 14v$ \\
    \hline
\end{tabular}
\section{Задача}
Реальный ВНП в прошлом году (2022) составил 800
\[
    \text{ВНП}_{2022} = 800
.\] 
\[
    \text{ВНП}_{2023} = 840
.\] 
Расчитать темпы роста и темпы прироста по сравнению с прошлым годом
\[
V = \frac{840}{800} * 100 = 105 \%
.\] 
темп роста
\[
    \tilde{V} = \frac{840 - 800}{800} * 100 = 5 \%
.\] 
\section{Задача}
В 2022 году, страна имела следущие показател
\[
    \text{ВНП} = 500
.\] 
\[
    \text{Чистые инвестиции частного сектора} = 75
.\] 
\[
    \text{Гос. закупки} =  80
.\] 
\[
    \text{Потребление домашних хозяйств}  = 250
.\] 
\[
    \text{Прямые налоги} = 30
.\] 
\[
    \text{Косвенные налоги} =20
.\] 
\[
    \text{Субсидии} = 250
.\] 
\[
    \text{Экспорт} = 150
.\] 
\[
    \text{Импорт} = 110
.\] 
Надо посчитать распологаемый доход домашних хозяйст, амортизационный фонд, состояние бюджета.\\
Национальный доход по использовани =  Потребление домашних хозяйств, частные инвестиции + Гос.Закупки + Экспрт - Импорт =  $250 + 75 + 80 + 150 - 110 = 445$ \\
Располагаемый доход = НД - Прямые налоги = 415\\
Амортизационный фонд = ВНП  -Чистый национальный продукт\\
    Чистый национальный продукт = НД + Косвенные налоги - Субсидии = 445 + 20 - 25 = 440\\
    Амортизационный фонд = 500 - 440 = 60\\
    Состояние бюджета = Гос.Расходы -  Поступления в бюджет = Гос.закупки + субсидии- (прямые + косвенные) = 55
\section{Задача}
В экономике без госуарства и без заграницы промежуточный продукт составляет 100, потребление домашних хозяйств 50, частнные (брутто) инвестиции 30, совокупный продукт 180, заработная плата 60, cумма амортизация 4,нераспределенная прибыль 1.\\
Нужно составить для предпринимательского сектора счета 
\begin{enumerate}
    \item счета производства
    \item использования дохода
    \item имущества
\end{enumerate}
Для домашних составить счет дохода и изменение имущества\\
Составить модель баланса\\
\begin{tabular}{|c|c|c|c|}
    Дебет & & & Кредит\\
    \hline
    Промежуточный продукт & 100 & &\\
    \hline
    Зп & 60 & & \\
    \hline
    Ам & 4 & & \\
    \hline
    Нераспроданный продукт & 1 & & \\
    \hline
    Распроданный продукт  & 15 & & \\
\end{tabular}
\end{document}
