\documentclass[14pt]{extarticle}
\usepackage{fontspec}
\usepackage[russian, english]{babel}
\setmainfont{Times New Roman}
\usepackage{amssymb}
\usepackage{setspace}
\onehalfspacing
\usepackage{amsmath}
\usepackage{amsthm}
\usepackage{listings}
\usepackage{indentfirst}
\usepackage{hyperref}
\setlength{\parindent}{1.25cm}
\usepackage[right=10mm,left=30mm,top=20mm,bottom=20mm]{geometry}
\newtheorem{theorem}{Теорема}
\newtheorem{definition}{Определение}
\newtheorem{zam}{Замечание}
\newtheorem{corollary}{Следствие}[theorem]
\newtheorem{lemma}[theorem]{Лемма}
\title{Лекции по дифференциальным уравнениям}
\author{}
\date{}
\begin{document}
\maketitle
\section{Список литературы}
\begin{enumerate}
	\item Филиппов Лекции по обыкновенным ОДУ.
	\item Филиппов Сборник задач по дифференциальным уравнениям
	\item Петровский Лекции по ОДУ
	\item Самойленко, Кривошея, Перестрюк. ДУ. Примеры и задачи.
	\item Антидемидович
\end{enumerate}
\section{ДУ первого порядка}
\[
	f(x,y(x),y'(x)) = 0
	.\]
\begin{enumerate}
	\item x - независимая переменная
	\item $y(x)$ неизвестная функция
	\item  $y'(x)$ ее производная
\end{enumerate}
Решить ДУ -- найти $y(x)$
\section{Примеры}
\begin{enumerate}
	\item $y'(x) = y(x)$
	\item Найти  $y(x) ~$  $y'(x) = f(x)$
\end{enumerate}
\section{ДУ n-го порядка}
\[
	f(x,y,y',\dots,y^{(n)})
	.\]
\section{Ду разрешимое относительно производных}
\[
	y' = f(x,y)
	.\]
Будем заниматься только такими уравнениями
\section{Модель экспонециального роста (эпидемий)}
\[
	x(t) - \text{число бактерий}
	.\]
\[
	\dot{x}(t) ~ x(t)
	.\]
\[
	\dot{x} = kx
	.\]
\[
	x = C e^{kt}, C \in \mathbb{R}
	.\]
Модель роста с учетом эффекта насыщения называется логистической моделью.\\
Пусть $N$ -- максмимальное количество особей.
\[
	\dot{x} = kx (N - x)
	.\]
\[
	\dot{x} ~\text{максимальная, при } x = \frac{N}{2}
	.\]
\section{Интегрируемые дифференциальные уравнения первого порядка}
Знакомимся c уравнениями, которые можем решить в явном виде.
\subsection{Общие определения}
\begin{definition} \label{diff1}
	ДУ 1 порядка, разрешенным относительно производной называется уравнение вида
	\[
		y' =  f(x,y)
		.\] \label{1}
	где $x$ независимая переменная,  $y(x)$ искомая функция. Решить ДУ \ref{1} $\iff$
	найти $y(x)$. Будем считать $f : G \to \mathbb{R}, G \subset \mathbb{R}^2$, $G$ - связное и открытое множество\\
	В этой глае  $f$ будет элементарной (школьной) функцией. Так же считается, что $G$ область опредления уравненияю
\end{definition}
\begin{definition}
	Функция $y = \phi(x)$ называется решением ДУ  \ref{diff1} на промежутку $<a,b>$,
	если выполнены 3 условия
	\begin{enumerate}
		\item $\phi \in C^{1}(<a,b>)$ дифференцируема один раз на $<a,b>$
		\item $\text{Граф}~\phi := \{(x,(\phi(x)) \mid x \in <a,b>\} \subset G$
		\item $\phi'(x) \equiv f(x,\phi(x))$ на  $<a,b>$
	\end{enumerate}
\end{definition}
\begin{definition}
	График решения называют интегральной кривой ДУ(\ref{diff1}).
\end{definition}
\subsubsection{Самый простой пример}
\[
	y' = f(x)
	.\]
\[
	y =  \int\limits_{x_0}^{x} f(s) ds + C, C \in \mathbb{R}
	.\]
Решений бесконечно много. Общее решение ДУ(\ref{diff1}) имеет вид
\[
	y = \phi(x,C), C \in \mathbb{R}
	.\]
\subsubsection{Задача Коши}
Задано начальное значение решения.
\begin{definition} \label{kosh}
	Задача Коши -- называется задача следущего вида
	\begin{equation}
		\begin{cases}
			y' = f(x,y) \\
			y(x_0) = y_0
		\end{cases}
	\end{equation}
	где, $(x_0,y_0) \in G$
	Надо найти решения, которые удовлетворяют данному условию. Начальное условие означает, что график решения проходит через $(x_0,y_0)$
\end{definition}
\begin{definition}
	Говорят, что задача Коши \ref{kosh} имеет единственное решение или $(x_0,y_0)$
	есть точка единственности, если существует окрестность $(x_0,y_0) \in U$, такая что для любых двух интегральных кривых $\Gamma_1,\Gamma_2$, проходящих через точку $(x_0,y_0)$ выполняется $\Gamma_1 \cap U = \Gamma_2 \cap U$ (все интегральные кривые, проходящие $(x_0,y_0)$ совпадают в окрестности $U$). На языке епсилон-дельта
	\[
		\exists (x_0 -\delta,x_0 + \delta) \forall   \phi_1 , \phi_2 ~\text{Решения \ref{kosh}}~ \phi_1 \equiv \phi_2~  \text{на}~ (x_1 -\delta , x_0 +\delta)
		.\]
	Если имеет не единственное решение то $\forall U$ открестности найдутся 2 интегральные кривые различаются в окрестности.
\end{definition}
В дифференицальных уравнениях единственность понимаемся в локальном смысле.
\subsubsection{Пример Пеано}
\[
	\begin{cases}
		y' = 3y^{\frac{2}{3}} \\
		y(0) = 0
	\end{cases}
	.\]
Преверяем 2 решения
\[
	\phi_1 (x) = 0
	.\]
\[
	\phi_2(x) = x^{3}
	.\]
В любой окрестности  $\phi_1 \neq \phi_2$
\subsection{Уравнения с разделяющимся переменными}
\begin{definition}\label{urp}
	УРП -  уравнение вида $y' = f(x)g(y)$.
	Всегда предполагаем, что
	\begin{enumerate}
		\item f $\in C<a,b>$
		\item  $g \in C<\alpha,\beta>$
	\end{enumerate}
	\[
		G =  <a,b> \times <\alpha,\beta>
		.\]
\end{definition}
Как это решать придумал Якоб Бернулли.
\subsubsection{Неформальный рецепт}
\[
	\frac{dy}{dx} = f(x)g(y)
	.\]
\[
	\frac{dy}{g(y)} = f(x) dx
	.\]
\[
	\int \frac{dy}{g(y)} = \int f(x) dx + C
	.\]
\subsubsection{Нормальное доказательство}
\begin{theorem}
	\[
		f \in C(<a,b>)
		.\]
	\[
		g \in C(<\alpha,\beta>)
		.\]
	\[
		g(y) \neq 0 \forall  y \in <\alpha,\beta>
		.\]
	\begin{enumerate}
		\item $H(y) $ первообразная $\frac{1}{g(y)}$
		\item $F(x)$ первообразная  $f(x)$
	\end{enumerate}
	\begin{enumerate}
		\item Тогда формула из неформального рецепта задает общее решение  и только его
		\item $G = <a,b> \times <\alpha,\beta>$ -- область существоавания и единственности, для любой точки  $(x_0,y_0) \in G$ задача задача Коши имеет единственное решение.
		      И это решение задается формулами.
		      \[
			      H(x) = F(x) + H(y_0) - F(x_0)
			      .\]
	\end{enumerate}
\end{theorem}
\begin{proof}
	Заметим, что  $H'(y) = \frac{1}{g(y)} \neq 0$ $\implies \exists  H^{-1}$
	Тогда получаем
	\[
		y = H^{-1} (F(x) + C)
		.\]
	Формула 2 задает функции вида $y  = y(x)$
	\begin{enumerate}
		\item  Пусть $y = y(x)$ есть решение уравнения. Покажем, что оно вкладывается в формулу.  $\exists  C_0 : H(y(x)) - F(x) + C_0$
		      \[
			      \frac{d}{dx} (H(y(x)) - F(x)) \equiv 0
			      .\]
		      \[
			      H'(y(x)) y'(x) - F'(x) = 0
			      .\]
		      \[
			      \frac{1}{g(y(x))} * f(x) g(y(x)) - f(x) \equiv 0
			      .\]
		\item Теперь обратно
		      \[
			      H(y(x)) \equiv F(x) + C
			      .\]
		      Продифференцировали
		      \[
			      H'(y(x)) * y'(x) \equiv F'(x)
			      .\]
	\end{enumerate}
	Берем произвольную точку из области $(x_0,y_0) \in G$
	\[
		C \equiv H(y(x)) - F(x)
		.\]
	Подставим начальные условия из задачи коши
	\[
		C = H(y(x_0)) - F(x_0) = H(y_0) - F (x_0)
		.\]
	Тоесть для любой точки из $G$ можно определить единственным образом
\end{proof}
Ответ писать, надо даже если обратная функция не выражается в элементарных вроде $H(y) = y + \arctg{(y))}$, эта хрень считается ответом
\[
	H(y) - F(x) = C
	.\]
\[
	U(x,y) = H(y) - F(x)
	.\]
$U$ -- интеграл ДУ
\begin{definition}
	$U(x,y)$ называется интегралом ДУ, если выполняются следущие аксиомы (свойства)
	\begin{enumerate}
		\item $U \in C^{1}$
		\item $U'_{y} \neq 0$ (производная по y не ноль)
		\item U обращается в константу при подставлении решения ДУ.
	\end{enumerate}
\end{definition}
Все свойства выполняются для $U(x,y)$
\begin{definition}[Линия уровня]
	\[
		U^{-1}(c) = \{(x,y) \mid U(x,y) = c\}
		.\]
\end{definition}
Теперь рассмотрим случай, когда $g(y) = 0$
Пусть  $g(a) = 0 \implies$ стационарное решение
\[
	a' \equiv f(x)(g(a))
	.\]
\begin{theorem}
	Пусть $f \in C(<a,b>), g \in C^{1}(<\alpha,\beta>)$
	Тогда все решения уравнения задаются совокупностью
	\[
		H(y) = F(x) + c
		.\]
	\[
		g(y) = 0 ~\text{совокупность всех стационарных решений.}
		.\]
\end{theorem}
Эту хрень не доказываем, так как будет следовать из теоремы Пикара.
\begin{enumerate}
	\item Пример Пеано
	      \[
		      y' = 3 y^{\frac{2}{3}}
		      .\]
	      \[
		      y = 0
		      .\]
	      решение
	      \[
		      \int \frac{y' dy}{3y^{\frac{2}{3}}} = \frac{1}{3} *x^{1/3} * 3 + c = x^{\frac{1}{3}} + c
		      .\]
	      \[
		      y^{\frac{1}{3}} = x + c
		      .\]
	      \[
		      y = (x + c)^{3}
		      .\]
	      \[
		      \left[
		      \begin{gathered}
			      y = 0\\
			      y = (x + c)^{3}
		      \end{gathered}
		      \right
		      .\]
	      В точках $y = 0$ нарушается единственность, решения можно склеивать, получать новые, не из совокупности
	      \[
		      y =
		      \begin{cases}
			      (x - c_1)^{3} ,x <  c_1 \\
			      0 , c_1 \le  x \le  c_2 \\
			      (x - c_2)< x > c_2
		      \end{cases}
		      .\]
	      $g(y)$ не дифф в 0, условия теоремы 2 не выполняются.
	\item
	      \[
		      y' = y
		      .\]
	      $y = 0$ решение
	      \[
		      \int \frac{y'}{y} dy = \ln{y} +  c
		      .\]
	      \[
		      \ln{(|y|)} = x + c
		      .\]
	      \[
		      |y| = e^{x + c}
		      .\]
	      \[
		      \left[
		      \begin{gathered}
			      y = 0\\
			      y = \pm c^{x}y e^{c}
		      \end{gathered}
		      \right
		      .\]
	      \[
		      c_1 = \pm e^{c}
		      .\]
	      \[
		      \left [
		      \begin{gathered}
			      y = 0\\
			      y = C_1 e^{x}, \forall  c_1 \neq 0
		      \end{gathered}
		      \right
		      .\]
	      единственность не нарушена
\end{enumerate}
\section{Линейный уравнения (Неоднородное)}
\begin{definition}[Линейное уравнение (неоднородное)]\label{lin}
	\[
		y' = p(x) y + q(x)
		.\]
\end{definition}
\begin{definition}[Линейное однородное уравнение]\label{odn}
	\[
		y' = p(x) y
		.\]
\end{definition}
Ввел эти уравнения И.Бернулли. Как решать Д.Бернулли.
Рассмотрим вариант подстановки Бернулли, который в литературе называют метод вариации произвольной постоянной Лагранжа (метод Лагранжа).
\subsection{}
Пусть $p,q \in C(<a,b>)$. Тогда правая часть  $\ref{lin}$
$f(x,y) = p(x) y + q(x) \in C(G) G = <a,b> \times \mathbb{R}$ $G$ -- область определения \ref{lin}
\begin{theorem}
	Пусть $p,q$ непрерывны и есть только одно решение. Все решения линейного уравнения \ref{lin} задаются формулой
	\begin{eqnarray}\label{linsolve}
		y = e^{\int\limits_{x_0}^{x} p(t) dt } (C +\int\limits_{x_0}^{x} q(s) e^{-\int\limits_{x_0}^{s} p(t) dt } ds  )
	\end{eqnarray}
	$\forall  (x_0,y_0) \in G$ задача Коши имеет одно решение и оно задается формулой
	\begin{equation}
		y = e^{\int\limits_{x_0}^{x} p(t) dt } (y_0 +\int\limits_{x_0}^{x} q(s) e^{-\int\limits_{x_0}^{s} p(t) dt } ds  )
	\end{equation}
\end{theorem}
\subsubsection{Метод лагранжа}
\begin{enumerate}
	\item Берем однородное уравнение $y' = p(x) y$
	\item Это уравнение с разделяющимися переменными. Умеем решать.
	      \[
		      y = Ce^{\int\limits_{x_0}^{x} p(t) dt }
		      .\]
	\item В уравнение \ref{lin} замену переменных $y = z e^{\int\limits_{x_0}^{x} p(t)dt }$
	\item Делаем замену переменных
	      \begin{equation}
		      (z  e ^{\int\limits_{x_0}^{x} p(t) dt })' =
		      p(x) (z e^{\int\limits_{x_0}^{x} p(t) dt  }) + q(x)
	      \end{equation}
	      \[
		      z' e^{\int\limits_{x_0}^{x} p(t) dt } + z * e^{\int\limits_{x_0}^{x} p(t) dt } p(x) = p(x) z^{\int\limits_{x_0}^{x} p(t) dt  } + q(x)
		      .\]
	      \[
		      z' e^{\int\limits_{x_0}^{x} p(t) dt } = q(x)
		      .\]
	      \[
		      z' = q(x) e^{-\int\limits_{x_0}^{x} p(t) dt }
		      .\]
	      Тупо найти первообразную.
	      \[
		      z = \int\limits_{x_0}^{x}  q(s) e^{-\int\limits_{x_0}^{s} p(t) dt  }  ds + C
		      .\]
\end{enumerate}
\begin{proof}
	Сначала выводим формулу \ref{linsolve} Методом Лагранжа. Покажем, что $G$
	область существования и единственности. Берем  $\forall  (x_0,y_0) \in G$. Надо показать, что существует только одно решение график, которого проходит через эту точку. В \ref{linsolve} подставим $(y_0,x_0)$
	\[
		y_0 = e^{\int\limits_{x_0}^{x_0} p(t) dt } ( C + \int\limits_{x_0}^{x_0} q(s)
		e^{-\int\limits_{x_0}^{s} p(t) dt  })
		.\]
	\[
		e^{0} = 0
		.\]
	Интеграл в скобках тоже 0, является решением
\end{proof}
\subsubsection{Какую замену переменных надо делать в ДУ}
\begin{enumerate}
	\item $G \subset \mathbb{R}^2$ поэтому замена $\mathbb{R}^2 \to \mathbb{R}^2$
	      \[
		      (x,y) \to (x,z)
		      .\]
	\item Биективность.
	\item Замена должна сохранять гладкость (дифференцируемость)
\end{enumerate}
\subsubsection{Пример}
\[
	y' = \frac{y}{x} + x
	.\]
\[
	G = \mathbb{R} \backslash \{x=0\}
	.\]
\begin{enumerate}
	\item
	      \[
		      y' = \frac{y}{x}
		      .\]
	      $y = 0$ решение
	      \[
		      \int \frac{dy}{y} = \int \frac{dx}{x}
		      .\]
	      \[
		      \ln{|y|} = \ln{|x|} + c = \ln{C_1 |x|}
		      .\]
	      \[
		      y = \pm C_1 x , c_2 = \pm c_1
		      .\]
	      \[
		      y = c_2 x
		      .\]
	\item
	      Возвращаемся к ЛНУ
	      \[
		      y = z x
		      .\]
	      \[
		      (z x)' = \frac{z x}{x} + x
		      .\]
	      \[
		      z' x + z = z + x
		      .\]
	      \[
		      z' = 1
		      .\]
	      \[
		      z = x + c
		      .\]
	      \[
		      y = (x + c) x
		      .\]
\end{enumerate}
\subsection{Уравнения сводящиеся к уравнениям с разделяющимися переменными}
\subsubsection{Однородные уравнения}
\begin{equation}\label{ou}
	y' = f(x,y)
\end{equation}
Такое уравнение называется однородным $\iff$ если не меняется при замене  $(x,y) \to  (\lambda x,\lambda y) $
\begin{equation} \label{huita}
	M(x,y) dx + N(x,y) dy = 0
\end{equation}
\[
	\frac{dx}{dy} = \frac{1}{f(x,y)}
	.\]
\ref{huita} объединяет оба уравния
\subsubsection{Пример}
\[
	(x^2 - xy) dx + y^2 dy = 0
	.\]
\[
	y' = \frac{x^2 + xy }{y^2} = - (\frac{x}{y})^2 + \frac{x}{y}
	.\]
Пытаемся эту хуиту свести к $y' = g(\frac{y}{x})$
\[
	(x,y) \to (x,z)
	.\]
\[
	(zx)' = g(z)
	.\]
\[
	z' = \frac{g(z) - z}{x}
	.\]
\subsubsection{Ф. 101}
\[
	(x + 2y) dx - x dy = 0
	.\]
Попробуем замену $(x,y) \to (\lambda x, \lambda ,y)$
\[
	\lambda^2 (x + 2y) dx - \lambda^2 x dy = 0
	.\]
На лямбду сократили, уравнение однородно
\[
	(x,y) \to (x,z), y = zx
	.\]
\[
	(x + 2zx) dx 0 d(z x) =
	.\]
\[
	(x + 2zx ) dx -x (zdx + x dz)= 0
	.\]
$x = 0$ реш вертикальная прямая
\[
	(1 + 2z) dx - xdz - zdx = 0c
	.\]
\[
	(1 + z) dx = xdz
	.\]
$z = -1 : y = -x $ да является решением, горизонтальная прямая
\[
	\ln{|z + 1|} = \ln{|x|} + c
	.\]
\[
	\ln|\frac{y}{x} + 1| = \ln|x| +  c
	.\]
\[
	x = 0
	.\]
\[
	y = -x
	.\]
\section{Уравнение в полных дифференцалах}
\begin{equation}
	M(x,y) dx + N(x,y) dy = 0
\end{equation}
Это уравнение в полных дифференциалах
\[
	y' = f(x,y)
	.\]
Такое уравнение можно переписать ввиде уравнения в полных диффереенциалох
\[
	dy - f(x,y) dx = 0
	.\]
В уравнении 2 $x,y$ неравноправны
\subsection{Пример}
\[
	y' = \frac{1}{yx + y^2}
	.\]
\[
	\frac{dx}{dy} = yx + y^2
	.\]
уравнение линейное по $x$. Иногда полезно перевенуть уравнение
\subsection{Смысл записи в полных дифференциалах}
Для тех точек, где $N(x,y) \neq 0$ мы считаем что это уравнение тоже самое что
\[
	\frac{dy}{dx} = \frac{M(x,y)}{N(x,y)}
	.\]
Для тех точек, где $M(x,y) \neq 0$ пы считаем по определнию
\[
	\frac{dx}{dy} = -\frac{N(x,y)}{M(x,y)}
	.\]
Уравнение в полных дифференциалах это объедение такие уравнений.

Если $M = N = 0$ считаем что уравнение не определно и такие точки выкидыватся из области определения
\begin{definition}
	$(x_0,,y_0)$ такие что  $M(x_0,y_0) = N(x_0,y_0) - 0$. О
\end{definition}
\[
	M(x,\phi(x))x + N(x,\phi) d\phi(x) =
	(M(x,\phi(x))) + \phi'(x) N(x,\phi(x)))dx
	.\]
\begin{lemma}
	Пусть $u \in C^{1}(D)$ $\exists U'_{x}, U'_{y} \in C(G)$ хотя бы одна из частных производных не равняетчя 0, тогда  $U(x,y) = 0$ задает регулярную кривую
\end{lemma}
\begin{proof}
	\[
		\forall  (x_0,y_0) : U(x_0,y_0) = 0
		.\]
	\[
		(x_0,y_0) \in U^{-1}(0) := \{(x,y) \mid U(x,y) = 0\}
		.\]
	Пусть $U'_{y}(x_0,y_0) \neq 0$ по теореме о неявной функции тогда $\exists ! y = \phi(x)$ $x \in (x_0-\delta,x_0 + \delta), \phi \in C_1$
\end{proof}
\section{Уравнения, сводящиеся к однородным}
\subsection{}
\[
	y' = f(ax + by)
	.\]
\[
	(x,y) \to (x,z),~ z = ax + by
	.\]
\[
	z' =a + by' = a + bf(ax + by) = a+ bf(z)
	.\]
\subsubsection{Ф 65}
\[
	y' = \sqrt{4x + 2y - 1}
	.\]
\[
	(x,y) \to (x,z) , z=4x + 2y - 1
	.\]
\[
	z' = 4 + 2y' = 4 + 2\sqrt{z}
	.\]
\[
	\frac{dz}{d} = 4 + 2\sqrt{z}
	.\]
\[
	\frac{dz}{2 + 1\sqrt{z} }= 2dx
	.\]
\[
	\int \frac{dz}{2 + \sqrt{z} }
	.\]
\[
	t = 2 + \sqrt{z}
	.\]
\[
	z = (t - 2)^2
	.\]
\[
	dz =  2(t-2)dt
	.\]
\[
	\int \frac{2(t - 1)}{t} dt = t - 2 \ln{|t|} = x+c
	.\]
\[
	2 + \sqrt{z}  - 2\ln{(2 + \sqrt{z} )}=  x+ c
	.\]
\[
	2 + \sqrt{(4x + 2y -1) } - 2\ln{(2 + \sqrt{4x + 2y - 1} )} = x +c
	.\]
\subsection{Еще какаято ебанина}
\[
	y' = f(\frac{a_1x + b_1y + c}{a_2 x + b_2 y + c_2})
	.\]
Если $c_1= c_2 = 0$ такое уравнение является однородным. Каждая хуйня из дроби задает уравнение прямой. $(x_0,y_0)$ точка пересечения, нужен паралельный пернос, чтоб она стала $(0,0)$
\[
	(x,y) \to (u,v)
	.\]
\[
	\begin{cases}
		u = x-x_0 \\
		v = y - y_0
	\end{cases}
	.\]
\[
	du = dx
	.\]
\[
	dv = dy
	.\]
\[
	\frac{dv}{du} = f (\frac{a_1 u + b_1 v}{a_2 u + b_2 v})
	.\]
коээфициенты не меняются так как нормаль не меняется при параллельном переносе.\\
Если прямые паралельны то
\[
	a_2 = k a_1
	.\]
\[
	b_2 = k b_1
	.\]
Задача сводитя к прошлой
\subsubsection{Ф 118}
\[
	y' = 2 (\frac{y + 2}{x + y -1}) ^2
	.\]
\[
	\begin{cases}
		y = -2 \\
		x = 3
	\end{cases}
	.\]
\[
	\begin{cases}
		u = x - 3 \\
		v = y + 2
	\end{cases}
	.\]
\[
	\frac{dv}{du} = 2 (\frac{v}{u + v})^2
	.\]
\[
	(u,v) \to (u,z)
	.\]
\[
	v = zu
	.\]
\[
	(zu)' = u + uz' = 2(\frac{zu}{u + zu})^2 = 2 (\frac{z}{1 + z}) ^2
	.\]
\subsection{ф 113}
\[
	(2x - 4y + 6) dx + (x + y -3) dy = 0
	.\]
\[
	\begin{pmatrix}
		1 & 1  & \vline & 3  \\
		1 & -2 & \vline & -3 \\
	\end{pmatrix}
	\to
	\begin{pmatrix}
		1 & 1  & \vline & 3  \\
		0 & -3 & \vline & -6
	\end{pmatrix}
	.\]
\[
	(1,2)
	.\]
\[
	\begin{cases}
		u = x - 1 \\
		v = u - 2
	\end{cases}
	.\]
\[
	(2 u - 4 v) d u + (u + v) dv = 0
	.\]
\[
	(u,v) \to (u,z)
	.\]
\[
	v = zu
	.\]
\[
	(2u - 4zu) du + (u + uz) d(uz) = 0
	.\]
\[
	(2u - 4zu) du + u^2(1 + z) dz + u(z + z^2) du= 0
	.\]
$u = 0 $ не реш
\[
	.\]
\section{}
\[
	M(x,y) dx + N(x,y) dy = 0
	.\]
Эта хуйня тоже самое что
\[
	\frac{dy}{dx} = \frac{M(x,y)}{N(x,y)} \land \frac{dx}{dy} = - \frac{N(x,y)}{M(x,y)}
	.\]
\[
	y = \phi(x)
	.\]
\[
	x = \psi(y)
	.\]
\[
	U(x,y) = 0
	.\]
\[
	|U'_{x}| + |U'_{y}|  \neq 0
	.\]
Неявное решение задания решения
\begin{definition}
	Пусть $M,N \in C(G), G \subset \mathbb{R}^2, |M|+|N| \neq 0$ в $G$ \\
	Пусть $U \in C^{1}(G)$
	\begin{enumerate}
		\item
		      \begin{equation}
			      |U'_{x} | + |U'_{y} | \neq 0
		      \end{equation}
		\item
		      \begin{equation}
			      U'_{x} N - U'_{y} M \equiv 0
		      \end{equation}
	\end{enumerate}
	Тогда $U(x,y)$ называется интегралом ДУ
\end{definition}
\begin{theorem}
	Пусть $U(x,y)$ интеграл дифференциального уравнения тогда
	\begin{enumerate}
		\item Формула $U(x,y) = c$\label{6} задает множество всех решений
		\item  $G$ область существования и единственности
	\end{enumerate}
\end{theorem}
\begin{proof}
	$\forall $ кривая задаваемая \ref{6} удовлетворяет условиям леммы. Пусть $\gamma$
	регулярная кривая, задааваема  $U(x,y) =  c$ , либо  $U'_{x} \neq 0$, либо $U'_{y} \neq 0$
	\begin{enumerate}
		\item Пусть $U'_{y} \neq 0 \implies$
		      \[
			      y = \phi(x)
			      .\]
		      \[
			      U(x,\phi(x)) \equiv c
			      .\]
		      \[
			      \frac{d}{dx} U(x,\phi(x)) \equiv 0
			      .\]
		      \[
			      U'_{x}(x,\phi(x)) + U'_{y}(x,\phi(x)) \phi'(x) \equiv 0
			      .\]
		      \[
			      \phi'(x) \equiv - \frac{U'_{x} (x,\phi(x))}{U'_{y} (x,\phi(x))}
			      .\]
		      Покажем,ч то $N(x,\phi(x)) \neq 0$. Предположим обратное
		      \[
			      0 - U'_{y} M = 0
			      .\]
		      значит $M = 0$ получается полная фигня. Тогда
		      \[
			      \frac{U'_{x}}{U'_{y}} \equiv \frac{M}{n}
			      .\]
		      \[
			      \phi'(x) = - \frac{M(x,\phi(x))}{N(x,\phi(x))}
			      .\]
		\item  $U'_{x} \neq 0$
		      \[
			      U(\psi(y),y) \equiv 0
			      .\]
		      Взяли производную
		      \[
			      U'_{x}(\psi(y),y)\psi'(y) + U'_{y}(\psi(y),y) \equiv 0
			      .\]
		      \[
			      \psi =  - \frac{U'_{y}(\phi,y)}{U'_{x}(\phi,y)}
			      .\]
		      \[
			      M(\psi(y),y) \neq 0
			      .\]
		      \[
			      \psi'(y) \equiv - \frac{N(\phi(y),y)}{M(\psi,y)}
			      .\]
	\end{enumerate}
	Теперь доказываем второй пункт. Берем $\forall (x_0,y_0) \in G$ и $\exists !$ решение, проходяшее через $(x_0,y_0)$
	\[
		U(x,y) = C
		.\]
	Если это решение локально представимо в виде $y = \phi(x)$, то $y_0 = \phi(x_0)$
	\[
		U(x,\phi(x)) \equiv c
		.\]
	\[
		U(x,y) = U(x,0)
		.\]
\end{proof}
Перейдем к уравнениям в полных дифференциалах
\section{УПД}
\begin{definition}
	\[
		M(x,y) dx + N(x,y) dy = 0
		.\]
	Называется уравнением в полных дифференциалилах, если существует $U \in C^{1}(G)$
	\[
		M = U'_{x}
		.\]
	\[
		N = U'_{y}
		.\]
\end{definition}
\begin{theorem}
	Пусть уравнение в полных дифференциалах, тогда $U$ интеграл
\end{theorem}
\begin{proof}
	\begin{enumerate}
		\item $|U'_{x}| + |U'_{y}|  = |M| + |N| \neq 0$ если оба ноль, то мы такое не рассматриваем
		\item
		      $U'_{x} N - U'_{y} M = M N - N M = 0$
	\end{enumerate}
\end{proof}
\begin{corollary}
	Формула $U(x,y) = c$ все решения в обобщеном смысле
\end{corollary}
\begin{corollary}
	G - облатсь существования и единственности
\end{corollary}
\begin{theorem}[Критерий УПД]
	Пусть $M,N \in C(G)$ ,  $\exists M'_{y}, N'_{x} \in C(G)$, Тогда
	\begin{enumerate}
		\item уравнение УПД $\iff$  $M'_{y} \equiv N'_{x}$ (необходимое условие упд)
		\item Если $G$ выпуклая прошлый пунк достаточное условие
	\end{enumerate}
\end{theorem}
\section{Обобщеннo однородные уравния}
\[
	y' = f(x,y)
	.\]
\[
	(x,y) \to (\lambda x, \lambda y)
	.\]
квазирастяжение
\[
	(x,y) \to (\lambda^{\alpha}  x , \lambda^{\beta} y)
	.\]
\[
	\begin{cases}
		x = \lambda^{\alpha} x_0 \\
		y = \lambda^{\beta} x_0
	\end{cases}
	\lambda > 0
	.\]
перейдем к уравнению в явной форме
\[
	\lambda = (\frac{x}{x_0})^{\frac{1}{\alpha}}
	.\]
\[
	y = (\frac{x}{x_0})^{\frac{\beta}{\alpha}} y_0
	.\]
Пусть $c = \frac{y_0}{x_0^{\frac{\beta}{\alpha}}}$
\begin{definition}
	Уравнение называется ООУ (Квази ОУ) , если существует
\end{definition}
\[
	2x^4 y y' + y^{4} = 4x^{6}
	.\]
\[
	x \to \lambda^{\alpha} x
	.\]
\[
	y \to \lambda^{\beta} y
	.\]
\[
	\lambda^{4\beta} = \lambda^{4\alpha}
	.\]
надо делать замену $z = \frac{y^{\alpha}}{x^{}}$
\section{Теорема}
\begin{theorem}
	\[
		M dx + N dy = 0
		.\]
	\[
		M, N \in C(G)
		.\]
	\[
		\exists  M'_{y} ,N'_{x} \in C(G)
		.\]
	Тогда Уравнение упд $\implies$  $M_{y}' \equiv N_{x}'$ \\
	Если $G$ -- выпуклая то прошлое еще и достаточное условие.
\end{theorem}
\begin{proof}
	\begin{enumerate}
		\item Необходимость\\
		      Пусть уравнение упд значит
		      \[
			      \begin{cases}
				      M = U_{x}' \\
				      N = U_{y}'
			      \end{cases}
			      .\]
		      \[
			      \begin{cases}
				      M_{y}' = U''_{xy} \\
				      N_{x}' = U''_{yx}
			      \end{cases}
			      .\]
		      Значит $M_{y}' \equiv N_{x}'$
		\item Достаточность\\
		      Ищем функцию $U(x,y)$ в явном виде.
		      \[
			      \begin{cases}
				      U'_{x} = M \\
				      U'_{y} = N
			      \end{cases}
			      .\]
		      Первое уравнение интегрируем по $x$, тогда
		      \[
			      U(x,y) = \int_{x_0}^{x} M(s,y) ds + c(y)
			      .\]
		      Подбираем $c(y)$
		      \[
			      (  \int\limits_{x_0}^{x} M(s,y) ds + c(y) )' = N(x,y)
			      .\]
		      \[
			      (\int\limits_{x_0}^{x} M(s,x) ds )'_{y} + c'(y) = N(x,y)
			      .\]
		      \[
			      \int\limits_{x_0}^{x}  M(s,y)  sd = F(x,y)
			      .\]
		      \[
			      F'(x,y)_{y} = \int\limits_{x_0}^{x} M'_{y} (s,y)ds
			      .\]
		      \[
			      \int\limits_{x_0}^{x} M'_{y} (s,y) ds + c'(y) = N(x,y)
			      .\]
		      \[
			      \int\limits_{x_0}^{x}  N'_{x}(s,y)ds + c'(y) = N(x,y)
			      .\]
		      \[
			      N(x,y) - N(x_0,y) + c'(y) = N(x,y)
			      .\]
		      \[
			      c'(y) = N(x_0,y)
			      .\]
		      \[
			      c(y) = \int\limits_{y_0}^{y} N(x_0,t) dt
			      .\]
		      Без константы, тк нам надо найти тоько одну такую функцию.
		      \[
			      U(x,y) = \int\limits_{x_0}^{x}  M(s,y)  ds + \int\limits_{y_0}^{y} N(x_0,t) dt
			      .\]
	\end{enumerate}
\end{proof}
\section{Общие теоремы для систем дифференциальных уравнений}
\subsection{Основные понятия}
\begin{definition}
	Системой дифференциальных уравнений, разрешенных относительно старших производных,
	называется следущая система
	\[
		y_1^{(n_1)} (x) = f_1(x,y_1,y_1',y_1^{(n_1 - 1)},y_2,\dots,y_{2}^{( n_2-1 )},\dots,y_{m},\dots,y_{m}^{n_{m} - 1})
		.\]
	\[
		y_2^{(n_2)} = f_2(\text{тоже самое что и прошлом})
		.\]
	\[
		\dots
		.\]
	\[
		y_{m}^{(n_{m})} = f_{m} (\text{тоже самое что и в прошлом})
		.\]
	где x независимая переменнная, $y_1(x),\dots,y_{m}(x)$ неизвестные функции.
	Решить систему, найти эти функциии. ,$n_1,\dots,n_{m}$ порядки старших производных
	\[
		n = n_1 +  \dots + n_{m}~  \text{порядок сду}
		.\]
	Важные случаи
	\begin{enumerate}
		\item \[
			      m=1,n_1 = n
			      .\]
		      Это диффур n-го порядка
		      \[
			      y^{(n)} = f(x,y,y',\dots,y^{( n-1 )})
			      .\]

		\item
		      \[
			      n_1 = \dots = n_{m} = 1, n =m
			      .\]
		      Нормальная система
		      \[
			      \begin{cases}
				      y'_{1} =  f_1(x_1,y_1,\dots,y_{n}) \\
				      y_2' = f_2(x_1,y_1,\dots,y_{n})    \\
				      \dots                              \\
				      y_{n}' = f_n(x_1,y_1,\dots,y_{n})  \\
			      \end{cases}
			      .\]
	\end{enumerate}
\end{definition}
\begin{lemma}
	Любая система дифференциальных уравний может быть записана в виде эквивалентной нормальной системы
\end{lemma}
\begin{proof}
	Начинаем с частного случая. Диффур n-го порядка записываем в виде нормальной системы. ВВедем новые переменные $z_1,\dots,z_{n}$
	\[
		\begin{cases}
			z_1 = y  \\
			z_2 = y' \\
			\dots    \\
			z_{n} = y^{(n - 1)}
		\end{cases}
		.\]
	\[
		\begin{cases}
			z_1' = y' = z_2 \\
			z_2' = y''= z_3 \\
			\dots           \\
			z_{n}' = y^{(n)} = f(x,z_1,z_2,\dots,z_{n})
		\end{cases}
		.\]
	Убедимся в сущ биекции между решениями. Если $y =  \phi(x)$ тогда
	\[
		z_1 = \phi(x), z_2 = \phi'(x) ,\dots,z_{n} = \phi^{(n-1)}(x)
		.\]
	Обратно
	\[
		z_1= \phi_1(x),\dots,z_{n} = \phi_{n}(x)
		.\]
	то  $y =\phi'(x)$
	\[
		\phi_{k+1}(x) = \phi'_{k}(x)
		.\]
	\[
		\phi'_{n} = \phi^{(n)}(x) = f(x,\phi_1,\dots,\phi_{n}) = f(x,\phi_1,\dots,\phi_1^{(n-1)})
		.\]
	Теперь общий случай.
	\[
		y_{k}, y'_{k},\dots,y^{(n_{k} - 1)}_{k}
		.\]
	берем  в качестве новых переменнных
\end{proof}
\subsection{Векторная запись нормальной системы дифференциальных уравнений}
\[
	y(x)  = \begin{pmatrix}
		y_1(x) \\
		\dots  \\
		y_{n(x)}
	\end{pmatrix}
	.\]
\[
	<a,b> \to \mathbb{R}^{n}
	.\]
\[
	y(x) \in C(<a,b>) := \forall  y_{k}(x) \in C(<a,b>), k=1\dots n
	.\]
\[
	y(x) \in C^{1}(<a,b>) := \forall  y_{k} \in C^{1}(<a,b>)
	.\]
\[
	y'(x) := \begin{pmatrix}
		y'(x) \\
		\dots \\
		y'_{n}(x)
	\end{pmatrix}
	.\]
\[
	\int\limits_{a}^{b} y (x)  dx := \begin{pmatrix}
		\int\limits_{a}^{b} y_1(x)   dx \\
		\dots                           \\
		\int\limits_{a}^{b} y_{k} (x) dx
	\end{pmatrix}
	.\]
\[
	f(x,y) := \begin{pmatrix}
		f_1(x_1,y_1,\dots,y_{n}) \\
		\dots                    \\
		f_{n}(x,y_1,\dots,y_{n})
	\end{pmatrix}
	.\]
\[
	y(x)' = f(x,y)
	.\]
векторная система нормальной системы\\
Вводим переобозначения
\[
	x \to t
	.\]
\[
	y \to x \text{-- это набор координат}
	.\]
\[
	y(x) \to x(t)
	.\]
\[
	y' = \dot{x}
	.\]
\[
	\dot{x} = f(t,x)
	.\]
\[
	G \subset \mathbb{R}^{n =1}
	.\]
\[
	f \in C(G)
	.\]
\begin{definition}
	\[
		\phi(t) : <a,b> \to \mathbb{R}^{n}
		.\]
	называется решением нормальным решением если
	\begin{enumerate}
		\item \[
			      \phi \in C^{1}(<a,b>)
			      .\]
		\item
		      \[
			      \Gamma_{\phi} \subset G
			      .\]
		\item \[
			      \dot{\phi(t)} \equiv f(t,\phi(t)) \text{на} <a,b>
			      .\]
	\end{enumerate}
\end{definition}
Задача Коши.
\[
	(t_0,x_{0}) \in F
	.\]
\[
	\begin{cases}
		\dot{x} = f(t,x) \\
		x(t_0) = x_0
	\end{cases}
	.\]
\subsection{Единственность решения задачи коши}
$(t_0,x_0) \in G$ называется единсвтенной, если $\exists  U(t_0,x_0)$ такая что любые две инетгральные кривые, проходящие через $t_0,x_0$ в этой окрестности $U$ совпадают.
\section{Поле направлений. Ломанные направлений. Теорема Пеано}
Рассматриваем нормальную систему диффуров
\[
	\dot{x} = f(t,x)
	.\]
\[
	f: G \to \mathbb{R}^{n}
	.\]
\[
	G \subset \mathbb{R}^{n+1} \text{область}
	.\]
\[
	f \in C(G)
	.\]
\subsection{Поле направлений}
\[
	(t_0,x_0) \in G
	.\]
Пусть  $\phi(x_0) = t_0$ решениe системы. Само решение $\phi$ неизвестно, но
мы легко можем написать уравнение касательной к  $\phi$ в  $t_0$
\[
	x = \phi(t_0) + \dot{\phi}(t_0) (t - t_0)
	.\]
\[
	\dot{\phi}(t_0) = f(t_0,\phi(t_0)) = f(t_0,x_0)
	.\]
\[
	x = x_0 + f(t_0,x_0)(t -t_0)
	.\]
выписано в явном виде.
\begin{definition}[Поле направлений]
	Полем направлений называется отображения, которое $\forall  (t_0,x_0) \in G$ сопоставляет касательную прямую, проходящую через эту точку.
\end{definition}
Ясно что $\phi$ будет решением  $\iff$  в каждой своей точки касается прямой из поля направлений. Интегральные кривые в каждой точку поля направний  $\implies$ позволяет примерно рисовать их кривые.
\subsection{Ломанные Эйлера}
\begin{definition}[Ломанная Эйлера]
	Ломанной Эйлера называется любая ломанная, звенья которой, лежат на прямых из поля направлений. Апроксимация точной интегральной кривой.
\end{definition}
\subsection{Алгоритм построения ломанной Эйлера}
\[
	(t_0,x_0) \in G
	.\]
Строим лиоманную эйлера на $[t_0.T]$ в лево все аналогично.\\
Пусть N - число звеньев ребер
\[
	\frac{T - t_0}{N} = h ~\text{Шаг}
	.\]
Пусть $t_{k} = t_0 + kh$ дробление отрезка $[t_0,T]$ $k = 0\dots N$.
Линия эйлера строится по реккурентному алгоритму по следущей схеме
\[
	l(t) = x_0 + f(t_0,x_0)(t - t_0), [t_0,t_1]
	.\]
\[
	x_1 = l(t_1) = x_0 + f(t_0,x_0)h
	.\]
\[
	l(t) = x_1 + f(t_1,x_1))(t- t_1),  [t_1,t_2]
	.\]
\[
	x_2 = l(t_2)
	.\]
Пусть Ломанная Эйлера построена на $[t_0,t_{k}],x_{k} = l(t_{k})$
\[
	[t_{k_1} , t_{k + 1}] , l(t) = x_{k} + f(t_{k},x_{k}) (t - t_{k})
	.\]
\[
	\begin{cases}
		t_{k + 1} = t_{k}  + h \\
		x_{k + 1} = x_{k} + f(t_{k},x_k)h
	\end{cases}
	.\]
\subsection{Теорема Пеано}
\begin{theorem}[Пеано]
	Пусть $f \in C(G)$ тогда  $\forall  (t_0,x_0) \in G$ задача коши
	\[
		\dot{x} = f(t,x)\\
		x(t_0) = 0
		.\]
	имеет хотя бы одно решение (определенное на отрезке $[t_0-h,t_0 + h]$ )
\end{theorem}
Без доказательства
\section{Интегральные уравнения}
\[
	\dot{x} = f(t,x)
	.\]
\[
	f : G \to \mathbb{R}^{n}, G \subset \mathbb{R}^{n + 1}
	.\]
\[
	f \in C(G)
	.\]
Рассмотрим следущее интегральное уравнение.
\[
	x(t) = x_0 + \int\limits_{t_0}^{t} f(x,x(s))  ds
	.\]
Решить это уравние означает найти $x(t)$
\begin{definition}
	$\phi(t)$ называется решением интегральное уравнения на  $<a,b>$
	\begin{enumerate}
		\item $\phi \in C(<a,b>)$
		\item $\Gamma_{\phi} \subset G$
		\item $\phi(t) \equiv x_0 + \int\limits_{t_0}^{t} f(s,\phi(s)) ds $ на $<a,b>$
	\end{enumerate}
\end{definition}
\begin{theorem}[Об эквивалентность задачи Коши и интегральных уравнений]
	$\phi(t)$ будет решением задачи коши на промежутке  $\iff$ она решение интегрального уравнения
\end{theorem}
\begin{proof}
	\begin{enumerate}
		\item
		      \begin{enumerate}
			      \item $\phi \in C^{1} \implies \phi \in C$
			      \item Все понятно
			      \item
			            \[
				            \dot{\phi}(t) \equiv f(t,\phi(t)) \implies
				            \int\limits_{t_0}^{t} \dot{\phi}(s) dt = \int\limits_{t_0}^{t} f(s,\phi(s)) ds
				            .\]
			            \[
				            \phi(t) - \phi(t_0) = \int\limits_{t_0}^{t} f(s,\phi(s)) ds
				            .\]
		      \end{enumerate}
		\item
		      $\phi \in C \implies f(s,\phi(s)) \in C \implies \int\limits_{t_0}^{t} f(s,\phi(s)) ds \in C^{1}  $
		\item Понятно
		\item
		      \[
			      \phi(t) \equiv x_0 + \int\limits_{t_0}^{t} f(s,\phi(s)) ds \implies \phi \in C^{1}
			      .\]
        \item 
            \[
            \phi(t) \equiv x_0 + \int\limits_{t_0}^{t}   f(s,\phi(s)) ds
            .\] 
            \[
                \dot{\phi}(x)  \equiv 0 + f(t,\phi(t))
            .\] 
            Начальное условие $\phi(t_0) = x_0$
	\end{enumerate}
\end{proof}
\section{Условие Липщица}
\[
f : G \to \mathbb{R}^{n}
.\] 
\[
G \subset \mathbb{R}^{n+1}
.\] 
\begin{definition}
    f удовлетворяет условию липшица по переменной x в области $G$  c константой L
    \[
    \forall  (t,x_1),  (t,x_2) \in G
    .\] 
    \[
    ||f(t,x_1) - f(t,x_2)|| \le  L||x_1-x_2||
    .\] 
\end{definition}
\subsection{Комментарий}
f растет не быстрее линейной функции
\begin{lemma}
    для f выполняется условие липщица $\iff$  $f \in C^{0}_{x}(G)$
\end{lemma}
\begin{proof}
    \[
    ||f(t,x_1) - f(t,x_2)|| < \epsilon
    .\] 
    \[
    \delta = \frac{\epsilon}{L}
    .\] 
    \[
    ||f(t,x_1) - f(t,x_2)|| \le  L||x-x_2|| < \delta L
    .\] 
\end{proof}
\begin{definition}[Равномерная непрерывность]
    \[
    \forall  ~ \epsilon > 0 \exists  \delta =\delta(\epsilon), \forall  (t,x_1),(t,x_2) \in G ||x_1-x_2|| <\delta
    .\] 
\end{definition}
\begin{theorem}
    Пусть $\exists  f'_{x} \in C(G) \iff ~ \forall $ шара из $G$, для f выполняется условие липщица в B по x
\end{theorem}
    \section{Теорема Пикара}
    \[
        \dot{x} = f(x)
    .\] 
    Нормальная система дифференциальных уравнений
    \[
    G \to \mathbb{R}^{n}
    .\] 
    $G$ область в  $R^{n+1}_{t,x}$
    \begin{theorem}[Пикара]
        Пусть $f \in C(G), f \in Lip_{x}(G)$, тогда
        \[
        \forall  (t_0,x_0) \in G
        .\] 
        Задача Коши 
        \[
            \begin{cases}
            \dot{x} = f(t,x)\\
            x(t_0) = x_0
            \end{cases}
        .\] 
        имеет единственное решение, определенное на отрезке Пеано
        \[
            I = [t_0-h,t_0+h]
        .\] 
     \end{theorem}
     \begin{zam}
         \[
         C^{1}_{x} \implies Lip_{x}
         .\] 
         В теореме Пикара можно писать 
         \[
         f \in C, f \in C^{1}_{x} (f'_{x} \in C)
         .\] 
         менее общая формулировка, но она удобна на практике
     \end{zam}
     \subsection{Построение отрека Пеано}
     $G$ - открытое множество $\implies$  $(t_0,x_0) \in G$ вместе с открытым шаром шаром $B = R_{r}(t_0,x_0)$ 
     Уменьшм $r$ , $\overline{B} \subset G$ (замкнутый шар). В $\overline{B}$ построим цилиндр
      \[
          P := \{(t,x) \mid |t-t_0| \le  a, ||x-x_0|| \le b\}
      .\] 
      Пусть $a,b$ малы  $P \subset \overline{B}$. $P$ ограниченно и замкнуто.
      По теореме Вейштрасса (любая непрерывная функция на компакте достигает наибольшего значения). 
      \[
          \exists  M = \max{||f(t,x)||}
      .\] 
      \[
          h := \min{a,\frac{b}{M}}
      .\] 
      Cчитаем, что $M \neq 0$, если $M = 0 \implies f = 0$ все решается очень просто.
       \[
           I := [t_0-h,t_0 + h]
      .\] 
      Отрезок Пеано определяется неоднозначно
      \subsection{Определение Пикаровских приближений}
      Заменим задачу коши на эквивалентное ей интегральное ураавнение
      \[
      x(t) = x_0 + \int\limits_{t_0}^{t} f(s,x(s))  ds
      .\] 
      Мы докажем, что именно интегральное уравние имеет единственное решение на отрезке Пеано $I$. Пикаровские приближения,  $\phi_0(t),\phi_{1}(t),\dots,\phi_{t}(t)$
      \[
      \phi_0(t) = x_0
      .\] 
      \[
      \phi_1(t) = x_0 + \int\limits_{t_0}^{t} f(s,\phi(x_0)(s))  ds
      .\] 
      \[
      \dots
      .\] 
      \[
      \phi_{k+1}(t) = x_0 + \int\limits_{t_0}^{t} f(s,\phi_{k}(s))  ds
      .\] 
      \subsection{Лемма}
      \begin{lemma}
          \begin{enumerate}
              \item Все пикаровские приближения определены на $I, \phi_{k} \in C(I)$
              \item $\Gamma_{\phi_{k}} \subset P$
          \end{enumerate}
      \end{lemma}
      \begin{proof}
          По индукци $k = 0$
           \[
          \phi_{0}(t) = x_0
          .\] 
          все очевидно.\\
          Переход $\phi_{k} \in C(i) $ и $\Gamma_{\phi_{k}} \subset P$
          \[
          \phi_{k + 1} := x_0 + \int\limits_{t_0}^{t}f(s,\phi_{k}(s)) ds
          .\] 
          \[
          t \in I \iff |t-t_0|\le h
          .\] 
          S между  $t,t_0$
          \[
          s \in I \implies
          .\] 
          \[
              \phi_{k}(s) \text{полн оgh}
          .\] 
          \[
              (s,\phi_{k}(s)) \in P \forall  s\in I
          .\] 
          \[
          f(s,\phi_{k}(s))
          .\] 
          определена при 
          \[
          \forall  s \in I \implies \int\limits_{t_0}^{t} f 
          .\] 
          \[
              \forall  t \in I \implies \phi_{k +1}(t) \text{опредлена} \forall t\in I
          .\] 
          Пункт 1 доказан
          \[
              (t,\phi_{k+1}) \in P \forall  t \in I
          .\] 
          \[
          \begin{cases}
              |t - t_0| \le  a\\
              ||\phi_{k + 1} (t) - x_0|| \le  b
          \end{cases}
          .\] 
          Проверим каждое из условий
          \begin{enumerate}
              \item 
                  $t \in I \iff |t-t_0|\le h \le a$ 
              \item $||\phi_{k+1}(t) - x_0|| = || \int\limits_{t_0}^{t} f(s,\phi_{k}) ds||\le  \int\limits_{t_0}^{t} ||f(s,\phi_{k}(s))||ds \le |\int\limits_{t_0}^{t} ||f(s,\phi_{k}(s))||ds| \le  M | \int\limits_{t_0}^{t} ds = M|t-t_0|\le  Mh = b $
          \end{enumerate}
      \end{proof}
\end{document}
