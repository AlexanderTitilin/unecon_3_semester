\documentclass[14pt]{extarticle}
\usepackage{fontspec}
\usepackage[russian, english]{babel}
\setmainfont{Times New Roman}
\usepackage{amssymb}
\usepackage{setspace}
\onehalfspacing
\usepackage{amsmath}
\usepackage{amsthm}
\usepackage{listings}
\usepackage{indentfirst}
\usepackage{hyperref}
\setlength{\parindent}{1.25cm}
\usepackage[right=10mm,left=30mm,top=20mm,bottom=20mm]{geometry}
\newtheorem{theorem}{Теорема}
\newtheorem{definition}{Определение}
\newtheorem{corollary}{Следствие}[theorem]
\newtheorem{lemma}[theorem]{Лемма}
\title{Лекции по дифференциальным уравнениям}
\author{}
\date{}
\begin{document}
\maketitle
\section{Список литературы}
\begin{enumerate}
	\item Филиппов Лекции по обыкновенным ОДУ.
	\item Филиппов Сборник задач по дифференциальным уравнениям
	\item Петровский Лекции по ОДУ
	\item Самойленко, Кривошея, Перестрюк. ДУ. Примеры и задачи.
	\item Антидемидович
\end{enumerate}
\section{ДУ первого порядка}
\[
	f(x,y(x),y'(x)) = 0
	.\]
\begin{enumerate}
	\item x - независимая переменная
	\item $y(x)$ неизвестная функция
	\item  $y'(x)$ ее производная
\end{enumerate}
Решить ДУ -- найти $y(x)$
\section{Примеры}
\begin{enumerate}
	\item $y'(x) = y(x)$
	\item Найти  $y(x) ~$  $y'(x) = f(x)$
\end{enumerate}
\section{ДУ n-го порядка}
\[
	f(x,y,y',\dots,y^{(n)})
	.\]
\section{Ду разрешимое относительно производных}
\[
	y' = f(x,y)
	.\]
Будем заниматься только такими уравнениями
\section{Модель экспонециального роста (эпидемий)}
\[
	x(t) - \text{число бактерий}
	.\]
\[
	\dot{x}(t) ~ x(t)
	.\]
\[
	\dot{x} = kx
	.\]
\[
	x = C e^{kt}, C \in \mathbb{R}
	.\]
Модель роста с учетом эффекта насыщения называется логистической моделью.\\
Пусть $N$ -- максмимальное количество особей.
\[
	\dot{x} = kx (N - x)
	.\]
\[
	\dot{x} ~\text{максимальная, при } x = \frac{N}{2}
	.\]
\section{Интегрируемые дифференциальные уравнения первого порядка}
Знакомимся c уравнениями, которые можем решить в явном виде.
\subsection{Общие определения}
\begin{definition} \label{diff1}
	ДУ 1 порядка, разрешенным относительно производной называется уравнение вида
	\[
		y' =  f(x,y)
		.\] \label{1}
	где $x$ независимая переменная,  $y(x)$ искомая функция. Решить ДУ \ref{1} $\iff$
	найти $y(x)$. Будем считать $f : G \to \mathbb{R}, G \subset \mathbb{R}^2$, $G$ - связное и открытое множество\\
	В этой глае  $f$ будет элементарной (школьной) функцией. Так же считается, что $G$ область опредления уравненияю
\end{definition}
\begin{definition}
	Функция $y = \phi(x)$ называется решением ДУ  \ref{diff1} на промежутку $<a,b>$,
	если выполнены 3 условия
	\begin{enumerate}
		\item $\phi \in C^{1}(<a,b>)$ дифференцируема один раз на $<a,b>$
		\item $\text{Граф}~\phi := \{(x,(\phi(x)) \mid x \in <a,b>\} \subset G$
		\item $\phi'(x) \equiv f(x,\phi(x))$ на  $<a,b>$
	\end{enumerate}
\end{definition}
\begin{definition}
	График решения называют интегральной кривой ДУ(\ref{diff1}).
\end{definition}
\subsubsection{Самый простой пример}
\[
	y' = f(x)
	.\]
\[
	y =  \int\limits_{x_0}^{x} f(s) ds + C, C \in \mathbb{R}
	.\]
Решений бесконечно много. Общее решение ДУ(\ref{diff1}) имеет вид
\[
	y = \phi(x,C), C \in \mathbb{R}
	.\]
\subsubsection{Задача Коши}
Задано начальное значение решения.
\begin{definition} \label{kosh}
	Задача Коши -- называется задача следущего вида
	\begin{equation}
		\begin{cases}
			y' = f(x,y) \\
			y(x_0) = y_0
		\end{cases}
	\end{equation}
	где, $(x_0,y_0) \in G$
	Надо найти решения, которые удовлетворяют данному условию. Начальное условие означает, что график решения проходит через $(x_0,y_0)$
\end{definition}
\begin{definition}
	Говорят, что задача Коши \ref{kosh} имеет единственное решение или $(x_0,y_0)$
	есть точка единственности, если существует окрестность $(x_0,y_0) \in U$, такая что для любых двух интегральных кривых $\Gamma_1,\Gamma_2$, проходящих через точку $(x_0,y_0)$ выполняется $\Gamma_1 \cap U = \Gamma_2 \cap U$ (все интегральные кривые, проходящие $(x_0,y_0)$ совпадают в окрестности $U$). На языке епсилон-дельта
	\[
		\exists (x_0 -\delta,x_0 + \delta) \forall   \phi_1 , \phi_2 ~\text{Решения \ref{kosh}}~ \phi_1 \equiv \phi_2~  \text{на}~ (x_1 -\delta , x_0 +\delta)
		.\]
	Если имеет не единственное решение то $\forall U$ открестности найдутся 2 интегральные кривые различаются в окрестности.
\end{definition}
В дифференицальных уравнениях единственность понимаемся в локальном смысле.
\subsubsection{Пример Пеано}
\[
	\begin{cases}
		y' = 3y^{\frac{2}{3}} \\
		y(0) = 0
	\end{cases}
	.\]
    Преверяем 2 решения 
     \[
    \phi_1 (x) = 0
    .\] 
    \[
    \phi_2(x) = x^{3}
    .\] 
    В любой окрестности  $\phi_1 \neq \phi_2$
\subsection{Уравнения с разделяющимся переменными}
\begin{definition}\label{urp}
    УРП -  уравнение вида $y' = f(x)g(y)$.
    Всегда предполагаем, что
    \begin{enumerate}
        \item f $\in C<a,b>$
        \item  $g \in C<\alpha,\beta>$
    \end{enumerate}
    \[
    G =  <a,b> \times <\alpha,\beta>
    .\] 
\end{definition}
Как это решать придумал Якоб Бернулли.
\subsubsection{Неформальный рецепт}
\[
\frac{dy}{dx} = f(x)g(y)
.\] 
\[
\frac{dy}{g(y)} = f(x) dx
.\] 
\[
\int \frac{dy}{g(y)} = \int f(x) dx + C
.\] 
\subsubsection{Нормальное доказательство}
\begin{theorem}
    \[
    f \in C(<a,b>)
    .\] 
    \[
    g \in C(<\alpha,\beta>)
    .\] 
    \[
    g(y) \neq 0 \forall  y \in <\alpha,\beta>
    .\] 
    \begin{enumerate}
        \item $H(y) $ первообразная $\frac{1}{g(y)}$ 
        \item $F(x)$ первообразная  $f(x)$
    \end{enumerate}
    \begin{enumerate}
        \item Тогда формула из неформального рецепта задает общее решение  и только его
        \item $G = <a,b> \times <\alpha,\beta>$ -- область существоавания и единственности, для любой точки  $(x_0,y_0) \in G$ задача задача Коши имеет единственное решение.
            И это решение задается формулами.
            \[
            H(x) = F(x) + H(y_0) - F(x_0)
            .\] 
    \end{enumerate}
\end{theorem}
\begin{proof}
    Заметим, что  $H'(y) = \frac{1}{g(y)} \neq 0$ $\implies \exists  H^{-1}$ 
    Тогда получаем
    \[
    y = H^{-1} (F(x) + C)
    .\] 
    Формула 2 задает функции вида $y  = y(x)$
     \begin{enumerate}
        \item  Пусть $y = y(x)$ есть решение уравнения. Покажем, что оно вкладывается в формулу.  $\exists  C_0 : H(y(x)) - F(x) + C_0$
            \[
            \frac{d}{dx} (H(y(x)) - F(x)) \equiv 0
            .\] 
            \[
                H'(y(x)) y'(x) - F'(x) = 0
            .\] 
            \[
            \frac{1}{g(y(x))} * f(x) g(y(x)) - f(x) \equiv 0
            .\] 
        \item Теперь обратно
            \[
            H(y(x)) \equiv F(x) + C
            .\] 
            Продифференцировали
            \[
            H'(y(x)) * y'(x) \equiv F'(x)
            .\] 
    \end{enumerate}
    Берем произвольную точку из области $(x_0,y_0) \in G$
    \[
    C \equiv H(y(x)) - F(x)
    .\] 
    Подставим начальные условия из задачи коши
    \[
    C = H(y(x_0)) - F(x_0) = H(y_0) - F (x_0)
    .\] 
    Тоесть для любой точки из $G$ можно определить единственным образом
\end{proof}
Ответ писать, надо даже если обратная функция не выражается в элементарных вроде $H(y) = y + \arctg{(y))}$, эта хрень считается ответом
\[
H(y) - F(x) = C
.\] 
\[
U(x,y) = H(y) - F(x)
.\] 
$U$ -- интеграл ДУ
\begin{definition}
    $U(x,y)$ называется интегралом ДУ, если выполняются следущие аксиомы (свойства)
    \begin{enumerate}
        \item $U \in C^{1}$ 
        \item $U'_{y} \neq 0$ (производная по y не ноль)
        \item U обращается в константу при подставлении решения ДУ.
    \end{enumerate}
\end{definition}
Все свойства выполняются для $U(x,y)$
\begin{definition}[Линия уровня]
    \[
        U^{-1}(c) = \{(x,y) \mid U(x,y) = c\}
    .\] 
\end{definition}
Теперь рассмотрим случай, когда $g(y) = 0$
Пусть  $g(a) = 0 \implies$ стационарное решение
 \[
a' \equiv f(x)(g(a))
.\] 
\begin{theorem}
    Пусть $f \in C(<a,b>), g \in C^{1}(<\alpha,\beta>)$
     Тогда все решения уравнения задаются совокупностью
     \[
     H(y) = F(x) + c
     .\] 
     \[
         g(y) = 0 ~\text{совокупность всех стационарных решений.}
     .\] 
\end{theorem}
Эту хрень не доказываем, так как будет следовать из теоремы Пикара.
\begin{enumerate}
    \item Пример Пеано
        \[
        y' = 3 y^{\frac{2}{3}}
        .\] 
        \[
        y = 0
        .\] 
        решение
        \[
        \int \frac{y' dy}{3y^{\frac{2}{3}}} = \frac{1}{3} *x^{1/3} * 3 + c = x^{\frac{1}{3}} + c
        .\] 
        \[
        y^{\frac{1}{3}} = x + c
        .\] 
        \[
        y = (x + c)^{3}
        .\] 
        \[
        \left[
        \begin{gathered}
            y = 0\\
            y = (x + c)^{3}
        \end{gathered}
        \right
        .\] 
В точках $y = 0$ нарушается единственность, решения можно склеивать, получать новые, не из совокупности
 \[
y = 
\begin{cases}
    (x - c_1)^{3} ,x <  c_1\\
    0 , c_1 \le  x \le  c_2\\
    (x - c_2)< x > c_2
\end{cases}
.\] 
$g(y)$ не дифф в 0, условия теоремы 2 не выполняются.
\item
     \[
    y' = y
    .\] 
    $y = 0$ решение
     \[
         \int \frac{y'}{y} dy = \ln{y} +  c
    .\] 
    \[
        \ln{(|y|)} = x + c
    .\] 
    \[
    |y| = e^{x + c}
    .\] 
    \[
        \left[
            \begin{gathered}
                y = 0\\
                y = \pm c^{x}y e^{c}
            \end{gathered}
    \right
    .\] 
    \[
    c_1 = \pm e^{c}
    .\] 
    \[
        \left [
            \begin{gathered}
                y = 0\\
                y = C_1 e^{x}, \forall  c_1 \neq 0
            \end{gathered}
        \right
    .\] 
    единственность не нарушена
\end{enumerate}
\end{document}
